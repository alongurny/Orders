% I want to write a letter about proving things of ordinals.

\documentclass{article}

\usepackage{amsmath}
\usepackage{amssymb}
\usepackage{amsthm}

% Theorems
\newtheorem{theorem}{Theorem}
\newtheorem{corollary}{Corollary}
\newtheorem{lemma}{Lemma}
\newtheorem{proposition}{Proposition}

% Definitions
\newtheorem{definition}{Definition}
\newtheorem{notation}{Notation}
\newtheorem{notations}{Notations}

% Delimiters
\newcommand{\parens}[1]{\left( {#1} \right)}
\newcommand{\brackets}[1]{\left[ {#1} \right]}
\newcommand{\braces}[1]{\left\{ {#1} \right\}}
\newcommand{\iv}[1]{\left[ #1 \right)}

% Macros
\newcommand{\setcomp}[1]{\braces{#1}}
\newcommand{\ii}[1]{{\iota}_{#1}}
\newcommand{\kk}[1]{{\kappa}_{#1}}
\newcommand{\pp}[1]{{\pi}_{#1}}
\newcommand{\CC}{\mathcal{C}}

\begin{document}

\section{Ordinals}

In this section the logic is always MSO, and the structures are ordinals.

\begin{notations}
    Let $\ii{\alpha} \parens{X}$ denote
    that the set $X$ is a left-closed, right-open interval isomorphic to $\omega^\alpha$.

    Let $\pp{\alpha} \parens{X}$ denote
    that the set $X$ is a finite union of intervals and is isomorphic to $\omega^\alpha$.

    Let $\kk{\alpha} \parens{x}$ denote
    that $x$ is a multiple of $\omega^\alpha$.
\end{notations}


\begin{proposition}
    Let $\beta = \omega^\alpha$ be an ordinals
    and $x, y$ some ordinals.

    Then,
    $\iv{x, y}$ is isomorphic to $\omega^\alpha$
    iff there exists some $z$ which is a multiple of $\omega^\alpha$,
    such that $y$ is the least greater multiple of $\omega^\alpha$ s.t. $z \le x < y$.
\end{proposition}

\begin{proof}
    Divide $x = \gamma \cdot \beta + \delta$ where $\delta < \beta$, and as $x + \beta = y$,
    $y = \gamma \cdot \beta + \delta + \beta$.

    Now, as $\omega^\alpha$ is additively indecomposable, $\delta + \beta = \beta$,
    so $y = \gamma \cdot \beta + \beta = \parens{\gamma + 1} \cdot \beta$.
    Choosing $z = \gamma + 1$, indeed $y$ is the least greter multiple of $\omega^\alpha$,
    and $z \le x < y$, as we need.
\end{proof}

\begin{corollary}
    Over ordinals, $\ii{\alpha}$ is expressible via $\kk{\alpha}$.
\end{corollary}

\begin{proof}
    By the previous lemma,
    $$\ii{\alpha} \parens{X}
    \iff
    \exists x, y, z. 
           X = \brackets{x, y}
    \wedge \kk{\alpha} \parens{z}
    \wedge y = \min_b \setcomp{a : z < b \wedge \kk{\alpha} \parens{b}}$$
\end{proof}

\begin{proposition}
    Let $\beta = \omega^\alpha$.
    Then, a set $X$ satisfies $\pp{\alpha}$ iff it is a finite sum
    $X = \sum_{i=1}^m X_i$ where for $i<m$, $X_i$ is a strict prefix
    of some interval satisfying $\ii{\alpha}$, and
    for $i = m$, $X_i$ satisfies $\ii{\alpha}$.
\end{proposition}

\begin{proof}
    Suppose $X_i$ is indeed an interval which is less then $\omega^\alpha$ for $i < m$
    and equals $\omega^\alpha$ for $i = m$.
    Since $\omega^\alpha$ is additively indecomposable,
    the sum $\sum_{i=1}^m X_i$ is indeed $\omega^\alpha$.
    Therefore, $X$ is a disjoint sum of intervals isomorphic to $\omega^\alpha$.

    For the other direction,
    suppose $X$ is a disjoint sum of intervals isomorphic to $\omega^\alpha$.
    Let the intervals be $X_i$ for $i = 1, ..., m$.
    Since $\omega^\alpha$ is additively indecomposable,
    nessessarily $X_i < \beta$ for $i < m$ and $X_m = \beta$.
    
    Also, note that for $i < m$, the interval from the left end of $X_i$ to the
    right end of $X_m$ is at least $\omega^\alpha$,
    since it contains $X_i \cup X_m$ which is isomorphic to $\omega^\alpha$,
    and thus there is an interval in which $X_i$ is contained
    which is isomorphic to $\omega^\alpha$.
\end{proof}

\begin{corollary}
    Over ordinals, $\pp{\alpha}$ is expressible via $\ii{\alpha}$.
\end{corollary}

\begin{proof}
    By the previous corollary,
    it is enough to formulate that there
    exists a finite set of left endpoints and right endpoints,
    such that except for the last ones,
    All the intervals are strictly the prefix of
    an interval isomorphic to $\omega^\alpha$,
    and such that the last interval is
    isomorphic to $\omega^\alpha$.

    \begin{equation}
        \begin{split}
            \pp{\alpha} \parens{X}  \iff & \exists L \exists R.
        \\        & finite \parens{L}
        \\ \wedge & finite \parens{R}
        \\ \wedge & \min \parens{L \cup R} \in L
        \\ \wedge & \max \parens{L \cup R} \in R
        \\ \wedge &
        \forall x, y \in L. x < y \implies \exists z \in R. x < z < y
        \\ \wedge &
        \forall x, y \in R. x < y \implies \exists z \in L. x < z < y
        \\ \wedge &
        {
            \forall x \in L - \braces{\max L}
            \forall y \in R - \braces{\max R}
            \exists z. y < z \wedge \ii{\brackets{x, z}}
        }
        \\ \wedge & \ii{\braces{\max L, \max R}}
        \end{split}
    \end{equation}

\end{proof}

\begin{proposition}
    Let $\omega^\alpha$ be any ordinal.
    Let the Cantor normal form be
    $$\beta = \sum_{i=1}^k \beta_i$$ where $\beta_i = \omega^{\alpha_i}$.

    Then, a set $X$ satisfies $\pp{\alpha}$ iff it is an ordered sum
    $X = \sum_{i=1}^k X_i$ where $X_i$ satisfies $\pp{\alpha_i}$.
\end{proposition}

\begin{proof}
    Obviously a disjoint sum of such
    intervals satisfies $\pp{\alpha}$.

    For the other direction, suppose $X$ is a union
    of closed intervals and
    is isomorphic to $\omega^\alpha$.
    Let $\varphi : X \to \beta$ be the isomorphism map.
    Recall that $\beta = \sum_{i=1}^k Y_i$ where
    $Y_i \cong \beta_i$.
    
    Let $X_i := \varphi^{-1}(Y_i)$ for $i = 1, ..., k$.
    Then it can easily be seen that $X_i$ is
    a disjoint sum of closed intervals.
\end{proof}

\begin{corollary}
    Let $\omega^\alpha$ be any ordinal.
    Then $\pp{\alpha}$ is expressible via $\pp{\alpha_1}, ..., \pp{\alpha_k}$
    for some ordinals
    $\beta_1 = \omega^{\alpha_1}, ..., \beta_k = \omega^{\alpha_k}$.
\end{corollary}

\begin{proof}
    This is pretty much the same trick as the previous corrollary,
    but is very tedious to write down.
\end{proof}

\begin{theorem}
    Let $\omega^\alpha$ be any ordinal.

    Over ordinals, $\pp{\alpha}$ is expressible via
    $\kk{\alpha_1}, ..., \kk{\alpha_k}$
    where $\beta_1 = \omega^{\alpha_1}, ..., \beta_k = \omega^{\alpha_k}$.
\end{theorem}

\begin{proof}
    Combine the three previous corollaries.
\end{proof}

\begin{definition}
    Let $\beta > 0$ be a nonzero ordinal.

    Let $\xi$ be any ordinal.
    Let $\gamma$ and $\delta$ be the unique ordinals
    such that $\xi = \gamma \beta + \delta$.

    Then, define $\xi / \beta := \gamma$ and $\xi \% \beta := \delta$.

    That is, $\xi / \beta$ is the quotient of $\xi$ by $\omega^\alpha$, "rounded down".

    Now, suppose $\CC$ be a class of ordinals.

    Define $\CC_{/\beta} := \setcomp{\xi / \beta : \xi \in \CC}$
    Define $\CC_{\% \beta} := \setcomp{\xi \% \beta : \xi \in \CC}$

\end{definition}

\begin{theorem}
    Let $\CC$ be a class of ordinals,
    and let $\beta_1=\omega^{\alpha_1}, ..., \beta_k=\omega^{\alpha_k}$ be ordinals,
    where $\alpha_1 \ge \cdots \ge \alpha_k$, that is, $\beta_k \mid \cdots \mid \beta_1$.
    
    Then the logic $MSO \brackets{\kk{\alpha_1}, ..., \kk{\alpha_k}}$
    is decidable over $\CC$,

    using oracles to decide $MSO \brackets{\kk{\alpha_1}, ..., \kk{\alpha_{k-1}}}$
    over $\omega^\alpha$ and over $\CC_{\% \beta_k}$ and
    decide $MSO$ over $\CC_{/\beta_k}$.
\end{theorem}

\begin{proof}
    Denote $\beta := \beta_k$.

    By the composition theorem, given $\varphi$ of quantifier rank $\rho$,
    we can compute $\psi'$,
    with free variables $\braces{X_\tau}_\tau$,
    where $\tau$ is a $\rho$-type.

    such that for any $\xi = \sum_{i \in \xi / \beta} \xi_i + \delta$,
    for $\delta < \beta$,
    
    (denoting $\xi_{\xi / \beta} := \delta$ we get $\xi = \sum_{i \in \xi / \beta + 1} \xi_i$),
    
    $$
    \xi \vDash \varphi \iff \xi/\beta + 1, Q_1, ..., Q_r\vDash \psi'
    $$

    where $Q_\tau = \setcomp{\xi_i \vDash \tau : i \in \xi / \beta + 1}$

    Now, since the $Q_\tau$ are all (except maybe the last index) or
    nothing (except including maybe last the index),
    we can actually describe them with two propositional/boolean variables.
    That is, we can replace each time $X_\tau$ appears in $\psi'$,
    by $\setcomp{i < \max : y_\tau} \cup \setcomp{\max : z_\tau}$
    where $y_\tau$ and $z_\tau$ are two new propositional variables, and then:

    $$
    \xi \vDash \varphi \iff \xi/\beta + 1, Y_1, Z_1, ..., Y_r, Z_r\vDash \psi''
    $$

    where $Y_\tau := \beta \vDash \tau$,
    and $Z_\tau := \delta \vDash \tau$,
    and $\psi''$ is $\psi'$ after the replacement.

    Therefore, given a class $\CC$ of ordinals,
    we first determine for each $\tau$ what $Y_\tau$ is,
    using the fact that $\omega^\alpha$ is decidable over $MSO \brackets{\kk{\alpha_1}, ..., \kk{\alpha_{k-1}}}$,
    and that over $\omega^\alpha$ we can replace $\kk{\alpha_k}$ with $=0$.

    Now, we calculate all the possible values of $Z_\tau$: for that,
    we check if $\tau$ holds in all of $\CC_{\% \beta}$, and if
    $\neg \tau$ holds in all of $\CC_{\% \beta}$.

    This is possible since $\CC_{\% \beta}$ is decidable over $MSO \brackets{\kk{\alpha_1}, ..., \kk{\alpha_{k-1}}}$,
    and again, we can replace $\kk{\alpha_k}$ with $=0$.

    Finally, for each possible combination of values
    $Y_1, Z_1, ..., Y_r, Z_r$, we calculate whether $\psi''$ holds over
    $\CC_{/\beta}$. This is possible since $MSO$ is decidable over $\CC_{/\beta}$.
    

    Then, suppose that $\varphi$ holds over all $\xi \in \CC$.

    Then, it must be the case that
    $\psi''$ holds for any such combination. 

    And also for the contrary, every ordinal $\xi \in \CC$ gives rise to some combination.
\end{proof}

\begin{corollary}
    Suppose in the previous theorem that $\CC$ is either:
    \begin{enumerate}
        \item a \textbf{single countable} ordinal: $\CC = \setcomp{\xi}$
        \item the class of \textbf{all countable} ordinals
    \end{enumerate}
    Then $MSO \brackets{\kk{\alpha_1}, ..., \kk{\alpha_k}}$ is decidable.
\end{corollary}

\begin{proof}
    For a single countable ordinal $\xi$:
    $\omega^\alpha$ over $MSO \brackets{\kk{\alpha_1}, ..., \kk{\alpha_{k-1}}}$
    is decidable by induction on $k$ (for the induction step),
    since $MSO$ is decidable over a single countable ordinal (for the base case)

    $\CC_{/\beta} = \setcomp{\xi/\beta}$ again is decidable for the same reason.

    and $\CC_{\% \beta} = \setcomp{\xi \% \beta}$ 
    is a single countable ordinal and so $MSO$ is decidable over it.


    Now, suppose $\CC$ is the class of all countable ordinals,
    then $\CC_{/ \beta} = \CC$,
    so $MSO \brackets{\kk{\alpha_1}, ..., \kk{\alpha_{k-1}}}$ is decidable
    over it by induction on $k$, or since $MSO$ is decidable over
    the class of all countable ordinals (in the base case),

    and $\CC_{\% \beta} = \setcomp{\delta : \delta < \beta}$.
    $MSO$ is decidable over it, since a formula
    $\varphi$ is true over $\setcomp{\delta : \delta < \beta}$
    iff $\varphi \vee \exists X. \kk{\alpha}$ is true over the class
    of all countable ordinals, and we have already established that
    this $MSO \brackets{\kk{\alpha}}$ is decidable over a single countable ordinal.
\end{proof}

\begin{theorem}
    Under the same conditions, $MSO \brackets{\pp{\alpha_1}, ..., \pp{\alpha_k}}$ is decidable.
\end{theorem}

\begin{proof}
    Combine all the results we have had so far.
\end{proof}

\end{document}
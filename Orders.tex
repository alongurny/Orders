% I want to write a letter about proving things of orders.

\documentclass{article}

\usepackage{amsmath}
\usepackage{amssymb}
\usepackage{amsthm}

% Theorems
\newtheorem{theorem}{Theorem}
\newtheorem{corollary}{Corollary}
\newtheorem{lemma}{Lemma}
\newtheorem{proposition}{Proposition}
\newtheorem{conjecture}{Conjecture}

% Definitions
\newtheorem{definition}{Definition}
\newtheorem{notation}{Notation}
\newtheorem{notations}{Notations}

% Delimiters
\newcommand{\parens}[1]{\left( {#1} \right)}
\newcommand{\brackets}[1]{\left[ {#1} \right]}
\newcommand{\braces}[1]{\left\{ {#1} \right\}}

% Macros
\newcommand{\setcomp}[1]{\braces{#1}}
\newcommand{\rankm}{\mathbf{rank^-}}
\newcommand{\rankp}{\mathbf{rank}}

% Letters
\newcommand{\LL}{\mathcal{L}}
\newcommand{\agemo}{-\omega}
\newcommand{\otp}{\mathbf{otp}}

% Sets
\newcommand{\ZZ}{\mathbb{Z}}

\begin{document}

\section{Orders}

In this section the logic is always MSO, and the structures are orders.
When we mention a linear order $\LL$, we assume implicitly that it is countable.
When we mention an ordinal $\alpha$, we assume implicitly that it is a limit ordinal.
We shall write this as CLO (countable linear order).

% Definition of $\sim_\alpha$
\begin{definition}
    Let $\LL$ be a CLO.
    We define two relation $\rankm \LL \le \alpha$ and $\rankp \LL \le \alpha$
    by transfinite induction on $\alpha$ as follows:

    $\rankm \LL \le 0$ iff $\LL$ is $1$.

    $\rankp \LL \le \alpha$ iff $\LL$ is a finite sum of CLOs $\LL_i$ such that
    $\rankm \LL_i < \alpha$ for all $i$.

    For $\alpha>0$, $\rankm \LL \le \alpha$ iff $\LL = \sum_{i \in I} \LL_i$
    where $I$ is either finite or $\pm \omega$ or $\eta$.
    and $\rankp \LL_i < \alpha$ for all $i \in I$.

    Now, we define $\rankm \LL$, if it exists, as the minimal $\alpha$
    such that $\rankm \LL \le \alpha$, and similarly we (partially) define
    $\rankp \LL$.
\end{definition}

% Closesness under finite sums
\begin{lemma}
    For every $\alpha$,
        \begin{enumerate}
            \item
            Let $\LL_1$ and $\LL_2$ be CLOs such that $\rankp \LL_i = \alpha_i$.

            Then $\rankp \parens{\LL_1 + \LL_2} = \max \setcomp{\alpha_1, \alpha_2}$.
            \item
            Let $\LL_1$ and $\LL_2$ be CLOs such that $\rankm \LL_1 = \rankm \LL_2 = \alpha$.

            Then $\rankm \parens{\LL_1 + \LL_2} = \alpha + 1$.
        \end{enumerate}
\end{lemma}

% Proof
\begin{proof}
    This is obvious from the definition.
\end{proof}

% CLO good 2-colorability
\begin{definition}[Good 2-coloring]
    Let $\LL$ be a CLO. A good $2$-coloring of $\LL$ is a coloring of $\LL$ with two colors, such that:
    * If there is only a single endpoint, it is red.
    * If there are two endpoints, the leftmost one is red.
\end{definition}

% Existence and uniqueness of good 2-coloring
\begin{lemma}[Existence and uniqueness]
    Every CLO has a good $2$-coloring.
    All such $2$-colorings are isomorphic.
\end{lemma}

\begin{proof}
    Every rank-$1$ equivalence class is obviously $2$-colorable this way, since it either has an endpoint or is isomorphic
    to $\eta := \agemo + \omega = \otp \parens{\ZZ}$

    The coloring is obviously unique, except for the case of $\eta$, where the two possible good colorings are isomorphic.

    Since the rank-$1$ equivalence classes have no edges between them, the proof is done. 

    \qed
\end{proof}

\begin{lemma}
    The fact that $R$ is a good $2$-coloring is expressible in MSO over CLOs. 
\end{lemma}

\begin{proof}
    Obvious.

    \qed
\end{proof}

% The relation $\sim_\alpha$
\begin{definition}
    Let $\LL$ be a CLO and let $\alpha$ be an ordinal.
    We define the relation $\sim_\alpha$ on $\LL$ as follows:
    $x \sim_\alpha y$ iff $\rankp (x, y) < \alpha$
\end{definition}

\begin{lemma}
    Let $\alpha$ be a limit ordinal.
    Then, the following are equivalent:
    \begin{enumerate}
        \item $x \sim_\alpha y$
        \item the interval $(x, y)$ is of $\rankp < \alpha$
        \item the interval $[x, y)$ is of $\rankp < \alpha$
        \item the interval $(x, y]$ is of $\rankp < \alpha$
        \item the interval $[x, y]$ is of $\rankp < \alpha$
    \end{enumerate}
\end{lemma}

\begin{proof}
    Follows easily from the definition of $\rankp$.
\end{proof}

\begin{lemma}
    The relation $\sim_\alpha$ is an equivalence relation.
\end{lemma}

\begin{proof}
    Reflexivity and symmetry are obvious.
    Trnasitivity follows easily from the fact that $\alpha$ is a limit ordinal and the previous lemma.
    \qed
\end{proof}

\begin{definition}
    Let $\LL$ be a CLO together with a good coloring $C$ of $\LL / \sim_\alpha$, from which
    there is an induced coloring of $\LL$ (which is valid with regard to $\alpha$-classes).

    We denote by $R_C \parens{x}$ the formula expressing the fact that
    the $\alpha$-class of $x$ is red in $C$.
\end{definition}

\begin{lemma}
    An interval $X$ is of $\rankp < \alpha$ iff it is contained in an equivalence class of $\sim_\alpha$.
\end{lemma}

\begin{proof}
    Both directions are obvious. 

    \qed
\end{proof}

\begin{lemma}
    $X$ is an equivalence class of $\sim_\alpha$ iff $X$ is a maximal subset
    such that it is connected and every subinterval of $X$ is of $\rankp < \alpha$.
\end{lemma}

\begin{proof}
    It is enough to show that every set with this property is contained in an equivalence class of $\sim_\alpha$,
    and that the equivalence classes possess this property.

    Clearly, every nonempty set $X$ with this property is contained in a single equivalence class: 
    Let $x \in X$ be a specific choice, and let $y \in X$ be another point.
    Then $[x, y]$ is contained in $X$ since $X$ is connected, and therefore is of $\rankp < \alpha$.
    Therefore, $x \sim_\alpha y$.

    That is, $X = [x]_{\sim_\alpha}$.

    Now, let $X = [x]_{\sim_\alpha}$.
    Since $\alpha$ is a limit ordinal, every subinterval of $X$ is of $\rankp < \alpha$.
    And also clearly it is connected.

    \qed
\end{proof}

\begin{corollary}
    Let $\LL$ be a CLO and let $C$ be a good coloring of $\LL / \sim_\alpha$.
    The fact that $X$ is an interval of $\rankp < \alpha$ is expressible in MSO over CLOs using $R_C$.
\end{corollary}

\begin{proof}
    Since an $\alpha$-class of $\LL$ is just a maximal homochromatic connected subset this is clear from the previous lemma.

    \qed
\end{proof}

\begin{theorem}
    MSO extended by $R_C$ is decidable over all CLOs with a good coloring.
\end{theorem}

\begin{proof}
    Let $\LL$ be a CLO of rank $\nu$ and let $\alpha$ be a limit ordinal.

    Let $\gamma := \LL / \sim_\alpha $. Then $\gamma$ is of rank $\nu - \alpha$.

    Let $C$ be a good coloring of $\gamma$.

    Then we can write $\LL = \sum_{i \in \gamma} L_i$ where each $L_i$ is an $\alpha$-class.
    
    Moreover, over each $L_i$, the coloring is monochromatic.

    Let $\varphi$ be a formula of qd rank $r$ over $R_C$.
    
    Then, by the composition theorem, there exists a formula $\psi$ such that

    $$
    \LL \vDash \varphi
    \iff
    \gamma, \setcomp{Q_\tau}_\tau \vDash \psi 
    $$

    where $Q_\tau := \setcomp{L_i : L_i \vDash \tau}$, where $\tau$ is an $r$-rank.

    To decide $\varphi$ over all $\LL$, we need to decide $\psi$ over all the possible combinations of
    $\gamma$ and $\setcomp{Q_\tau}_\tau$ that arise from $\LL$.

    Now clearly, the possible values of $\gamma$ are all the CLOs.
    
    Now, we need to calculate all the values of $\tau$ for which there exists some equivalence class of some $\LL$
    that satisfies $\tau$.
\end{proof}


\end{document}
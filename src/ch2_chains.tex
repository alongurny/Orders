\section{Linear Orders}

% Definition of a (labeled) linear order
\begin{definitions}[(Labeled) Linear Order]
  A \emph{(labeled) linear order} a (labeled) preorder which is symmetric and total.
\end{definitions}

% Definition of a property
\begin{definition}[Property of linear orders]
  A \emph{property} $\pp$ of linear orders is a class of \emph{labeled} linear orders which
  is closed under isomorphism.
\end{definition}

% Definition of an interval, subintervals and bounded subintervals
\begin{definition}{Subintervals}
Let $M$ be a linear order,
and let $x, y \in M$, such that $x \le y$.

Then we define the \emph{bounded subintervals} $[x, y]$,
$(x, y]$, $[x, y)$ and $(x, y)$ as usual.

We also define the \emph{semi-bounded subintervals} $(-\infty, x]$,
$[x, \infty)$, $(-\infty, x)$ and $(x, \infty)$ as usual.

We also define \emph{the unbounded subinterval} $(-\infty, \infty)$ as the whole linear order $M$,
as usual.

A \emph{subinterval} is either 
a bounded subinterval, a semi-bounded subinterval or the unbounded subinterval.

If $x > y$ then we define the intervals as follows:
\begin{align*}
  [x, y] & := [y, x] \\
  (x, y] & := (y, x] \\
  [x, y) & := [y, x) \\
  (x, y) & := (y, x)
\end{align*}

\end{definition}

% Definition of left/right/bi-directionally cofinal
\begin{definition}
  Let $M$ be a linear order.

  A set $A \subseteq M$ is \emph{left cofinal} in $M$ if for every $x \in M$,
  there exists $y \in A$ such that $y < x$.

  A set $A \subseteq M$ is \emph{right cofinal} in $M$ if for every $x \in M$,
  there exists $y \in A$ such that $x < y$.

  A set $A \subseteq M$ is \emph{bi-directionally cofinal} in $M$ if it is both left and right cofinal.
\end{definition}

\begin{lemma}
  Let $\pp$ be an additive property of linear orders.

  Then $1 \in \pp$.
\end{lemma}

\begin{note}
  The above lemma is false if we do not restrict ourselves to linear orders.

  For example, $\ps{1 \uplus 1}^+$ is a property of preorders
  which is additive, but does not contain $1$.
\end{note}

\begin{proof}
  Let $M \in \pp$ be any linear order.

  Let $x \in M$. Then,
  $M = (-\infty, x) + \set{x} + (x, \infty)$,
  where $(-\infty, x)$ and/or $(x, \infty)$ may be empty.

  Since $\pp$ is additive, we conclude that $\set{x} \in \pp$.
\end{proof}

\begin{corollary}\label{additive-endpoints}
  Let $\pp$ be an additive property of linear orders.

  Let $M$ be a linear order.

  Let $x, y \in M$ be any two points in a linear order $M$.
  Then the following are equivalent:

  \begin{enumerate}
    \item $(x, y) \in \pp$
    \item $(x, y] \in \pp$
    \item $[x, y) \in \pp$
    \item $[x, y] \in \pp$
  \end{enumerate}
\end{corollary}

\begin{proof}
  This is just applying the definition of an additive property
  to the orders $[x, y]$ and $1$.
\end{proof}

\begin{corollary}\label{additive-transitivity}
  Let $\pp$ be an additive property of linear orders.

  Let $M$ be a linear order.

  Let $x, y, z \in M$ be any three points in a linear order $M$,
  such that $[x, y] \in \pp$ and $[y, z] \in \pp$.

  Then $[x, z] \in \pp$.
\end{corollary}

\begin{proof}
  If $y \in [x, z]$, then $[x, z] = [x, y] + (y, z]$,
  and $(y, z] \in \pp$ by~\cref{additive-endpoints}.

  Otherwise, either $x \in [y, z]$ or $z \in [x, y]$.
  WLOG, suppose $z \in [x, y]$.

  Then $[x, y] = [x, z] + (z, y]$,
  so $[x, z] \in \pp$ by the fact that $\pp$ is additive.
\end{proof}

% Definition of $\bounded{\pp}$
\begin{definitions}
  Let $\pp$ be a property of linear orders.

  We define the following properties of linear orders:
  \begin{itemize}
    \item $\bounded{\pp}$ is the class of linear orders $M$ such that for every $x, y \in M$,
          the bounded subinterval $[x, y]$ is in $\pp$.
    \item $\lb{\pp}$ is the class of linear orders $M$ such that for every $x \in M$,
          the left-bounded ray $[x, \infty) = \set{y \in M : x \le y}$ is in $\pp$.
    \item $\rb{\pp}$ is the class of linear orders $M$ such that for every $x \in M$,
          the right-bounded ray $(-\infty, x] = \set{y \in M : y \le x}$ is in $\pp$.
  \end{itemize}
\end{definitions}

% Definition of a star property
\begin{definition}
  A property $\pp$ of linear orders is a \emph{star property} if
  for every linear orders $M$, and every family $\mathcal{F} \subseteq \pp$
  of subintervals of $M$ such that $J_1 \cap J_2 \ne \emptyset$
  for every $J_1, J_2 \in \mathcal{F}$, we have that
  $\bigcup \mathcal{F} \in \pp$.
\end{definition}

% Partition according to a star property
\begin{lemma}
  Let $\pp$ be a star property.

  Then for every linear order $M$,
  and every point $x \in M$, there exists a largest subinterval $J \subseteq M$ such that
  $J \in \pp$.

  Equivalently, we can define a convex equivalence relation $\sim_{\pp}$ on $M$ such that $x \sim_{\pp} y$ iff $[x, y] \in \pp$.

  That is,
  $x \sim_{\pp} y$ iff $x$ and $y$ are in the same largest $\pp$-subinterval.

\end{lemma}

\begin{proof}
  Let $J \subseteq M$ be the union of all $\bounded{\pp}$-subintervals containing $x$.
  All such subintervals intersect at $x$.

  Therefore, by the star lemma, $J$ is in $\bounded{\pp}$, and by definition
  $J$ is the largest $\pp$-subinterval containing $x$.

  Thus we can define the equivalence relation $\sim_{\pp}$ as above.
\end{proof}

% Lemma: $\bounded{\pp}$ is a star property
\begin{lemma}[Star Lemma]\label{star-lemma}
  Let $\pp$ be an additive property of linear orders.

  Then the property $\bounded{\pp}$ is a star property.
\end{lemma}

% Proof of the star lemma
\begin{proof}
  Let $M$ be a linear order,
  and let $\mathcal{F} \subseteq \bounded{\pp}$ be a family of subintervals of $M$.

  Let $[x, y] \subseteq \bigcup \mathcal{F}$ be any bounded subinterval. We need to prove
  it is in $\pp$.

  Suppose $x \in J_1$ and $y \in J_2$ for $J_1, J_2 \in \mathcal{F}$.

  Since $J_1 \cap J_2 \ne \emptyset$, we can take $z \in J_1 \cap J_2$.

  Then $[x, z] \subseteq J_1$ and $[z, y] \subseteq J_2$,
  and thus by the definition of $\bounded{\pp}$, $[x, z], [z, y] \in \pp$.
  Since $\pp$ is additive, by~\cref{additive-transitivity}, we have $[x, y] \in \pp$.
\end{proof}

\begin{lemma}\label{bounded-properties}
  Let $\pp$ be an additive property of linear orders.

  Then,
  \begin{itemize}
    \item $\bounded{\pp} = \set{M : 1 + M + 1 \in \pp}$
    \item $\lb{\pp} = \set{M : M + 1 \in \pp}$
    \item $\rb{\pp} = \set{M : 1 + M \in \pp}$
    \item $\pp = \lb{\pp} \cap \rb{\pp}$
  \end{itemize}
\end{lemma}

\begin{proof}
  TBC.
\end{proof}

\begin{lemma}
  Let $\pp$ be an additive property of linear orders.

  Then,
  \[
    \begin{aligned}
      \bounded{\pp} & = \pp                                                            \\
                    & \uplus \ps{\lb{\pp} \setminus \rb{\pp}}                          \\
                    & \uplus \ps{\rb{\pp} \setminus \lb{\pp}}                          \\
                    & \uplus \ps{ \bounded{\pp} \setminus \ps{\lb{\pp} \cup \rb{\pp}}}
    \end{aligned}
  \]
\end{lemma}

\begin{proof}
  By~\cref{bounded-properties}, we conclude that
  $\lb{\pp}, \rb{\pp} \subseteq \bounded{\pp}$,
  since $M + 1 \in \pp$ and $1 + M \in \pp$ both imply $1 + M + 1 \in \pp$.

  Thus,
  \[
    \begin{aligned}
      \bounded{\pp} & = \ps{\lb{\pp} \cap \rb{\pp}}                                                      \\
                    & \uplus \ps{\lb{\pp} \setminus \rb{\pp}}                          \\
                    & \uplus \ps{\rb{\pp} \setminus \lb{\pp}}                          \\
                    & \uplus \ps{ \bounded{\pp} \setminus \ps{\lb{\pp} \cup \rb{\pp}}}
    \end{aligned}
  \]

  Since by~\cref{bounded-properties} $\pp = \lb{\pp} \cap \rb{\pp}$,
  we conclude what we wanted to prove.
\end{proof}

\begin{lemma}[Associativity of sum]
  Let $\pp_1$, $\pp_2$ and $\pp_3$ be properties.

  Then $\sum_{\pp_1}{\sum_{\pp_2}{\pp_3}} = \sum_{\sum_{\pp_1}{\pp_2}}{\pp_3}$.
\end{lemma}

\begin{proof}
  It follows directly from the associativity of the sum operation on
  linear orders. Actually, it generalizes to any algebraic equation
  which holds on linear orders.
\end{proof}

\begin{lemma}[Sum over a union]
  Let $\ppp$ be a family of properties.

  Let $\qq$ be a property.

  Then $\sum_{\bigcup \ppp}{\qq} = \bigcup_{\pp \in \ppp}{\sum_{\pp}{\qq}}$.
\end{lemma}

\begin{proof}
  This is obvious from the definition of the sum operation.
\end{proof}


\begin{definition}
  Let $\beta \ge \omega$ be a limit ordinal.

  We define $\Gamma_{\beta} := \set{\gamma : \gamma \subseteq \beta^\ast + \beta}^+$.
\end{definition}

\begin{example}
  \[
    \qqo = \set{1, \omega, \agemo}^+
  \]
\end{example}

\begin{observation}
  Let $\beta \ge \omega$ be a limit ordinal.

  Then $\Gamma_{\beta}$ is a good property of linear orders.
\end{observation}
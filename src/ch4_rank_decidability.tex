\section{Decidability of the Hausdorff Rank}

In this chapter, we will try to establish the decidability of the $\mso$-satisfiability
over the minor and major classes: $\ho{< \alpha}$, $\rmj{= \alpha}$, $\lmj{= \alpha}$, and $\bmj{= \alpha}$,
for every ordinal $\alpha \ge \om$.

% Types of a class
\begin{definition}
    Let $\pp$ be a class of preorders.

    Let $n \in \NN$.

    We define $\type{n}{\pp}$ as the set of all
    $n$-types satisfiable in $\pp$.
\end{definition}

\begin{definition}
    Let $\pp_1$ and $\pp_2$ be classes of preorders.

    Let $n \in \NN$.

    Then we say that $\pp_1$ and $\pp_2$ are \emph{$n$-equivalent},
    denoted $\pp_1 \equiv_n \pp_2$,
    if $\type{n}{\pp_1} = \type{n}{\pp_2}$.
\end{definition}

\begin{lemma}\label{minor-computable}
    There exists a computable function $f: \NN \to \NN$ such that
    for every $n \in \NN$ and every ordinal $\alpha \ge f(n)$,
    $\ho{< \alpha} \equiv_n \ho{< f(n)}$.
\end{lemma}

\begin{proof}
    Since there are only finitely many $n$-types,
    and the ordinal sequence \[\braces{\type{n}{\ho{< \kappa}}}_{\kappa}\]
    is monotone,
    there must be some minimal $\kappa_0 \in \om$ where the sequence stabilizes.

    This $\kappa_0$ is computable as a function of $n$ by
    exhaustive search.
\end{proof}

\begin{lemma}\label{ab-lemma}
    There exist global computable functions $a, b : \NN \to \NN$ such that
    for all $n, c_1, c_2 \in \NN$ such that $c_1, c_2 \ge a(n)$ and $c_1 \equiv c_2 \mod b(n)$,
    $\ho{= c_1} \equiv_n \ho{= c_2}$.
\end{lemma}

\begin{proof}
    Let $n \in \NN$.

    Since there are only finitely many sets of $n$-types,
    there exist (and can be computed by exhaustive search)
    some $a(n) \ge f(n)$, $a(n) + b(n)$ such that

    \[
        \ho{= a(n)} \equiv_n \ho{= a(n) + b(n)}
    \]

    By induction if follows that for all $c \ge a(n)$,
    \[
        \ho{= c} \equiv_n \ho{= c + b(n)}
    \]
    since $\ho{= c + 1} = \sum_{\Om} \ho{= c}$.
\end{proof}

\begin{note}
    Actually we can give explicit formulas
    for some pair of $a(n)$ and $b(n)$ (not necessarily minimal),
    but there is no need to do so in this thesis.
\end{note}

\begin{lemma}\label{major-computable}
    For every $n \in \NN$ and for every ordinal $\alpha \ge \om$,
    \[
        \rmj{\alpha} \equiv_n \sum_{\om} \ho{[a(n), a(n) + b(n)]}
    \]
    \[
        \lmj{\alpha} \equiv_n \sum_{\mo} \ho{[a(n), a(n) + b(n)]}
    \]
    \[
        \bmj{\alpha} \equiv_n \sum_{\oo} \ho{[a(n), a(n) + b(n)]}
    \]
    \[
        \ho{= \alpha} \equiv_n \ho{[a(n), a(n) + b(n)]}
    \]
    In particular, $\type{n}{\rmj{= \alpha}}$ can be computed,
    and is independent of the choice of $\alpha \ge \om$.
\end{lemma}

\begin{proof}
    We proceed by induction on $\alpha \ge \om$.

    For $\alpha = \om$, let $\set{\alpha_i}_{i \in \om}$ be an increasing $\om$-sequence of (finite) ordinals
    such that $a(n) \le \alpha_i$ for all $i \in \om$, and $\sup_{i \in \om} \alpha_i = \om$.

    By~\cref{ab-lemma}, we conclude for all $i \in \om$,
    \[
        \type{n}{\ho{[\alpha_i, \om)}} = \type{n}{\ho{[a(n), a(n) + b(n))}}
    \]

    Then $\rmj{\om} = \sum_{\om} \ho{[\alpha_i, \om)}$
    by~\cref{rmj-decomposition},
    so $\rmj{\om} \equiv_n \sum_{\om} \ho{[a(n), a(n) + b(n))}$.

    \[
        \lmj{\om} \equiv_n \sum_{\mo} \ho{[a(n), a(n) + b(n))}
    \]
    and
    \[
        \bmj{\om} \equiv_n \sum_{\oo} \ho{[a(n), a(n) + b(n))}
    \]
    and $\ho{\le \om} \equiv_n \ho{< f(n)}$.

    Thus, by~\cref{eq-alpha-corollary},
    \[
        \ho{= \om}  \equiv_n \ho{< f(n)} +
        \ps{\sum_{\set{\om, \mo, \oo}} \ho{[a(n), a(n) + b(n))}}
        + \ho{< f(n)}
    \]

    But also,
    \[
        \ho{[a(n), a(n) + b(n))} \equiv_n \ho{< f(n)}
        + \ps{\sum_{\set{\om, \mo, \oo}} \ho{[a(n), a(n) + b(n))}}
        + \ho{< f(n)}
    \]

    Thus $\ho{= \om} \equiv_n \ho{[a(n), a(n) + b(n))}$ as required.

    Now, for $\alpha > \om$, we can take a set $\set{\alpha_i}_{i \in \om}$ of ordinals
    such that $\om \le \alpha_i$ for all $i \in \om$,
    and $\sup_{i \in \om} \ps{\alpha_i + 1} = \alpha$.

    By the induction hypothesis,
    \[
        \ho{[\alpha_i, \alpha)} \equiv_n \ho{[a(n), a(n) + b(n))}
    \]

    Then $\rmj{= \alpha} = \sum_{i \in \om} \ho{[\alpha_i, \alpha)}$,
    and thus \[
        \rmj{= \alpha} \equiv_n \sum_{\om} \ho{[a(n), a(n) + b(n))}
    \]
    as required.

    Again, the corollary for $\lmj{= \alpha}$ and $\bmj{= \alpha}$
    is immediate.

    Similarly, we have by~\cref{eq-alpha-corollary},
    \[
        \ho{= \alpha} \equiv_n \ho{< f(n)}
        + \ps{\sum_{\set{\om, \mo, \oo}} \ho{[a(n), a(n) + b(n))}}
        + \ho{< f(n)}
    \]
    and thus $\ho{= \alpha} \equiv_n \ho{= \omega}$,
    so we are done.
\end{proof}

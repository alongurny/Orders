\documentclass{article}
\usepackage[T1]{fontenc}
\usepackage[utf8]{inputenc}
\usepackage{xcolor}
\usepackage{blindtext}
\usepackage{hyperref}
%\hypersetup{hidelinks,breaklinks}
\usepackage{mathcommand}
\usepackage{amssymb,amsfonts,amsmath,amsthm,mdwlist,verbatim}
\usepackage{cleveref}
\usepackage[notion,quotation]{knowledge}
\usepackage{marginnote}
\usepackage{paralist}
\knowledgeconfigure{anchor point shape=tiny corner,anchor point color=blue}
\bibliographystyle{plainurl}

\newcommand{\MSO}{\mathit{MSO}}
\def\Int{\Gamma}
\def\fS{\mathbb{S}}
\def\nat{\mathbb{N}}

\def\MM{\mathit{MM}}
\def\NM{\mathit{NM}}
\def\M{\mathit{M}}
\def\W{\mathit{W}}
\def\Mn{\mathit{MNx}}
\def\L{\mathcal{L}}
%\ProvidesPackage{macros}

\usepackage{amsmath}
\usepackage{amssymb}
\usepackage{amsthm}
\usepackage[hidelinks]{hyperref}
\usepackage{cleveref}

% Theorems
\newtheorem{theorem}{Theorem}[section]
\newtheorem{corollary}[theorem]{Corollary}
\newtheorem{observation}[theorem]{Observation}
\newtheorem{observations}[theorem]{Observations}
\newtheorem{lemma}[theorem]{Lemma}
\newtheorem{proposition}[theorem]{Proposition}
\newtheorem{conjecture}[theorem]{Conjecture}
\newtheorem{claim}[theorem]{Claim}

% Definitions
\newtheorem{definition}[theorem]{Definition}
\newtheorem{definitions}[theorem]{Definitions}
\newtheorem{notation}[theorem]{Notation}
\newtheorem{notations}[theorem]{Notations}

% Examples
\newtheorem{example}[theorem]{Example}
\newtheorem{examples}[theorem]{Examples}
\newtheorem{remark}[theorem]{Remark}
\newtheorem{remarks}[theorem]{Remarks}
\newtheorem{note}[theorem]{Note}
\newtheorem{notes}[theorem]{Notes}


% Delimiters
\newcommand{\parens}[1]{\left( {#1} \right)}
\newcommand{\brackets}[1]{\left[ {#1} \right]}
\newcommand{\braces}[1]{\left\{ {#1} \right\}}

\newcommand{\ps}[1]{\parens{#1}}
\newcommand{\bs}[1]{\brackets{#1}}
\newcommand{\set}[1]{\braces{#1}}

% Macros
\newcommand{\hh}[2]{\mathbf{H}_{#1}^{#2}}
\newcommand{\ho}[1]{\hh{}{#1}}
\newcommand{\hhrank}[2]{\mathbf{hrank}_{#1} \ps{#2}}
\newcommand{\hrank}[1]{\hhrank{}{#1}}
\newcommand{\rmj}[1]{\mathbf{RM}_{#1}}
\newcommand{\lmj}[1]{\mathbf{LM}_{#1}}
\newcommand{\bmj}[1]{\mathbf{BM}_{#1}}

% Letters
\newcommand{\om}{\omega}
\newcommand{\mo}{\om^\ast}
\newcommand{\Om}{\om}
\newcommand{\oo}{\mo + \om}
\newcommand{\otp}{\mathbf{otp}}
\newcommand{\fso}{\mathbf{FSO}}
\newcommand{\mso}{\mathbf{MSO}}
\newcommand{\bounded}[1]{ \mathcal{B} \bs{{#1}} }
\newcommand{\lb}[1]{ \mathcal{L} \bs{{#1}} }
\newcommand{\rb}[1]{ \mathcal{R} \bs{{#1}} }
\newcommand{\pp}{\mathbf{P}}
\newcommand{\qq}{\mathbf{Q}}

\newcommand{\ppp}{\mathcal{P}}
\newcommand{\qqq}{\mathcal{Q}}

\newcommand{\cnt}{\mathbf{CNT}}
\newcommand{\dfn}{\mathbf{DFN}}

% Sets
\newcommand{\NN}{\mathbb{N}}

% Types
\newcommand{\type}[2]{\mathbf{type}_{#1} \bs{#2}}

% Monoids
\newcommand{\zero}{\cdot}

\newcommand{\abs}[1]{\left| {#1} \right|}

\newcommand{\from}{\leftarrow}

\newcommand{\good}[1]{\mathrm{good}_{#1}}

\newcommand{\Int}[1]{\mathrm{Int}_{#1}}

\newcommand{\qd}[1]{\mathbf{qd}_{#1}}

\newcommand{\otp}[1]{\mathbf{otp}_{#1}}

\newcommand{\qq}{\mathbf{Q}}
\usepackage{mathtools}   % This package is used only to declare \osum.
\DeclareMathOperator*{\osum}{\mathrlap{\hspace{1ex}\circ}{\sum}}
\def\om{\omega}
\def\Chom{\mathit{Cof}_\om}
\def\Dcof{\mathit{DCof}_\om}
\def\WN{\mathit{\Omega\mathit{-Nested}}}
\def\cM{{\mathcal{M}}}
\def\Cofcl{\mathit{Class_{Cof}}}
\def\Nats{\mathbb{N}}
\def\aN{a^{new}}
\def\bN{b^{new}}
\def\stp{\Rightarrow_{FS}}
\def\stpb{\Rightarrow^!_{FS}}
\newcommand{\bb}[1]{\mathbb{#1}}

\newcommand{\reso}[1]{|_{({#1}]}}
\newcommand{\resc}[1]{|_{[{#1}]}}

\newcommand{\rresc}[1]{|_{[{#1})}}
\newcommand{\resO}[1]{|_{({#1})}}
\def\Cl{\mathit{CL}}
\def\vp{\varphi}
\def\vrho{\varrho}
\def\vtheta{\vartheta}
\def\CB{\mathit{CB}}
\def\Hr{\mathit{HR}}
\def\Mhr{\mathit{MHR}}

\def\mM{{\mathcal{M}}}
\def\cN{{\mathcal{N}}}
\def\cL{{\mathcal{L}}}
\def\cF{{\mathcal{F}}}
\def\cS{{\mathcal{S}}}
\def\cW{{\mathcal{W}}}
\def\cU{{\mathcal{U}}}
\def\cT{{\mathcal{T}}}

\def\yield{{\mathit{yield}}}
\def\type{{\mathit{type}}}

\def\MTh{\mathit{MTH}}
\def\Th{\mathit{Th}}
\def\CoP{\mathit{Cof}_\om}
\def\cl{\mathit{Cl}}
\def\Intpr{\mathit{Intr}}
\def\FS{\mathit{FS}}
\def\SBP{\mathit{SBS}}
\def\GSBS{\mathit{GSBS}}
\def\BP{\mathit{BS}}
\def\OTP{\mathit{OTP}}
\def\Col{\mathit{Collapse}}
\def\RELIM{{\mathrm{REPLACELIMIT}}}
\newrobustcmd\alex[1]{\textcolor{teal}{A: #1}}
\let\ALEX\alex
\newrobustcmd\thomas[1]{\textcolor{violet}{T: #1}}
\let\THOMAS\thomas

%\input{knowledge-main}

%\def\restr{\downharpoonright}
\def\restr{|_}

\newcommand{\N}{\mathbb{N}}
\usepackage{cleveref,thm-restate}
\def\Rt{\mathit{RightT}}
\def\lang{\langle}
\def\rang{\rangle}
\def\B{\mathit{Big}}
\def\SBS{\mathit{SBS}}
\def\BSL{\mathit{BSL}}
\def\BS{\mathit{BS}}
\def\sub{\subseteq}

\def\hsuc{\mathit{Hsuc}}
\def\vsuc{\mathit{Vsuc}}

\newtheorem{lemma}{Lemma}[section]
\newtheorem{thm}[lemma]{Theorem}
\newtheorem{dfn}[lemma]{Definition}
\newtheorem{cor}[lemma]{Corollary}
\newtheorem{definition}[lemma]{Definition}
\newtheorem{theorem}[lemma]{Theorem }
\newtheorem{examp}[lemma]{Example}
\newtheorem{remark}[lemma]{Remark}
\newtheorem*{nota}{Notations}
\newtheorem{prop}[lemma]{Proposition}
%\newtheorem{example}[theorem]{Example}
\newenvironment{claim}[1]{\par\noindent\underline{Claim:}\space#1}{}
\newenvironment{claimproof}[1]{\par\noindent\underline{Proof:}\space#1}{\hfill
$\blacksquare$}

\newcommand{\intreebs}{\mathit{Int} _{tree\rightarrow bs}}

\def\vp{\varphi}
\def\rar{\rightarrow}
\newcommand{\cC}{\mathcal{C}}
\newcommand{\T}{{\mathfrak T}}
\def\Ntriv{\mathit{Non-singl}}
\def\Def{\mathit{Im}}
\def\Im{\mathit{Im}}
\def\otp{\mathit{otp}}
\def\pIm{\mathit{Im}^{-1}}
\newcommand{\dom}{\partial}
\newcommand{\Power}{{\mathbb P}}
\newcommand{\A}{{\mathfrak A}}
\newcommand{\qr}{{\mathrm{qr}}}
\newcommand{\Formula}[2]{{\mathfrak{Form}^{#1}_{#2}}}
\newcommand{\Hint}[2]{H^{#1}_{#2}}
\newcommand{\tup}[1]{\mbox{$\overline {#1}$}}
\newcommand{\Typ}[2]{{\mathrm{type}^{#1}_{#2}}}
\newcommand{\typ}[1]{{\mathrm{type}^{#1}}}
\newcommand{\frakB}{\mathfrak{B}}
\def\BWT{\mathit{{BWT}}}
\def\vX{\overline{X}}
\def\sm{\mathit{small}}

\knowledgenewcommand\hrank[1]{\cmdkl{\mathsf{Hrank}}_{#1}}

\knowledgenewcommand\OTp{\mathit{\cmdkl{OTP}}} % For trees
\knowledgenewcommand\subtree[2]{#1^{#2\cmdkl{\uparrow}}}
\knowledgenewcommand\downclosure[1]{#1{\cmdkl{\downarrow}}}
\knowledgenewcommand\llex{\mathbin{\cmdkl{<_{\mathrm{lex}}}}}

%\knowledgenewcommand\BWT{\mathit{\cmdkl{BWT}}}
\knowledgenewcommand\pcbrank[1]{\cmdkl{\mathsf{CBrank}}_{#1}}
\knowledgenewcommand\otP[1]{\cmdkl{\mathsf{otp}_{#1}}}
\knowledgenewcommand\cbrank{\cmdkl{\mathsf{CBrank}}}
%\knowledgenewcommand\hrank{\cmdkl{\mathsf{Hrank}}}

\def\Big{\mathit{Big}}
\def\P{\mathit{P}}
\def\D{\mathit{D}}
\title{$\MSO$ of with Hausdorff rank of intervals}
% \author{Thomas Colcombet and Alexander Rabinovich}
\begin{document}
\maketitle
\begin{abstract}
  %   Ordinal fundamental sequences are crucial in set theory and   logic.
  % We investigate the Monadic Second-Order theory   of systems of
  % fundamental sequences.
\end{abstract}
\tableofcontents
\section{Introduction}

The "Hausdorff  rank" of a linear order is an ordinal that measure its  complexity.  
We denote by $\Hr[\alpha]$ the set of linear orders  of the Hausdorff  rank $\alpha$ and by $\Hr$ the set of linear orders  with defined Hausdorff  rank,  i.e., $\Hr:=\cup\Hr[\alpha]$.

\begin{theorem}[Decidability]  \label{th:dec}
The monadic theory of $\Hr$ is decidable. 
\end{theorem} 
\begin{theorem}[Definable  model property]\label{th:sm}
  If $\vp$ is satisfiable in a $\Hr$, then 
 % \item[Small model property] $\psi$ is satisfiable in $BP_\alpha$ for $\alpha<\om^\om$.
  %\item[Regular Model]
   there is $\psi$ which has a unique model $T$  and $\vp $ is satisfiable in $T$.
Moreover,  the Hausdorff rank of $T$ is finite. 
% 
%  Moreover, $
%  \alpha<\om^\om$ and $FS$ is $\MSO[<]$-definable in $\alpha$. 
   \end{theorem}
  

\begin{theorem}[Monadic Theory]\label{th:code}
The monadic theory of  $\Hr[\om]$  is decidable,
The monadic theory of  $\Hr[\om]$ is the same as the monadic theory of  $\Hr[\alpha] $  for every $\alpha\geq \om$.  
\end{theorem} 

%A subset $X$ of a tree is downward closed if whenever $v\in X$, then all the nodes  on the path from the root of $T$ to $v$ are in $X$.
Given an ordinal $\alpha$, for an interval $I$, let $\intro*\hrank\alpha(I)$ express that   ``$I$ has  "Hausdorff  rank"~$\alpha$.'' 
We denote by  $\intro*\MSO[\hrank\alpha]$ monadic second-order logic  extended with the new predicate $\hrank\alpha(-)$. We prove:
%\begin{theorem}\label{theorem:main-trees}

\begin{theorem}\label{th:main-hr} %
	For all countable ordinals $\alpha$,  
	 the $\MSO[\hrank\alpha]$-theory of the $\Hr$ is decidable,
	 \end{theorem}

\section{Proof Plan of \Cref{th:main-hr}}
We say that an interval $I$ is  Major  of $\Hr$ $\alpha$ ($\Mhr_\alpha(I)$), if $I$ has Hausdorff  rank $\alpha$ and $\forall x,y\in I$ the Hausdorff  rank of $[x,y]$ is less than $\alpha$. 
%
%Define $x\sim y$ if there is $I$  such that $\Mhr_\alpha(I)$  such that $x,y\in I$ or $[x,y]\cap I=\emptyset$
%for every $I$  such that $\Mhr_\alpha(I)$. 
\begin{dfn}[$\sim$-equivalence]
  $x\sim y$ if either
  \begin{enumerate}
    \item there is $I$  such that $\Mhr_\alpha(I)$  and  $x,y\in I$, or 
    \item $[x,y]\cap I=\emptyset$
for every $I$  such that $\Mhr_\alpha(I)$. 
  \end{enumerate}
  %there is $I$  such that $\Mhr_\alpha(I)$)  such that $x,y\in I$ or $[x,y]\cap I=\emptyset$
%for every $I$  such that $\Mhr_\alpha(I)$. 
\end{dfn}
\begin{examp}
  Find $\sim$ for $\alpha=1$ and $L:=\om^2 +25 +Z +3 +(-\om)+(-\om)$.
\end{examp}
\begin{lemma} The $\sim$ equivalence classes of $L$  are intervals. If $J$ is an equivalence class %of $x$
 then
either $\Mhr_\alpha(J)$ or $J\in \Hr_\beta$ for $\beta<\alpha$.
  
\end{lemma}
We call $\sim$-class big/small if \dots

Note that $\sim$ is $\MSO[\hrank\alpha]$-definable but we are not going to use this fact.
 %\footnote{Can you prove Composition theorem
%for  $\MSO[\hrank\alpha]$.}. 
 \begin{lemma}[Translation]\label{lem:translation} There is an algorithm that 
 for every $\vp (\vX)\in \MSO[\hrank\alpha]$ constructs  $\psi(\vX, P_{big},P_\sm)\in \MSO$ % $\psi(\vX, P_\om,P_{-\om},P_Z,P_\sm)\in \MSO$
 such that $\dots$
    
 \end{lemma}
 
%Properties of $\sim$:
 
 If $I_1$ and $I_2$ are successor $\sim$-classes,
 then either $I_1$ does not have maximal or $I_2$ does not have minimal.
 
 Four types of $\sim$-classes: no maximal no minimal of type $Z$, maximal no minimal of type $-\om$, minimum no maximal of type $\om$, small\footnote{Explain these types of intervals $\om$, $-\om$, $Z$ sums.}.
 \begin{lemma}[Properties of $\sim$]\label{lem:properties}
Assume  $I_1$ and $I_2$ are successor $\sim$-classes.
   \begin{enumerate}
     \item % If $I_1$ and $I_2$ are successor $\sim$-classes,
% then 
Either $I_1$ does not have maximal or $I_2$ does not have minimal.
 
     \item If $I_1$ is small, then $I_2$ is not small.
     \item  If $I_1$ is small, then $I_2$ is not of type $\om$.
     \item  If $I_2$ is small, then $I_1$ is not of type $-\om$.
   \end{enumerate}
 \end{lemma}
  
 \begin{lemma} For each  of these types $\dag$, the monadic theory of the interval of type $\dag$ is decidable.
 \end{lemma}
 Let $L$ be a scattered (general) linear order labeled by four types of $\dag\in \{\om.-\om,Z,\mathit{small}\}$. %  and $n$-types.
 We say that the labeling is consistent if $\dots$
 \begin{lemma}
   Assume that $L$ is consistently labeled linear order and $L_i$ for ($i\in L$) be a family of linear orders which is consistent with the labelling of $L$.
   Then the $\sim$-classes on $\sum_{i\in L} L_i$ are $\dots$
 \end{lemma}
 \begin{lemma}[Translation]\label{lem:translation2} There is an algorithm that 
 for every $\vp (\vX)\in \MSO[\hrank\alpha]$ constructs $\psi(\vX, P_\om,P_{-\om},P_Z,P_\sm)\in \MSO$
 such that $\dots$
    
 \end{lemma}
An $n$-type  $\sigma$ is satisfiable if there is a consistent labeling be the corresponding $n$-type.

%\input{trees}

%\input{appendix}

\bibliography{ref}

\end{document}
\input{introduction}
\input{prel-mso-sbs}
\input{sect-undecidability}

\input{sect-elements-of-composition}
\input{sect-decidability-definability}
%\section{Decidability and Definable Model Property for
% $\SBS$}\label{sect:decid-definability}
\input{canonical}
%\section{Monadic Theory of $SBS[\alpha]$ and $\BS[\alpha]$}
\input{code}
\input{sect-invariant}\label{sect:invariant}
% \section{Bi-Interpretations  of Simple Bachmann Systems  in
% Trees}\label{sect:bi-inter}
\input{uniformization}
\input{section-bi-interpretation}
\input{unif-sbs-rt}
\input{sect-from-sbs-bs}

\bibliography{ref}
\end{document}

\section{Everything Better}

\begin{theorem}
    Let $\mathcal{C}$ be a computable property of linear orders,
    such that $\mathcal{C}$ is closed under taking subintervals,
    projections and inverse-projections (i.e, of one of the colors), and all finite-sums and $\mathcal{C}$-sums.

    Let $\pp_1, \ldots, \pp_k \subseteq \mathcal{C}$ be
    computable properties of linear orders.

    Let $\mso[P_1, \ldots, P_k]$ be monadic second order logic of order
    over $\mathcal{C}$,
    with $P_1, \ldots, P_k$ as monadic predicates whose semantics are:
    $P_i(X)$ holds iff $X$ is a subinterval which satisfies $\pp_i$.

    Given $\phi$ a formula of $\mso[P_1, \ldots, P_k]$ (possibly with free variables)
    we define \[ \mathcal{C}_{\phi} = \set{M \in \mathcal{C} : M \models \phi} \]

    Then $\mathcal{C}_{\phi}$ is a computable property of linear orders.
\end{theorem}

\begin{proof}
    By structural induction on $\phi$.

    Suppose $\phi$ is an atomic formula.
    If $\phi$ is of the form $X \subseteq Y$ or $X \le Y$,
    \[
        \mathcal{C}_{\phi} = \set{M \in \mathcal{C} : M \models \phi}
    \]
    and thus,
    \[
        \type{n}{\mathcal{C}_{\phi}} = \set{\tau \in \type{n}{\mathcal{C}} : \tau \models \phi}
    \]

    which is computable since $\type{n}{\mathcal{C}}$ is computable,
    and we can then compute whether $\tau \models \phi$ for each $\tau \in \type{n}{\mathcal{C}}$.

    If $\phi$ is of the form $P_i(X)$,
    then
    \[
        \mathcal{C}_{\phi} = \set{M \in \mathcal{C} : M \models P_i(X)}
    \]
    and thus,
    \[
        \type{n}{\mathcal{C}_{\phi}} = \type{n}{\pp_i}
    \]

    which is computable since $\pp_i$ is computable.

    If $\phi = \neg \phi_1$,
    then
    \[
        \mathcal{C}_{\phi} = \mathcal{C} \setminus \mathcal{C}_{\phi_1}
    \]

\end{proof}
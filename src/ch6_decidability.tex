\section{Decidability of the rank}

% Definition of extension to types
\begin{definition}
  Let $\kappa \ge 0$ be an ordinal.

  Let $\mathbf{op}$ be a $\kappa$-ary operation on linear orders.

  Suppose that $\mathbf{op}$ is invariant under $\equiv_n$, for some $n \in \NN$.

  Then we extend $\mathbf{op}$ to a \emph{partial} operation on types as follows:
  \[
    \mathbf{op} \ps{ \set{M_i}_{i \in \kappa} }
    := \type{n}{\mathbf{op} \ps{M_i}_{i \in \kappa}}
  \]

  This is well defined because of the invariance. Furthermore, we can
  extend it to a partial operation on sets of types in the obvious way.
\end{definition}

\begin{lemma}\label{f-lemma}
  Let $\qq$ be a good property of linear orders.

  There exists a computable function $f_{\qq} = f: \NN \to \NN$ such that
  for all $n \in \NN$, and all $a \in \NN$ such that $a \ge f(n)$,
  $\type{n}{\qq^{\le a}} = \type{n}{\qq^{\le f(n)}}$.

  Equivalently, every linear order of finite rank is $n$-equivalent to some linear order of rank $\le f(n)$.
\end{lemma}

\begin{proof}
  Since there are only finitely many $n$-types,
  and the $\omega$-sequence \[\braces{\type{n}{\qq^{\le k}}}_{k \in \omega}\]
  is monotone,
  there must be some $k$ where the sequence stabilizes.

  This point $k$ is computable as a function of $n$, because
  $\type{n}{\qq^{\le k}}$ is computable for every finite $k$.
\end{proof}

\begin{lemma}\label{ab-lemma}
  There exist global computable functions $a, b : \NN \to \NN$ such that
  for all $n, c_1, c_2 \in \NN$ such that $c_1, c_2 \ge a(n)$ and $c_1 \equiv c_2 \mod b(n)$,
  \[\type{n}{\qq^{= c_1}} = \type{n}{\qq^{= c_2}}\]

  Equivalently, the sequence $\braces{\type{n}{\qq^{k}}}_{k \in \omega}$
  is ultimately periodic for all $n \in \NN$. Furthermore, the starting point and the period
  itself can be computed as a function of $n$.
\end{lemma}

\begin{proof}
  Let $n \in \NN$.

  Since there are only finitely many sets of $n$-types,
  there exist (and can be computed)
  some $a(n) > f(n)$, $a(n) + b(n)$ such that

  \[\type{n}{\qq^{= a(n)}} = \type{n}{\qq^{= a(n) + b(n)}}\]

  holds for every $s$.

  We shall prove by induction that for all $c \ge a(n)$,

  \[\type{n}{\qq^{= c}} = \type{n}{\qq^{= c + b(n)}}\]

  This will complete the proof.

  The base case $c = a(n)$ has been proven in the beginning.

  Suppose the induction hypothesis holds for $c$.

  Let $M$ be of rank $c + 1$.

  Write $M = \sum_{i \in I} M_i$ where $M_i \in \qqo^{< c + 1}$,
  and $M_i \in \qqo^{= c}$ infinitely many times.

  By the induction hypothesis,
  if $M_i \in \qqo^{= c}$, we can find $N_i \equiv_n M_i$ with $N_i \in \qqo^{= c + b(n)}$.
  Setting $N_i := M_i$ for all other $i$, we conclude that $N := \sum_{i \in I} N_i$
  is $n$-equivalent to $M$.

  However, clearly $N \in \qqo^{= c + b(n) + 1}$. So overall,
  \[\type{n}{\qq^{= c + 1}} \subseteq \type{n}{\qq^{= c + b(n) + 1}}\]

  Conversely, suppose $M$ is of rank $c + b(n) + 1$.
  Write $M$ = $\sum_{i \in I} M_i$ where $M_i \in \qqo^{< c + b(n) + 1}$,
  and $M_i \in \qqo^{= c + b(n)}$ infinitely many times.

  By the induction hypothesis,
  we can find for all $i$ such that $M_i \in \qqo^{= c + b(n)}$ some
  $N_i \equiv_n M_i$ with $N_i \in \qqo^{= c}$.
  Furthermore, since $c \ge a(n) > f(n)$, we can
  find $N_i \equiv_n M_i$ with $N_i \in \qqo^{\le f(n) < c}$ for all other $i$.

  We conclude that $N := \sum_{i \in I} N_i$ is $n$-equivalent to $M$.
  However, clearly $N \in \qqo^{= c + 1}$. So overall,
  \[\type{n}{\qq^{= c + b(n) + 1}} \subseteq \type{n}{\qq^{= c + 1}}\]

  So we have proven the induction step, and the lemma follows.
\end{proof}

\begin{corollary}
  Let $n \in \NN$, and let $\alpha \ge \omega$ be an ordinal.

  Let $s \in \set{1, \omega, \agemo, \oo}$ be a shape.

  Then there exists a computable function $b(n)$ such that
  for all $c_1, c_2 \in \NN$ such that $c_1, c_2 > a(n)$ and $c_1 \equiv c_2 \mod b(n)$,
  we have
  \[\type{n}{\mathcal{S}^{s}_{c_1}} = \type{n}{\mathcal{S}^{s}_{c_2}}\]
\end{corollary}

\begin{proof}
  For $s = 1$, it follows from~\cref{f-lemma}, since $\mathcal{S}^{1}_{c} = \qq^{< c} = \qq^{\le c-1}$
  by~\cref{b_alpha-structure-lemma-1}.
  and $c > a(n) \ge f(n)$ so $c-1 \ge f(n)$ for $c \in \set{c_1, c_2}$.

  For $s \in \set{\omega, \agemo, \oo}$, it follows easily from~\cref{b_alpha-structure-lemma-2}
  and~\cref{ab-lemma}.
\end{proof}

\begin{lemma}
  Let $n \in \NN$, and let $\alpha \ge \omega$ be an ordinal.

  Let $s \in \set{1, \omega, \agemo, \oo}$ be a shape.

  \[\type{n}{\mathcal{S}^{s}_{\alpha}} = \bigcup_{c < b(n)}{\mathcal{S}^{s}_{a(n) + c}                                                                                                                                                                                                                                                                                                                                                                                                                                       }\]

  In particular, $\type{n}{\mathcal{S}^{s}_{\alpha}}$ can be
  computed, and is independent of the choice $\alpha \ge \omega$.
\end{lemma}

\begin{proof}
  Draft:
  
\end{proof}

\begin{definition}
  Let $\alpha \ge \omega$ be an ordinal.

  Let $M$ be a linear order and $x \in M$.

  We define the convex equivalence relation: 
  \[\sim_\alpha := \sim_{\bounded{\qqo^{< \alpha}}}\]
  and $\bs{x}_{\alpha} := \bs{x}_{\bounded{\qqo^{< \alpha}}}$ (that is,
  $\bs{x}_{\alpha}$ is the largest $\bounded{\qqo^{< \alpha}}$-subinterval
  containing $x$ in $M$).

  We define $\sigma_{\alpha} \parens{M}$ as
  the $\alpha$-shape of $M$.
\end{definition}


\begin{lemma}
  The property
  $\cdot \in \qqo^{= \alpha}$ is expressible over intervals
  in $\mso[\sim_{\alpha}, \sigma_{\alpha}]$
\end{lemma}

\begin{proof}
  It is equivalent to being a sum of $\sim_{\alpha}$-subintervals,
  of which at least one has $\sigma_{\alpha} \ne 1$.
\end{proof}

\begin{theorem}
  There is an oracle reduction from SAT for $\mso[\sim_{\alpha}, \sigma_{\alpha}]$,
  to SAT for $\mso$.
\end{theorem}

\begin{proof}
  By the decomposition theorem, there exists a translation,
  that given an $\mso[\sim_{\alpha}, \sigma_{\alpha}]$ formula $\varphi$ of quantifier-depth $n$.
  outputs an $\mso$ formula $\psi(\Pi)$ such that...

  Let $\varphi$ be an $\mso[\sim_{\alpha}, \sigma_{\alpha}]$ formula,
  and let $n$ be the quantifier-depth of $\varphi$.

  WLOG, assume that $\varphi$ is a sentence.

  First, let us calculate the sets:

  \[T_s := \type{n}{\mathcal{S}^{s}_{\alpha}}\]
  for every shape $s$.

  Now we create the formulae:
  \[\theta_s(\Pi, \Xi) := \set{i : \bigvee_{\tau \in S_s}{\Xi(\Pi(i)) = s}}\]
  \[L(\Pi, \Xi) := \theta_{\omega}(\Pi, \Xi) \vee \theta_{\oo}(\Pi, \Xi)\]
  \[R(\Pi, \Xi) := \theta_{\agemo}(\Pi, \Xi) \vee \theta_{\oo}(\Pi, \Xi)\]

  We create the formula $\chi(\Pi, \Xi)$ as follows:

  \[\chi := \Pi = domain(\Xi) \wedge \forall i, i'. i' = i + 1 \implies {i \in R(\Pi, \Xi) \vee i' \in L(\Pi, \Xi)}\]

  Now we claim that $\varphi$ is satisfiable in $\mso[\sim_{\alpha}, \sigma_{\alpha}]$
  iff $\psi \land \chi$ is satisfiable in $\mso$.

  If $\varphi$ is satisfiable, then there exists a model $M$ of $\varphi$.

  Let $M = \sum_{i \in I} M_i$ be the decomposition of $M$
  where $I = \sim_{\alpha}$ and $M_i$ are the $\sim_{\alpha}$-equivalence classes.

  By the decomposition theorem, $\Psi$ holds
  in $I, \Pi := \type{n}{\cdot}$.

  We claim that $\chi$ holds in $I, \Pi := \type{n}{\cdot}$.

  It follows from the star property of $\sim_{\alpha}$ that the constraint holds.

  Conversely, suppose $\psi \land \chi$ is satisfiable in $\mso$.

  Let $I, \Pi := T$ be a model of $\psi \land \chi$.

  Let us take a model $M_i$ with the appropriate type.
  Now define $M := \sum_{i \in I} M_i$.

  We claim that each $M_i$ is a \emph{maximum} $\bounded{\qqo^{< \alpha}}$-subinterval
  of $M$.

  Suppose $[M_i, M_j]$ is a $\bounded{\qqo^{< \alpha}}$-subinterval of $M$.

  In particular, it has a rank, so it is scattered. So in particular,
  $[i, j] \subseteq I$ is a scattered interval.

  If $i = j$ we are done. Otherwise, let $i'$, $j'$ be such that $i \le i' < j' \le j$,
  and $j' = i' + 1$.  But it cannot be the case by the constraint.


\end{proof}

\begin{theorem}
  Suppose $\alpha_k > \alpha_{k - 1} > \ldots > \alpha_0 > 0$ are ordinals.

  Then there is an oracle reduction from SAT for $\mso[
      \sim_{\alpha_k}, \sigma_{\alpha_k}, \ldots, \sim_{\alpha_0}, \sigma_{\alpha_0}]$,

  to SAT for $\mso[
      \sim_{\delta_{k-1}}, \sigma_{\delta_{k-1}}, \ldots, \sim_{\delta_0}, \sigma_{\delta_0}]$.

  where $\alpha_0 + \delta_i = \alpha_{i + 1}$ for $0 \le i < k$.
\end{theorem}

\begin{proof}
  By the decomposition theorem, there exists a translation,
  that given an $\mso[\sim_{\alpha_k}, \sigma_{\alpha_k}, \ldots, \sim_{\alpha_0}, \sigma_{\alpha_0}]$ formula $\varphi$ of quantifier-depth $n$.
  outputs an $\mso$ formula $\psi(\Pi)$ such that...

  Let $\varphi$ be an $\mso[\sim_{\alpha_k}, \sigma_{\alpha_k}, \ldots, \sim_{\alpha_0}, \sigma_{\alpha_0}]$ formula,
  and let $n$ be the quantifier-depth of $\varphi$.

  WLOG, assume that $\varphi$ is a sentence.

  First, let us calculate the sets
  $\type{n}{\qqo^{< \alpha}}$,
  and 

  Now we create the formulae:
  \[\theta_s(\Pi, \Xi) := \set{i : \bigvee_{\tau \in S_s}{\Xi(\Pi(i)) = s}}\]
  \[L(\Pi, \Xi) := \theta_{\omega}(\Pi, \Xi) \vee \theta_{\oo}(\Pi, \Xi)\]
  \[R(\Pi, \Xi) := \theta_{\agemo}(\Pi, \Xi) \vee \theta_{\oo}(\Pi, \Xi)\]

  We create the formula $\chi(\Pi, \Xi)$ as follows:

  \[\chi := \Pi = domain(\Xi) \wedge \forall i, i'. i' = i + 1 \implies {i \in R(\Pi, \Xi) \vee i' \in L(\Pi, \Xi)}\]

  Now we claim that $\varphi$ is satisfiable in $\mso[\sim_{\alpha}, \sigma_{\alpha}]$
  iff $\psi \land \chi$ is satisfiable in $\mso$.

  If $\varphi$ is satisfiable, then there exists a model $M$ of $\varphi$.

  Let $M = \sum_{i \in I} M_i$ be the decomposition of $M$
  where $I = \sim_{\alpha}$ and $M_i$ are the $\sim_{\alpha}$-equivalence classes.

  By the decomposition theorem, $\Psi$ holds
  in $I, \Pi := \type{n}{\cdot}$.

  We claim that $\chi$ holds in $I, \Pi := \type{n}{\cdot}$.

  It follows from the star property of $\sim_{\alpha}$ that the constraint holds.

  Conversely, suppose $\psi \land \chi$ is satisfiable in $\mso$.

  Let $I, \Pi := T$ be a model of $\psi \land \chi$.

  Let us take a model $M_i$ with the appropriate type.
  Now define $M := \sum_{i \in I} M_i$.

  We claim that each $M_i$ is a \emph{maximum} $\bounded{\qqo^{< \alpha}}$-subinterval
  of $M$.

  Suppose $[M_i, M_j]$ is a $\bounded{\qqo^{< \alpha}}$-subinterval of $M$.

  In particular, it has a rank, so it is scattered. So in particular,
  $[i, j] \subseteq I$ is a scattered interval.

  If $i = j$ we are done. Otherwise, let $i'$, $j'$ be such that $i \le i' < j' \le j$,
  and $j' = i' + 1$.  But it cannot be the case by the constraint.

\end{proof}

\section{Decidability of the rank}

% $\alpha$ predicate
\begin{definition}
  Let $\qq$ be a property of linear orders.

  Let $M$ be a linear order.

  We define the predicate $\mathcal{I}_{\qq} \ps{J}$ as
  true in $M$ iff $J$ is a $\qq$-subinterval of $M$.
\end{definition}

\begin{lemma}
  Let $\alpha > 0$ be an ordinal.

  The predicates $\mathcal{I}_{\qqo^{\le \alpha}}$ and $\mathcal{I}_{\qqo^{= \alpha}}$
  are $\mso$-expressible over $\mso[\mathcal{I}_{\qqo^{< \alpha}}]$.
\end{lemma}

\begin{proof}
  Obviously, \[
    \mathcal{I}_{\qqo^{= \alpha}}
    \iff \mathcal{I}_{\qqo^{\le \alpha}} \wedge \neg \mathcal{I}_{\qqo^{< \alpha}}
  \]

  So it is enough to express $\mathcal{I}_{\qqo^{\le \alpha}}$.

  Now, $J$ is a $\qqo^{\le \alpha}$-subinterval of $M$ iff
  $J \in \sum_{\Omega}{\qqo^{< \alpha}}$. But this can be expressed, since it is expressible
  to check whether an arbitrary subset is in $\Omega$.
\end{proof}

\begin{definition}
  Let $\alpha > 0$ be an ordinal.

  Let $M$ be a linear order and $x \in M$.

  We define the convex equivalence relation:
  \[\sim_\alpha := \sim_{\bounded{\qqo^{\alpha}}}\]
  and $\bs{x}_{\alpha} := \bs{x}_{\bounded{\qqo^{\alpha}}}$.

  That is,
  $\bs{x}_{\alpha}$ is the largest $\bounded{\qqo^{\alpha}}$-subinterval
  containing $x$ in $M$.

  We define $\sigma_{\alpha} \parens{x}$ as
  the $\alpha$-shape of $\bs{x}_{\alpha}$.
\end{definition}

\begin{definition}
  Let $\alpha > 0$ be an ordinal.

  The predicate $\pi_{\alpha}(x)$ is defined as the representation of
  the convex equivalence relation $\sim_{\alpha}$.
\end{definition}

\begin{definition}
  Let $\alpha > 0$ be an ordinal.

  The function $\sigma_{\alpha}(x)$ is defined as the $\alpha$-shape of
  the convex equivalence class $\bs{x}_{\alpha}$.
\end{definition}

\begin{lemma}
  Let $\alpha > 0$ be an ordinal.

  The predicate $\mathcal{I}_{\qqo^{< \alpha}}$ is $\mso$-expressible over
  $\mso[\pi_{\alpha}, \sigma_{\alpha}]$.
\end{lemma}

\begin{proof}
  It can be proven that an interval $J$ is a $\qqo^{< \alpha}$-subinterval of $M$ iff
  it is contained in a single $\sim_{\alpha}$-equivalence class $K$, such that:
  \begin{itemize}
    \item If $K \notin \lb{\qqo^{< \alpha}}$, then $J$ is bounded in $K$ on the left.
    \item If $K \notin \rb{\qqo^{< \alpha}}$, then $J$ is bounded in $K$ on the right.
  \end{itemize}

  Since $\lb{\qqo^{< \alpha}}$ and $\rb{\qqo^{< \alpha}}$ are expressible,
  this finishes the proof.
\end{proof}

% Sum of computable properties over a computable index
\begin{theorem}\label{computable-sum}
  Let $\pp$ be a computable property of linear orders,
  labeled with finitely many colors $C$.

  Let $F$ be a function assigning
  to each color in $C$ a computable
  property of linear orders, labeled with
  finitely many colors.

  Then the sum $\sum_{\pp}{F}$ is a computable
  property of linear orders.
\end{theorem}

\begin{proof}
  We will use the decomposition theorem.
  Let $\tau(X_1, \ldots, X_m)$ be an $n$-type.

  Then we can compute a formula $\psi(\xi)$ (where
  $\xi$ has the type of a coloring whose range is
  the set of $n$-types) such that
  for any linear order $M = \sum_{i \in I} M_i$,

  and any given $A_1, \ldots, A_m \subseteq M$,
  \[
    M \models \tau(A_1, \ldots, A_m)
    \iff I \models \psi(\Xi)
  \]

  where $\Xi$ is the coloring assigning $i \in I$ the $n$-type of $M_i$.

  TBC.
\end{proof}

\begin{theorem}
  Let $\alpha$ be an ordinal.

  Let $C$ be the class of
  all countable linear orders labeled
  with $\bs{\cdot}_{\alpha}$ and $\sigma_{\alpha}$.

  Let $D_s$ for $s \in \set{1, \omega, \agemo, \oo}$ be the
  class $\mathcal{S}^{s}_{\alpha}$,
  labeled (trivially) with $\bs{\cdot}_{\alpha}$ and $\sigma_{\alpha}$.

  Let $\mathbf{G}$ be the class of all countable linear orders $I$,
  labeled with a coloring function $\gamma$
  whose range is $\set{1, \omega, \agemo, \oo}$,
  such that for pair $i, j \in I$ such that $j$ is the successor of $i$,
  either $\gamma(i) \in \set{\omega, \oo}$
  or $\gamma(j) \in \set{\agemo, \oo}$.

  Then, $C = \sum_{\mathbf{G}} \bs{s \mapsto D_s}$.
\end{theorem}

\begin{proof}
  TBC.
\end{proof}

\begin{theorem}
  Let $\alpha, \delta_1, \ldots, \delta_k$ be ordinals.

  Let $\alpha_i = \alpha + \delta_i$ for $i = 1, \ldots, k$.

  Let $C$ be the class of all countable linear orders
  labeled with $\bs{\cdot}_{\alpha}$ and $\sigma_{\alpha}$,
  and $\bs{\cdot}_{\alpha_i}$ and $\sigma_{\alpha_i}$ for $i = 1, \ldots, k$.

  Let $\mathbf{G}$ be the class of all countable linear orders $I$,
  labeled with a coloring function $\gamma$
  whose range is $\set{1, \omega, \agemo, \oo}$,
  such that for pair $i, j \in I$ such that $j$ is the successor of $i$,
  either $\gamma(i) \in \set{\omega, \oo}$
  or $\gamma(j) \in \set{\agemo, \oo}$.
\end{theorem}

\begin{proof}
  TBC.
\end{proof}
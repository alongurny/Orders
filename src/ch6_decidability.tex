\section{Decidability of the rank}

\begin{definition}
  Let $\alpha$ be an ordinal.

  Let $M$ be a linear order and $x \in M$.

  We define the convex equivalence relation:
  \[\sim_\alpha := \sim_{\bounded{\qqo^{\alpha}}}\]
  and $\bs{x}_{\alpha} := \bs{x}_{\bounded{\qqo^{\alpha}}}$.
  
  That is,
  $\bs{x}_{\alpha}$ is the largest $\bounded{\qqo^{\alpha}}$-subinterval
  containing $x$ in $M$.

  We define $\sigma_{\alpha} \parens{x}$ as
  the $\alpha$-shape of $\bs{x}_{\alpha}$.
\end{definition}

% Sum of computable properties over a computable index
\begin{theorem}\label{computable-sum}
    Let $\pp$ be a computable property of linear orders,
    labeled with finitely many colors $C$.

    Let $F$ be a function assigning
    to each color in $C$ a computable
    property of linear orders, labeled with
    finitely many colors.

    Then the sum $\sum_{\pp}{F}$ is a computable
    property of linear orders.
\end{theorem}

\begin{proof}
    We will use the decomposition theorem.
    Let $\tau(X_1, \ldots, X_m)$ be an $n$-type.

    Then we can compute a formula $\psi(\xi)$ (where
    $\xi$ has the type of a coloring whose range is
    the set of $n$-types) such that
    for any linear order $M = \sum_{i \in I} M_i$,

    and any given $A_1, \ldots, A_m \subseteq M$,
    \[
        M \models \tau(A_1, \ldots, A_m)
        \iff I \models \psi(\Xi)
    \]

    where $\Xi$ is the coloring assigning $i \in I$ the $n$-type of $M_i$.

    TBC.
\end{proof}

\begin{theorem}
  Let $\alpha$ be an ordinal.

  Let $C$ be the class of
  all countable linear orders labeled
  with $\bs{\cdot}_{\alpha}$ and $\sigma_{\alpha}$.

  Let $D_s$ for $s \in \set{1, \omega, \agemo, \oo}$ be the
  class $\mathcal{S}^{s}_{\alpha}$,
  labeled (trivially) with $\bs{\cdot}_{\alpha}$ and $\sigma_{\alpha}$.

  Let $\mathbf{G}$ be the class of all countable linear orders $I$,
  labeled with a coloring function $\gamma$
  whose range is $\set{1, \omega, \agemo, \oo}$,
  such that for pair $i, j \in I$ such that $j$ is the successor of $i$,
  either $\gamma(i) \in \set{\omega, \oo}$
  or $\gamma(j) \in \set{\agemo, \oo}$.

  Then, $C = \sum_{\mathbf{G}} \bs{s \mapsto D_s}$.
\end{theorem}

\begin{proof}
  TBC.
\end{proof}

\begin{theorem}
  Let $\alpha, \delta_1, \ldots, \delta_k$ be ordinals.

  Let $\alpha_i = \alpha + \delta_i$ for $i = 1, \ldots, k$.

  Let $C$ be the class of all countable linear orders
  labeled with $\bs{\cdot}_{\alpha}$ and $\sigma_{\alpha}$,
  and $\bs{\cdot}_{\alpha_i}$ and $\sigma_{\alpha_i}$ for $i = 1, \ldots, k$.

  Let $\mathbf{G}$ be the class of all countable linear orders $I$,
  labeled with a coloring function $\gamma$
  whose range is $\set{1, \omega, \agemo, \oo}$,
  such that for pair $i, j \in I$ such that $j$ is the successor of $i$,
  either $\gamma(i) \in \set{\omega, \oo}$
  or $\gamma(j) \in \set{\agemo, \oo}$.
\end{theorem}

\begin{proof}
  TBC.
\end{proof}
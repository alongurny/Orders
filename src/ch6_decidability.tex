\section{Decidability of the rank}

Decidability questions lie at the heart of mathematical logic and theoretical computer science. In this chapter, we investigate the decidability of rank-related classes for linear orders, connecting the structural results of previous chapters with algorithmic considerations. We introduce key predicates and equivalence relations, and show how they can be expressed and manipulated in logical frameworks.

% $\alpha$ predicate
\begin{definition}
  Let $\qq$ be a class of linear orders.

  Let $M$ be a linear order.

  We define the predicate $\mathbf{Int}_{\qq} \ps{J}$ as
  true in $M$ iff $J$ is a $\qq$-subinterval of $M$.
\end{definition}

\begin{lemma}
  Let $\alpha > 0$ be an ordinal.

  Then predicates $\mathbf{Int}_{\ho{\le \alpha}}$, $\mathbf{Int}_{\ho{= \alpha}}$
  are expressible in $\mso[\mathbf{Int}_{\ho{< \alpha}}]$.
\end{lemma}

\begin{proof}
  Obviously, \[
    \mathbf{Int}_{\ho{= \alpha}}
    \iff \mathbf{Int}_{\ho{\le \alpha}} \wedge \neg \mathbf{Int}_{\ho{< \alpha}}
  \]

  So it is enough to express $\mathbf{Int}_{\ho{\le \alpha}}$.

  Now, $J$ is a $\ho{\le \alpha}$-subinterval of $M$ iff
  $J \in \sum_{\Om}{\ho{< \alpha}}$.

  But this can be expressed in $\mso$ since it is expressible
  to check whether an arbitrary subset is in $\Om$.
\end{proof}

\begin{definition}
  Let $\alpha > 0$ be an ordinal.

  Let $M$ be a linear order and $x \in M$.

  We define the convex equivalence relation:
  \[\sim_\alpha := \sim_{\bounded{\ho{< \alpha}}}\]
  and $\bs{x}_{\alpha} := \bs{x}_{\bounded{\ho{< \alpha}}}$.

  That is,
  $\bs{x}_{\alpha}$ is the largest $\bounded{\ho{< \alpha}}$-subinterval
  containing $x$ in $M$.

  We define
  $\mathbf{L}_{\alpha} \ps{x} = \mathbf{1}_{\lb{\ho{< \alpha}}} \ps{\bs{x}_{\alpha}}$ and
  $\mathbf{R}_{\alpha} \ps{x} = \mathbf{1}_{\rb{\ho{< \alpha}}} \ps{\bs{x}_{\alpha}}$

  (where $\mathbf{1}_{A}$ is the indicator function of a set $A$).

  We define the \emph{the $\alpha$-shape}, $\sigma_{\alpha} \ps{x}$ as follows:
  \[
    \sigma_{\alpha} \ps{x} := \begin{cases}
      \ho{< \alpha} & \text{if } \mathbf{L}_{\alpha} \ps{x} = \mathbf{R}_{\alpha} \ps{x} = 0    \\
      \rmj{\alpha}  & \text{if } \mathbf{L}_{\alpha} \ps{x} = 0, \mathbf{R}_{\alpha} \ps{x} = 1 \\
      \lmj{\alpha}  & \text{if } \mathbf{L}_{\alpha} \ps{x} = 1, \mathbf{R}_{\alpha} \ps{x} = 0 \\
      \bmj{\alpha}  & \text{if } \mathbf{L}_{\alpha} \ps{x} = \mathbf{R}_{\alpha} \ps{x} = 1
    \end{cases}
  \]
\end{definition}

\begin{lemma}\label{alpha-expressible}
  Let $M$ be a linear order and $\alpha > 0$ an ordinal.

  Let $J \subseteq M$ be an interval.

  Then $J \in \ho{< \alpha}$ iff
  it is contained in a single $\sim_{\alpha}$-equivalence class $K$, such that:
  \begin{itemize}
    \item Either $K \in \lb{\ho{< \alpha}}$ or
          there exists some $x \in K$ such that $x < J$.
    \item Either $K \in \rb{\ho{< \alpha}}$ or
          there exists some $x \in K$ such that $x > J$.
  \end{itemize}
\end{lemma}

\begin{proof}
  Suppose $J \in \ho{< \alpha}$.
  Then obviously $J$ is contained in a single $\sim_{\alpha}$-equivalence class $K$.

  We will show the first condition, the second is symmetric.

  Suppose that for all $x \in K$, $J \le x$.
  Then we can write $K = J + J'$.
  Since $J \in \ho{< \alpha}$, it follows that $K \in \lb{\ho{< \alpha}}$.
\end{proof}

\begin{corollary}\label{int-expressible}
  Let $\alpha > 0$ be an ordinal.

  Let $P_{\alpha}$ be any predicate representing $\sim_{\alpha}$,
  let $L_{\alpha} = \mathbf{L}_{\alpha}$ and $R_{\alpha} = \mathbf{R}_{\alpha}$.

  Then $\mathbf{Int}_{\ho{< \alpha}}$ is $\mso$-expressible over
  $\mso[P_{\alpha}, L_{\alpha}, R_{\alpha}]$.
\end{corollary}

% Sum of computable classes over a computable index
\begin{theorem}\label{computable-sum}
  Let $\pp$ be a computable class of linear orders of some finite signature,
  including $C_1, \ldots, C_k$.

  Let $\qqq$ be a \emph{finite} set of computable classes of linear orders
  over some finite signature which is disjoint from the signature of $\pp$.

  Let $F : 2^k \to \qqq$ be any function.

  Then $\bigcup_{I \in \pp} \sum_{i \in I} F(C_1(i), \ldots, C_k(i))$ is a computable class of linear orders.
\end{theorem}

\begin{proof}
  We will use the decomposition theorem.
  Let $\varphi$ be a formula of quantifier depth $n$. WLOG, $\varphi$ is a sentence.

  Then we can compute a formula $\psi(\xi)$ (where
  $\xi$ has the type of a coloring whose range is
  the set of $n$-types) such that
  for any linear order $M = \sum_{i \in I} M_i$,

  \[
    M \models \varphi \iff I \models \psi(\Xi)
  \]

  where $\Xi$ is the coloring assigning $i \in I$ the $n$-type of $M_i$.

  Thus, there is some $M \in \bigcup_{I \in \pp} \sum_{i \in I} \qq_i$,
  such that $M \models \varphi$
  iff there exists some $I \in \pp$, and assignment $\Xi$ of $n$-types,
  such that $\Xi(i)$ is satisfiable in $\qq_i$ for all $i \in I$, and $I \models \psi(\Xi)$.

  Equivalently, $\varphi$ is satisfiable over $\bigcup_{I \in \pp} \sum_{i \in I} \qq_i$
  iff
  \begin{align*}
    \exists \xi. \psi(\xi)
     & \wedge \xi \text{ is a coloring with }n\text{-types}             \\
     & \wedge \forall i. \xi(i) \in \type{n}{F(C_1(i), \ldots, C_k(i))}
  \end{align*}
  is satisfiable over $\pp$.

  Since $\qqq$ has only computable classes, We can pre-compute $\type{n}{F(\vec{c})}$
  for any value $\vec{c} \in 2^k$ so we can actually write the formula above in
  $\mso$. Furthermore, since $\pp$ is computable, we can check whether it is satisfiable
  over $\pp$. So we are done.
\end{proof}

\begin{lemma}{\label{single-ordinal-decomposition}}
  Let $\alpha$ be an ordinal.

  Let $C$ be the class of
  all countable linear orders labeled with $P_\alpha$ which represents $\sim_{\alpha}$,
  $L_{\alpha} = \mathbf{L}_{\alpha}$ and $R_{\alpha} = \mathbf{R}_{\alpha}$,

  Let $\mathbf{G}_{\alpha}$ be the class of all countable linear orders $I$,
  labeled with a $P_\alpha$, $L_\alpha$ and $R_\alpha$,
  such that for every pair $i, i' \in I$ such that $i'$ is the successor of $i$,
  $P_\alpha(i) \ne P_\alpha(i')$,
  and either $R_\alpha(i) = 0$ or $L_\alpha(i') = 0$.

  Then,
  \[
    C = \sum_{I \in \mathbf{G}_{\alpha}} \sum_{i \in I} \sigma_\alpha(i)
  \]
\end{lemma}

\begin{proof}
  ($\subseteq$) Let $M$ be a countable linear order labeled with $P_\alpha$, $L_\alpha$ and $R_\alpha$ as above.

  Let $I = M / \sim_{\alpha}$ be the quotient of $M$ by the equivalence relation $\sim_{\alpha}$.

  Then $M = \sum_{i \in I} M_i$,
  where $\set{M_i}_{i \in I}$ are the $\sim_{\alpha}$-equivalence class of $I$.

  Then for each $i \in I$, $M_i \in \bounded{\ho{< \alpha}}$,
  and by definition $\sigma_\alpha(i) = \sigma_{\alpha} \ps{M_i}$.

  Let $i'$ be the successor of $i$ in $I$.

  Then $P_\alpha(i) \ne P_\alpha(i')$ since $P_\alpha$ represents $\sim_{\alpha}$.

  Furthermore, suppose $R_\alpha(i) = L_\alpha(i') = 1$ holds.
  Then $M_i \in \rb{\ho{< \alpha}}$ and $M_{i'} \in \lb{\ho{< \alpha}}$.
  so $M_i$ and $M_{i'}$ are the same $\sim_{\alpha}$-equivalence class of $M$,
  which is a contradiction.

  Thus either $R_\alpha(i) = 0$ or $L_\alpha(i') = 0$.

  ($\supseteq$) Let $M = \sum_{i \in I} M_i$ be a linear order
  such that $I \in \mathbf{G}_{\alpha}$ and $M_i \in \sigma_\alpha(i)$ for each $i \in I$.

  In particular $M_i \in \bounded{\ho{< \alpha}}$ for each $i \in I$, so it is contained
  in a single $\sim_{\alpha}$-equivalence class of $M$.

  Suppose that there exist distinct $j ,k \in I$ such that $j < k$, and
  $M_j, M_k$ are in the same $\sim_{\alpha}$-equivalence class.

  Let $x \in M_j$ and $y \in M_k$.
  Then $[x, y] \in \ho{< \alpha}$,
  and thus $[j, k] \in \ho{< \alpha}$,
  and in particular it is sparse.

  Then there exist some $j', k' \in I$ such that $j < j' < k' < k$,
  and $k'$ is the successor of $j'$ in $I$.

  Then $M_{j'}$ and $M_{k'}$ are in the same $\sim_{\alpha}$-equivalence class.
  Thus it must be the case that $M_{j'} \in \rb{\ho{< \alpha}}$ and $M_{k'} \in \lb{\ho{< \alpha}}$,
  which implies $R_\alpha(j') = L_\alpha(k') = 1$, which is a contradiction.

  Thus $\set{M_i}_{i \in I}$ are pairwise distinct $\sim_{\alpha}$-equivalence classes,
  and obviously the conditions holds,
  so $M \in C$ and we are done.
\end{proof}

\begin{corollary}\label{countables-computable}
  Let $\alpha > 0$ be an ordinal.

  Let $C$ be defined as in~\cref{single-ordinal-decomposition}.

  Then $C$ is a computable class of linear orders.
\end{corollary}

\begin{proof}
  Since $\mathbf{G}_{\alpha}$ is clearly computable,
  it follows from combining~\cref{computable-sum} and~\cref{single-ordinal-decomposition}.
\end{proof}

\begin{theorem}\label{single-ordinal-satisfiability}
  Let $\alpha > 0$ be an ordinal.

  Satisfiability of $\mso[\mathbf{Int}_{\ho{< \alpha}}]$
  over all countable linear orders is decidable.
\end{theorem}

\begin{proof}
  First, by~\cref{int-expressible}, we can convert
  any formula in $\mso[\mathbf{Int}_{\ho{< \alpha}}]$
  is equisatisfiable over $C$ to a formula in $\mso[P_\alpha, L_\alpha, R_\alpha]$,
  which is decidable by~\cref{countables-computable}.
\end{proof}

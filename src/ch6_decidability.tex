\section{Decidability of the rank}

% $\alpha$ predicate
\begin{definition}
  Let $\qq$ be a property of linear orders.

  Let $M$ be a linear order.

  We define the predicate $\mathbf{Int}_{\qq} \ps{J}$ as
  true in $M$ iff $J$ is a $\qq$-subinterval of $M$.
\end{definition}

\begin{lemma}
  Let $\alpha > 0$ be an ordinal.

  Then predicates $\mathbf{Int}_{\qqo^{\le \alpha}}$, $\mathbf{Int}_{\qqo^{= \alpha}}$
  are expressible in $\mso[\mathbf{Int}_{\qqo^{< \alpha}}]$.
\end{lemma}

\begin{proof}
  Obviously, \[
    \mathbf{Int}_{\qqo^{= \alpha}}
    \iff \mathbf{Int}_{\qqo^{\le \alpha}} \wedge \neg \mathbf{Int}_{\qqo^{< \alpha}}
  \]

  So it is enough to express $\mathbf{Int}_{\qqo^{\le \alpha}}$.

  Now, $J$ is a $\qqo^{\le \alpha}$-subinterval of $M$ iff
  $J \in \sum_{\qqo}{\qqo^{< \alpha}}$.

  But this can be expressed in $\mso$ since it is expressible
  to check whether an arbitrary subset is in $\qqo$.
\end{proof}

\begin{definition}
  Let $\alpha > 0$ be an ordinal.

  Let $M$ be a linear order and $x \in M$.

  We define the convex equivalence relation:
  \[\sim_\alpha := \sim_{\bounded{\qqo^{\alpha}}}\]
  and $\bs{x}_{\alpha} := \bs{x}_{\bounded{\qqo^{\alpha}}}$.

  That is,
  $\bs{x}_{\alpha}$ is the largest $\bounded{\qqo^{\alpha}}$-subinterval
  containing $x$ in $M$.

  We define $\sigma_{\alpha} \ps{x}$ as
  the $\alpha$-shape of $\bs{x}_{\alpha}$.

  We define
  $L_{\alpha} \ps{x} = \mathbf{1}_{\bs{x}_{\alpha} \in \lb{\qqo^{< \alpha}}}$ and
  $R_{\alpha} \ps{x} = \mathbf{1}_{\bs{x}_{\alpha} \in \rb{\qqo^{< \alpha}}}$.
\end{definition}

\begin{lemma}\label{alpha-expressible}
  Let $M$ be a linear order and $\alpha > 0$ an ordinal.

  Let $J \subseteq M$ be an interval.

  Then $J \in \qqo^{< \alpha}$ iff
  it is contained in a single $\sim_{\alpha}$-equivalence class $K$, such that:
  \begin{itemize}
    \item Either $K \in \lb{\qqo^{< \alpha}}$ or
          there exists some $x \in K$ such that $x < J$.
    \item Either $K \in \rb{\qqo^{< \alpha}}$ or
          there exists some $x \in K$ such that $x > J$.
  \end{itemize}
\end{lemma}

\begin{proof}
  TBC.
\end{proof}

\begin{corollary}\label{int-expressible}
  Let $\alpha > 0$ be an ordinal.

  The predicate $\mathbf{Int}_{\qqo^{< \alpha}}$ is $\mso$-expressible over
  $\mso[\bs{\cdot}_{\alpha}, L_{\alpha}, R_{\alpha}]$.
\end{corollary}

% Sum of computable properties over a computable index
\begin{theorem}\label{computable-sum}
  Let $\pp$ be a computable property of linear orders of some finite signature.

  Let $\set{\qq_i}_{i \in I}$ be a \emph{finite} family of computable properties of linear orders
  over some finite signature which is disjoint from the signature of $\pp$.

  Then $\bigcup_{I \in \pp} \sum_{i \in I} \qq_{i}$ is a computable property of linear orders.
\end{theorem}

\begin{proof}
  We will use the decomposition theorem.
  Let $\tau(X_1, \ldots, X_m)$ be an $n$-type.

  Then we can compute a formula $\psi(\xi)$ (where
  $\xi$ has the type of a coloring whose range is
  the set of $n$-types) such that
  for any linear order $M = \sum_{i \in I} M_i$,

  and any given $A_1, \ldots, A_m \subseteq M$,
  \[
    M \models \tau(A_1, \ldots, A_m)
    \iff I \models \psi(\Xi)
  \]

  where $\Xi$ is the coloring assigning $i \in I$ the $n$-type of $M_i$.

  TBC.
\end{proof}

\begin{lemma}{\label{single-ordinal-decomposition}}
  Let $\alpha$ be an ordinal.

  Let $P$, $L$ and $R$ be
  first-order unary predicates.

  Let $C$ be the class of
  all countable linear orders labeled with $P$, $L$ and $R$,
  such that $P$ represents $\sim_{\alpha}$,
  $L_{\alpha} \ps{x} \iff \bs{x}_{\alpha} \in \lb{\qqo^{< \alpha}}$ and
  $R_{\alpha} \ps{x} \iff \bs{x}_{\alpha} \in \rb{\qqo^{< \alpha}}$.

  Let $\mathbf{G}$ be the class of all countable linear orders $I$,
  labeled with a $P$, $L$ and $R$,
  such that for every pair $i, i' \in I$ such that $i'$ is the successor of $i$,
  $P(i) \ne P(i')$,
  and either $R(i) = 0$ or $L(i') = 0$.

  Let $\sigma(i) \in \set{1, \omega, \agemo, \oo}$ be such that
  $L(i) = 1$ iff $\sigma(i) \in \set{1, \omega}$ and
  $R(i) = 1$ iff $\sigma(i) \in \set{1, \agemo}$.

  Then, $C = \bigcup_{I \in \mathbf{G}} \sum_{i \in I} \mathcal{S}^{\sigma(i)}_{\alpha}$.
\end{lemma}

\begin{proof}
  ($\subseteq$) Let $M$ be a countable linear order labeled with $P$, $L$ and $R$ as above.

  Let $I = M / \sim_{\alpha}$ be the quotient of $M$ by the equivalence relation $\sim_{\alpha}$.

  Then $M = \sum_{i \in I} M_i$,
  where $\set{M_i}_{i \in I}$ are the $\sim_{\alpha}$-equivalence class of $I$.

  Then for each $i \in I$, $M_i \in \bounded{\qqo^{< \alpha}}$,
  and by definition $\sigma(i) = \sigma_{\alpha} \ps{M_i}$.

  Let $i'$ be the successor of $i$ in $I$.

  Then $P(i) \ne P(i')$ since $P$ represents $\sim_{\alpha}$.

  Furthermore, suppose $R(i) = L(i') = 1$ holds.
  Then $M_i \in \rb{\qqo^{< \alpha}}$ and $M_{i'} \in \lb{\qqo^{< \alpha}}$.
  so $M_i$ and $M_{i'}$ are the same $\sim_{\alpha}$-equivalence class of $M$,
  which is a contradiction.
  
  Thus either $R(i) = 0$ or $L(i') = 0$.

  ($\supseteq$) Let $M = \sum_{i \in I} M_i$ be a linear order
  such that $I \in \mathbf{G}$ and $M_i \in \mathcal{S}^{\sigma(i)}_{\alpha}$ for each $i \in I$.

  In particular $M_i \in \bounded{\qqo^{< \alpha}}$ for each $i \in I$, so it is contained
  in a single $\sim_{\alpha}$-equivalence class of $M$.

  Suppose that there exist distinct $j ,k \in I$ such that $j < k$, and
  $M_j, M_k$ are in the same $\sim_{\alpha}$-equivalence class.

  Let $x \in M_j$ and $y \in M_k$.
  Then $[x, y] \in \qqo^{< \alpha}$, 
  and thus $[j, k] \in \qqo^{< \alpha}$,
  and in particular it is sparse.
  
  Then there exist some $j', k' \in I$ such that $j < j' < k' < k$,
  and $k'$ is the successor of $j'$ in $I$.

  Then $M_{j'}$ and $M_{k'}$ are in the same $\sim_{\alpha}$-equivalence class.
  Thus it must be the case that $M_{j'} \in \rb{\qqo^{< \alpha}}$ and $M_{k'} \in \lb{\qqo^{< \alpha}}$,
  which implies $R(j') = L(k') = 1$, which is a contradiction.

  Thus $\set{M_i}_{i \in I}$ are pairwise distinct $\sim_{\alpha}$-equivalence classes,
  and obviously the conditions holds,
  so $M \in C$ and we are done.
\end{proof}

\begin{corollary}\label{countables-computable}
  Let $C$ be defined as in~\cref{single-ordinal-decomposition}.

  Then $C$ is a computable property.
\end{corollary}

\begin{theorem}\label{single-ordinal-satisfiability}
  Let $\alpha > 0$ be an ordinal.

  Satisfiability of $\mso[\mathbf{Int}_{\qqo^{< \alpha}}]$ 
  over all countable linear orders is decidable.
\end{theorem}

\begin{proof}
  First, by~\cref{int-expressible}, we can convert
  any formula in $\mso[\mathbf{int}_{\qqo^{< \alpha}}]$
  to an equivalent formula $\varphi$ in $\mso[\bs{\cdot}_{\alpha}, L_{\alpha}, R_{\alpha}]$.

  Now, we shall replace every occurrence of $\bs{\cdot}_{\alpha}$ in $\varphi$ with $P$,
  every occurrence of $L_{\alpha}$ with $L$,
  and every occurrence of $R_{\alpha}$ with $R$,
  getting a new formula $\varphi'$.

  Then, satisfiability of $\varphi$ over all countable linear orders,
  amounts to satisfiability of $\varphi'$ over $C$,
  which is computable by~\cref{countables-computable}.

  Thus we can compute $\type{n}{C}$ and $\type{n}{\varphi'}$,
  and thus we can compute whether $\varphi$ is satisfiable over all countable linear orders,
  by seeing if these sets intersect.
\end{proof}

\begin{theorem}
  Let $\alpha, \delta_1, \ldots, \delta_k$ be ordinals.

  Let $\alpha_i = \alpha + \delta_i$ for $i = 1, \ldots, k$.

  Let $C$ be the class of all countable linear orders
  labeled with $\pi_{\alpha}$ and $\sigma_{\alpha}$,
  and $\pi_{\alpha_i}$ and $\sigma_{\alpha_i}$ for $i = 1, \ldots, k$.

  Let $\mathbf{G}$ be the class of all countable linear orders $I$,
  labeled with a coloring function $\gamma$
  whose range is $\set{1, \omega, \agemo, \oo}$,
  such that for pair $i, j \in I$ such that $j$ is the successor of $i$,
  either $\gamma(i) \in \set{\omega, \oo}$
  or $\gamma(j) \in \set{\agemo, \oo}$.
\end{theorem}

\begin{proof}
  TBC.
\end{proof}
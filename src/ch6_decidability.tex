\section{Decidability of the rank}

\begin{definition}
  Let $\alpha \ge \omega$ be an ordinal.

  Let $M$ be a linear order and $x \in M$.

  We define the convex equivalence relation:
  \[\sim_\alpha := \sim_{\bounded{\qqo^{\alpha}}}\]
  and $\bs{x}_{\alpha} := \bs{x}_{\bounded{\qqo^{\alpha}}}$ (that is,
  $\bs{x}_{\alpha}$ is the largest $\bounded{\qqo^{\alpha}}$-subinterval
  containing $x$ in $M$).

  We define $\sigma_{\alpha} \parens{M}$ as
  the $\alpha$-shape of $M$.
\end{definition}

\begin{lemma}
  The property
  $\qqo^{= \alpha}$ is expressible over intervals
  in $\mso[\sim_{\alpha}, \sigma_{\alpha}]$.

  That is, there exists a formula $\varphi_{\alpha}(\Pi, \Xi)$
  such that for every linear order $M$ and every $\bounded{\qqo^{\alpha}}$-subinterval
  $I$ of $M$, we have
  \[
    M, \Pi, \Xi \models \varphi_{\alpha}(\Pi, \Xi) \iff
    I = \sum_{i \in I} M_i
    \text{ where } M_i \in \qqo^{= \alpha} \text{ for all } i
  \]
\end{lemma}

\begin{proof}
  It is equivalent to being a sum of $\sim_{\alpha}$-subintervals,
  of which at least one has $\sigma_{\alpha} \ne 1$.
\end{proof}

\begin{theorem}
  There is an oracle reduction from SAT for $\mso[\sim_{\alpha}, \sigma_{\alpha}]$,
  to SAT for $\mso$.
\end{theorem}

\begin{proof}
  By the decomposition theorem, there exists a translation,
  that given an $\mso[\sim_{\alpha}, \sigma_{\alpha}]$ formula $\varphi$ of quantifier-depth $n$.
  outputs an $\mso$ formula $\psi(\Pi)$ such that...

  Let $\varphi$ be an $\mso[\sim_{\alpha}, \sigma_{\alpha}]$ formula,
  and let $n$ be the quantifier-depth of $\varphi$.

  WLOG, assume that $\varphi$ is a sentence.

  First, let us calculate the sets:

  \[T_s := \type{n}{\mathcal{S}^{s}_{\alpha}}\]
  for every shape $s$.

  Now we create the formulae:
  \[\theta_s(\Pi, \Xi) := \set{i : \bigvee_{\tau \in S_s}{\Xi(\Pi(i)) = s}}\]
  \[L(\Pi, \Xi) := \theta_{\omega}(\Pi, \Xi) \vee \theta_{\oo}(\Pi, \Xi)\]
  \[R(\Pi, \Xi) := \theta_{\agemo}(\Pi, \Xi) \vee \theta_{\oo}(\Pi, \Xi)\]

  We create the formula $\chi(\Pi, \Xi)$ as follows:

  \[\chi := \Pi = domain(\Xi) \wedge \forall i, i'. i' = i + 1 \implies {i \in R(\Pi, \Xi) \vee i' \in L(\Pi, \Xi)}\]

  Now we claim that $\varphi$ is satisfiable in $\mso[\sim_{\alpha}, \sigma_{\alpha}]$
  iff $\psi \land \chi$ is satisfiable in $\mso$.

  If $\varphi$ is satisfiable, then there exists a model $M$ of $\varphi$.

  Let $M = \sum_{i \in I} M_i$ be the decomposition of $M$
  where $I = \sim_{\alpha}$ and $M_i$ are the $\sim_{\alpha}$-equivalence classes.

  By the decomposition theorem, $\Psi$ holds
  in $I, \Pi := \type{n}{\cdot}$.

  We claim that $\chi$ holds in $I, \Pi := \type{n}{\cdot}$.

  It follows from the star property of $\sim_{\alpha}$ that the constraint holds.

  Conversely, suppose $\psi \land \chi$ is satisfiable in $\mso$.

  Let $I, \Pi := T$ be a model of $\psi \land \chi$.

  Let us take a model $M_i$ with the appropriate type.
  Now define $M := \sum_{i \in I} M_i$.

  We claim that each $M_i$ is a \emph{maximum} $\bounded{\qqo^{\alpha}}$-subinterval
  of $M$.

  Suppose $[M_i, M_j]$ is a $\bounded{\qqo^{\alpha}}$-subinterval of $M$.

  In particular, it has a rank, so it is scattered. So in particular,
  $[i, j] \subseteq I$ is a scattered interval.

  If $i = j$ we are done. Otherwise, let $i'$, $j'$ be such that $i \le i' < j' \le j$,
  and $j' = i' + 1$.  But it cannot be the case by the constraint.


\end{proof}

\begin{definition}
  Let $\alpha_1, \ldots, \alpha_k$ be ordinals.

  We define $C \bs{\alpha_1, \ldots, \alpha_k}$ as the class of
  countable linear orders, labeled with
  $\pi_{\alpha_i}$ and $\sigma_{\alpha_i}$ for $1 \le i \le k$.
\end{definition}

\begin{theorem}
  Let $\alpha_1, \ldots, \alpha_k$ be ordinals.

  Let $\alpha$ be an ordinal such that $\alpha < \alpha_i$
  for all $1 \le i \le k$.

  Let $\delta_i > 0$ for $1 \le i \le k$ be such that
  $\alpha_i = \alpha + \delta_i$.



  Let $\pp$ be the class of countable linear orders,

  Then $C_0 = \sum_{\pp}{C_1}$.

\end{theorem}



\begin{definition}
  A property $\pp$ of preorders is called a
  \emph{computable} property if
  $\type{n}{\pp}$ is computable as a function of $n$.
\end{definition}

\begin{theorem}[Decomposition theorem]
  There exists a computable translation $\mathcal{T}$
  from $MSO$ formulae to $MSO$ formulae,

  such that for any $M = \sum_{i \in I} M_i$,
  formula $\varphi \ps{\vec{X}}$,
  vector $\vec{A}$ of the same length as $\vec{X}$,
  if $n$ is the quantifier-depth of $\varphi$,
  then
  \[
    M, \vec{X} := \vec{A} \models \varphi \iff I, \Pi \models \mathcal{T} {\varphi}
  \]

  where $\Pi \ps{i} = \type{n}{M_i}$.
\end{theorem}

\begin{lemma}
  There exists a global computable function
  $h : \NN \to \NN$ such that the following holds.

  Let $\set{C_i}_{i=1}^k$ be a finite set of colors.

  Let $\pp$ be a property of linear orders, labeled
  by the colors $\set{C_i}_{i=1}^k$.

  Let $\set{\qq_i}_{i=1}^k$ be a finite set of properties of linear orders.

  Then $\type{n}{\sumv{\pp}{\vec{C}}{\vec{\qq}}}$ is
  a computable function of $\type{h(n)}{\pp}$
  and $\type{n}{\vec{\qq}} = \set{\type{n}{\qq_i}}_{i=1}^k$.
\end{lemma}

\begin{proof}
  TBC.
\end{proof}

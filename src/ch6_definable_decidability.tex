\section{Decidability of Definable Intervals}

\begin{definition}
    Let $\dfn$ be the class of all linear orders defined
    by an $\mso$-formula
\end{definition}


\begin{theorem}
    Let $\varphi$ be an $\mso$-sentence. The following are equivalent:
    \begin{enumerate}
        \item $\varphi$ has at least $2^{\aleph_0}$ models.
        \item $\varphi$ has uncountably many models.
        \item $\varphi$ has an undefinable model.
        \item $\varphi$ has an undefinable model of finite $\hrank{}$.
    \end{enumerate}
\end{theorem}

\begin{proof}
    ($1 \implies 2$) Trivial.
    ($2 \implies 3$) Since there are only countably many
    definable linear orders, if $\varphi$ has uncountably many models,
    then there is an undefinable model.

    ($3 \implies 4$)
    Let $M$ be an undefinable model of $\varphi$
    and let $n = \qd{\varphi}$.

    $M$ is $n$-equivalent to some linear order of finite $\hrank{}$,
    which is obviously undefinable, since $\type{n}{M}$ is not
    a defining formula.

    ($4 \implies 1$)
    Let $M$ be an undefinable model of $\varphi$
    with $\hrank{M} = a$
    and let $n = \qd{\varphi}$.

    We proceed by induction on $a$.

    If $a = 0$, the claim is vacuously true, since $M = 1$.

    Otherwise, let $M = \sum_{i \in I} M_i$,
    where $I = M / \sim_{a}$ and $\set{M_i}_{i \in I}$
    are the equivalence classes of $\sim_{a}$.

    If some $M_i$ is undefinable, then $\type{n}{M_i}$
    is not a defining formula, so by the induction
    hypothesis $\type{n}{M_i}$ has at least
    $2^{\aleph_0}$ models.

    Thus, we can replace $M_i$ with one of these models.
    The resulting models are all non-isomorphic,
    since they are distinguishable by the formula saying
    "the $i$-th equivalence class satisfies $\type{n}{M}$".

    Otherwise, all $M_i$ are definable. Then $I = I_1 + \cdots + I_k$,
    where $I_j \in \set{1, \om, \mo}$ for all $j \in [k]$.

    Let $N_j = \sum_{i \in I_j} M_i$.

    Let $j \in [k]$. WLOG $I_j = \omega$.
    Then $(x, y) \mapsto \type{n}{(x, y]}$ (where $x < y$) induces an additive coloring of
    the $N_j$, so by Shelah's theorem
    there is a cofinal homogenous set,
    i.e. $\type{n}{N_j} = \type{n}{(x, y]} \cdot \omega$ for some
    $x, y \in N_j$. But $(x, y]$ is a a subinterval
    of a finite sum of definable linear orders,
    and thus definable itself.

    Therefore, $N_j$ is definable, since
    definable linear orders are closed under $(\cdot \omega)$.

    Finally, $M = N_1 + \cdots + N_k$ is a finite sum of definable linear orders,
    and thus definable itself, contrary to the assumption that $M$ is undefinable.

    So we are done.
\end{proof}
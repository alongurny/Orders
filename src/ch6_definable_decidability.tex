\section{Decidability of Definable Intervals}

\begin{definition}
    Let $\dfn$ be the class of all linear orders defined
    by an $\mso$-formula
\end{definition}

\begin{notation}
    Let $I$ be a linear order and let $M = \sum_{i \in I} M_i$ be a linear order.

    Let $J \subseteq I$ be a subinterval.

    Then we denote by $M_J$ the linear order
    $\sum_{i \in J} M_i$.
\end{notation}


\begin{theorem}\label{undefinable-scattered-models}
    Let $\varphi$ be an $\mso$-sentence. The following are equivalent:
    \begin{enumerate}
        \item $\varphi$ has at least $2^{\aleph_0}$ scattered models.
        \item $\varphi$ has uncountably many scattered models.
        \item $\varphi$ has an undefinable scattered model.
    \end{enumerate}
\end{theorem}

\begin{proof}
    ($1 \implies 2$) Trivial.
    ($2 \implies 3$) Since there are only countably many
    definable linear orders,
    if $\varphi$ has uncountably many scattered models,
    then one of them is undefinable.

    ($3 \implies 1$)
    Let $M$ be an undefinable scattered model of $\varphi$
    with $\alpha = \hrank{M}$.

    Let $n = \qd{\varphi}$.

    We proceed by induction on $\alpha$.

    If $\alpha = 0$, $M$ is definable, contrary to the assumption.

    Otherwise, let $M = \sum_{i \in I} M_i$,
    where $I$ is a scattered linear order
    and each $M_i$ is a scattered linear order with $\hrank{M_i} < \alpha$.

    If all $M_i$ are definable. Then $I = I_1 + \cdots + I_k$,
    where $I_j \in \set{1, \om, \mo}$ for all $j \in [k]$.

    Let $N_j = \sum_{i \in I_j} M_i$.

    Let $j \in [k]$. WLOG $I_j = \omega$.
    Then $(i_1, i_2) \mapsto \type{n}{M_{(i_1, i_2]}}$ (where $i_1 < i_2$)
    induces an additive coloring of $I_j$, so by Shelah's theorem
    there is a cofinal homogenous set,
    i.e. $\type{n}{N_j} = \type{n}{M_{(i_1, i_2]}} \cdot \omega$ for some
    $i_1, i_2 \in N_j$. But $M_{(i_1, i_2]}$ is a subinterval
    of a finite sum of definable linear orders,
    and thus definable itself.

    Therefore, $N_j$ is definable, since
    definable linear orders are closed under $(\cdot \omega)$.

    Finally, $M = N_1 + \cdots + N_k$ is a finite sum of definable linear orders,
    and thus definable itself, contrary to the assumption that $M$ is undefinable.

    Thus, some $M_i$ is undefinable, by the induction
    hypothesis $\type{n}{M_i}$ has at least
    $2^{\aleph_0}$ models.

    We claim that if $N_1, N_2$ are two different models of $\type{n}{M_i}$,
    then replacing $M_i$ with $N_1$ or $N_2$ in $M$
    results in two different models of $\varphi$.
    This is because any isomorphism must map
    between the equivalence classes of $\sim_{\alpha}$,
    and thus must map $N_1$ to $N_2$.

    Thus, we have at least $2^{\aleph_0}$ models of $\varphi$,
    so we are done.
\end{proof}

\begin{theorem}\label{undefinable-models}
    Let $\varphi$ be an $\mso$-sentence. The following are equivalent:
    \begin{enumerate}
        \item $\varphi$ has at least $2^{\aleph_0}$ models.
        \item $\varphi$ has uncountably many models.
        \item $\varphi$ has an undefinable model.
    \end{enumerate}
\end{theorem}

\begin{proof}
    ($1 \implies 2$) Trivial.

    ($2 \implies 3$) Since there are only countably many
    definable linear orders,
    if $\varphi$ has uncountably many models,
    then one of them is undefinable.

    ($3 \implies 1$)
    Let $M$ be an undefinable model of $\varphi$.

    If $M$ is scattered, then by~\cref{undefinable-scattered-models}
    we are done.

    Otherwise, $M = \sum_{i \in I} M_i$,
    where each $M_i$ is scattered, and
    \[I \in \set{\eta, 1 + \eta, \eta + 1, 1 + \eta + 1}\]
    (where $\eta = \otp{\QQ}$).
    
    Let us choose one representative of each $n$-type occurring
    in $\set{\type{n}{M_i}}_{i \in I}$, and replace
    each $M_i$ with the representative.
    
    This results in an $n$-equivalent
    structure, which in particular is still a model of $\varphi$.

    So WLOG, we can assume that $M$ is an $\eta$-shuffle of
    finitely many scattered models.

    If all $M_i$ are definable, then so is $M$, in contrary to the assumption.

    Otherwise, some $M_i$ is undefinable.
\end{proof}
\section{Decidability of Definable Intervals}

\begin{definition}
    Let $\dfn$ be the class of all linear orders defined
    by an $\mso$-formula
\end{definition}

\begin{definition}
    Let $\kappa$ be a cardinal.

    Let $\varphi$ be an $\mso$-formula.
    We say that $\varphi$ is \emph{$\le \kappa$-ambiguous}
    if it has at most $\kappa$ many models.
\end{definition}

\begin{definition}
    Let $\kappa$ be a cardinal.

    Let $M$ be a linear order.
    We say that $M$ is \emph{$\le \kappa$-ambiguous}
    if it is a model of some $\le \kappa$-ambiguous $\mso$-formula.
\end{definition}

\begin{lemma}
    Let $I$ be $\omega$ labeled over $P_1, \ldots, P_k$, and suppose $I$ is $\le \kappa$-ambiguous for some cardinal $\kappa$.
    Let $F$ be a function assigning each truth vector in $2^k$ a definable linear order.

    Then $M = \sum_{i \in I} F(P_1(i), \ldots, P_k(i))$ is $\le \kappa$-ambiguous.
\end{lemma}

\begin{proof}
    It is immediate.
\end{proof}

\begin{lemma}
    Let $M_1$ be a $\le \kappa_1$-ambiguous linear order for some cardinal $\kappa_1$,
    and let $M_2$ be a $\le \kappa_2$-ambiguous linear order for some cardinal $\kappa_2$.

    Then $M_1 + M_2$ is $\le \kappa_1 \kappa_2$-ambiguous.
\end{lemma}

\begin{proof}
    It is immediate.
\end{proof}


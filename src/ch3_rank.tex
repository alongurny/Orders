\section{The Hausdorff Rank}

\begin{definition}
  \[
    \Om = \set{1, \om, \mo}^+
  \]
\end{definition}

\begin{observation}
  $\Om$ is a monotone, additive class of linear orders.
\end{observation}

% Definition of the Hausdorff power
\begin{definition}
  We define a class $\ho{< \alpha}$
  for every ordinal $\alpha$ as follows:

  \begin{itemize}
    \item For $\alpha = 0$, $\ho{< 0} = \emptyset$.
    \item For $\alpha = 1$, $\ho{< 1} = \set{1}$.
    \item For $\alpha = \gamma + 1$ where $\gamma > 0$,
          \[\ho{< \alpha} = \sum_{\Om}{\ho{< \gamma}}\]
    \item For $\alpha$ a limit ordinal,
          \[\ho{< \alpha} = \bigcup_{\beta < \alpha} \ho{< \beta}\]
  \end{itemize}

\end{definition}

% Definition of exact power
\begin{definition}
  Let $\alpha, \beta$ be ordinals such that with $0 < \alpha < \beta$.

  We define,
  \begin{enumerate}
    \item $\ho{\le \alpha} := \ho{< \alpha + 1}$
    \item $\ho{= \alpha} := \ho{\le \alpha} \setminus \ho{< \alpha}$
    \item $\ho{[\alpha, \beta)} := \ho{< \beta} \setminus \ho{< \alpha}$
  \end{enumerate}
\end{definition}

% Definition of the Hausdorff rank
\begin{definition}
  We define the Hausdorff rank as a \emph{partial} mapping
  from linear orders to ordinals, such that
  \[
    \hrank{M} = \min \set{\alpha : M \in \ho{\le \alpha}}
  \]

  Equivalently, \[
    \hrank{M} = \alpha \iff M \in \ho{= \alpha}
  \]

\end{definition}


\begin{definition}\label{bounded-definitions}
  Let $\alpha > 0$ be an ordinal.

  We define:
  \begin{enumerate}
    \item (Right $\alpha$-Major) $\rmj{\alpha} := \rb{\ho{< \alpha}} \setminus \lb{\ho{< \alpha}}$
    \item (Left $\alpha$-Major) $\lmj{\alpha} := \lb{\ho{< \alpha}} \setminus \rb{\ho{< \alpha}}$
    \item (Bounded $\alpha$-Major) $\bmj{\alpha} := \bounded{\ho{< \alpha}} \setminus \ps{ \lb{\ho{< \alpha}} \cup \rb{\ho{< \alpha}} }$
  \end{enumerate}
\end{definition}

\begin{note}
  Obviously $\lmj{\alpha} = \rmj{\alpha}^\ast$ by symmetry.

  By~\cref{bounded-is-left-plus-right}, $\bmj{\alpha} = \lmj{\alpha} + \rmj{\alpha}$.

  Also, by the definition and~\cref{bounded-classes},
  \[
    \bounded{\ho{< \alpha}} = \ho{< \alpha} \uplus \lmj{\alpha}
    \uplus \rmj{\alpha} \uplus \bmj{\alpha}
  \]
  \[
    \lmj{\alpha} \uplus \rmj{\alpha} \uplus \bmj{\alpha} \subseteq \ho{= \alpha}
  \]
\end{note}

\begin{lemma}\label{bounded-structure}
  Let $\alpha > 0$ be an ordinal.

  Then $\rb{\ho{< \alpha}} = \sum_{\om}{\ho{< \alpha}}$.
\end{lemma}

\begin{proof}
  ($\supseteq$) Let $M \in \sum_{\om}{\ho{< \alpha}}$ be a linear order.

  Let $M = \sum_{i \in \om} M_i$ be the decomposition of $M$,
  where $M_i \in \ho{< \alpha}$.

  Let $x, y \in M$ be any two points in $M$. WLOG $x \le y$.

  Suppose $x \in M_i$ and $y \in M_j$ for $i, j \in \om$.

  Since $i$ and $j$ have a finite distance in $\om$,
  we conclude $[x, y] \subseteq M_i + \ldots + M_j$,
  and thus $[x, y] \subseteq \ps{\ho{< \alpha}}^+ = \ho{< \alpha}$.

  ($\subseteq$) Let $M \in \rb{\ho{< \alpha}}$ be a linear order.

  Since $M$ is countable, let $\set{x_i}_{i \in \om} M$ be a right cofinal
  $\om$-sequence in $M$.

  Let $M_0 = (-\infty, x_0]$ and and $M_i = (x_{i-1}, x_{i}]$ for $i > 0$.

  Then $M = \sum_{i \in \om} M_i$.

  But $M_i$ is a right-bounded interval and thus $M_i \in \ho{< \alpha}$, so $M \in \sum_{\om}{\ho{< \alpha}}$.
\end{proof}

An immediate corollary of~\cref{bounded-structure} is that "major" is a good name, in the sense that every interval of rank $\alpha$ is a finite sum
of $\alpha$-major and "$\alpha$-minor" (i.e, of rank $< \alpha$) intervals.

\begin{corollary}\label{le-alpha-corollary}
  Let $\alpha > 0$ be an ordinal.

  Then,
  \[
    \ho{\le \alpha} = \ps{\bounded{\ho{< \alpha}}}^+
  \]
\end{corollary}

\begin{proof}
  By the definition,
  \begin{align*}
    \ho{\le \alpha} & = \sum_{\Om}{\ho{< \alpha}}                                                                                                            \\
                    & = \sum_{\fnt} \ps{ {\ho{< \alpha}} \uplus \sum_{\om}{\ho{< \alpha}} \uplus \sum_{\mo}{\ho{< \alpha}}  \uplus \sum_{\oo}{\ho{< \alpha}}
    }                                                                                                                                                        \\
                    & = \sum_{\fnt} \ps{\ho{< \alpha} \uplus \lmj{\alpha}
    \uplus \rmj{\alpha} \uplus \bmj{\alpha}}                                                                                                                 \\
                    & = \ps{\bounded{\ho{< \alpha}}}^+
  \end{align*}
\end{proof}

\begin{corollary}\label{eq-alpha-corollary}
  Let $\alpha > 0$ be an ordinal.

  Then,
  \begin{align*}
    \ho{= \alpha} & = \ho{\le \alpha}                                          \\
                  & + (\lmj{\alpha} \uplus \rmj{\alpha} \uplus \bmj{\alpha}) \\
                  & + \ho{\le \alpha}
  \end{align*}
\end{corollary}

\begin{proof}
  ($\supseteq$) is immediate.

  ($\subseteq$) Let $M \in \ho{= \alpha}$.

  Then $M$ is a finite sum of $\bounded{\ho{< \alpha}}$-intervals by the previous lemma,
  but not all of them are $\ho{< \alpha}$-intervals,
  otherwise we would have $M \in \ho{< \alpha}$, which is a contradiction.
\end{proof}

\begin{lemma}\label{rmj-decomposition}
  Let $\alpha > 0$ be an ordinal.

  Let $\set{\alpha_i}_{i \in \om}$ be a non-decreasing $\omega$-sequence of ordinals.

  Suppose $\sup_{i \in \om} \ps{\alpha_i + 1} = \alpha$.

  Then,
  \[
    \rmj{\alpha} = \sum_{i \in \om}{\ho{[\alpha_i, \alpha)}}
  \]
\end{lemma}

\begin{proof}
  ($\subseteq$) Let $M \in \rmj{\alpha}$.
  Let $\set{y_i}_{i < \om}$ be a right cofinal $\om$-sequence in $M$.

  Thus we can choose some $x_0$ far enough such that $(-\infty, x_0] \in \ho{[\alpha_0, \alpha)}$,
          and $x_0 > y_0$.
          Now by induction we choose $x_1$ such that $(x_0, x_1] \in \ho{[\alpha_1, \alpha)}$,
  and $x_1 > y_1$.

  By iterating $\om$ times we get an $\om$-sequence $\set{M_i}_{i \in \om}$
  such that $M = \sum_{i \in \om} M_i$ and $M_i \in \ho{[\alpha_i, \alpha)}$,
          where $M_i = (x_{i-1}, x_i]$ (where $x_{-1} := -\infty$).


  ($\supseteq$) Let $M \in \sum_{i \in \om}{\ho{[\alpha_i, \alpha)}}$.
  It is obvious that $M \in \rb{\ho{< \alpha}}$ since every right-bounded
  ray is a finite sum of $\ho{< \alpha}$-intervals.

  However, $M \notin \ho{< \alpha_i}$ for any $i \in \om$.
  If $\alpha$ is a limit ordinal, it implies $M \notin \ho{< \alpha}$,
  and thus $M \notin \lb{\ho{< \alpha}}$, so we are done.

  Otherwise, $\alpha$ is a successor ordinal. Say $\alpha = \gamma + 1$,
  and WLOG $\alpha_i = \gamma$ for all $i \in \om$.

  If $\gamma = 0$,
  then indeed $M = \sum_{\om}{1} = \omega \in \rmj{1}$.

  Otherwise, suppose for the sake of contradiction that $M \in \ho{< \alpha}$.
  That is, $M \in \ho{\le \gamma} = \ps{\bounded{\ho{< \gamma}}}^+$.

  Then $M$ can be decomposed into a finite sum of $\bounded{\ho{< \gamma}}$-intervals, the last of which contains almost all of the $M_i$,
  so WLOG $M \in \bounded{\ho{< \gamma}}$.

  In particular $M_1 \in \ho{< \gamma}$, since $M_1$ is bounded between
  $M_0$ and $M_2$, but $M_1 \in \ho{= \gamma}$ which is a contradiction.
\end{proof}

Obviously, we can use the decomposition of $\rmj{\alpha}$ to decompose
$\lmj{\alpha}$ and $\bmj{\alpha}$ as well,
as $\lmj{\alpha} = \rmj{\alpha}^\ast$ and $\bmj{\alpha} = \lmj{\alpha} + \rmj{\alpha}$.
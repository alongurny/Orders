\section{General Hausdorff Rank}

The concept of rank provides a powerful tool for measuring the complexity of linear orders and related structures. In this chapter, we introduce the Hausdorff rank and its generalizations, which allow us to stratify classes of orders according to their structural depth. The results here lay the groundwork for the analysis of types and decidability in subsequent chapters.

% Definition of the Hausdorff power
\begin{definition}
  Let $\qq$ be a class of linear orders.

  We define a class $\hh{\qq}{< \alpha}$
  for every ordinal $\alpha$ as follows:

  \begin{itemize}
    \item For $\alpha = 0$, $\hh{\qq}{< 0} = \emptyset$.
    \item For $\alpha = 1$, $\hh{\qq}{< 1} = \set{1}$.
    \item For $\alpha = \gamma + 1$ where $\gamma > 0$,
          \[\hh{\qq}{< \alpha} = \sum_{\qq}{\hh{\qq}{< \gamma}}\]
    \item For $\alpha$ a limit ordinal,
          \[\hh{\qq}{< \alpha} = \bigcup_{\beta < \alpha} \hh{\qq}{< \beta}\]
  \end{itemize}

\end{definition}

\begin{example}
  Let $\qq$ be a class of linear orders.

  Then $\hh{\qq}{\le 1} = \qq$.
\end{example}

% Definition of exact power
\begin{definition}
  Let $\qq$ be a class of linear orders.

  Let $\alpha, \beta$ be ordinals such that with $0 < \alpha < \beta$.

  We define,
  \begin{enumerate}
    \item $\hh{\qq}{\le \alpha} := \hh{\qq}{< \alpha + 1}$
    \item $\hh{\qq}{= \alpha} := \hh{\qq}{\le \alpha} \setminus \hh{\qq}{< \alpha}$
    \item $\hh{\qq}{[\alpha, \beta)} := \hh{\qq}{< \beta} \setminus \hh{\qq}{< \alpha}$
  \end{enumerate}
\end{definition}

% Definition of the Hausdorff rank
\begin{definition}
  Let $\qq$ be a class of linear orders.

  We define the $\qq$-Hausdorff rank as a \emph{partial} mapping
  from linear orders to ordinals, such that
  \[
    \hhrank{\qq}{M} = \min \set{\alpha : M \in \hh{\qq}{\le \alpha}}
  \]

  Equivalently, $\hhrank{\qq}{M}$, is the unique ordinal $\alpha$ such that
  $M \in \hh{\qq}{= \alpha}$ (if it exists, otherwise it is undefined).

\end{definition}


% Definition of $\Om$
\begin{definition}
  Let $\gamma \ge \om$ be a limit ordinal.

  We define $\Gamma_{\gamma} := \set{\beta : \beta \subseteq \gamma^\ast + \gamma}^+$.

  We define $\Om := \Gamma_{\om}$.
\end{definition}

\begin{example}
  \[
    \Om = \set{1, \om, \mo}^+
  \]
\end{example}

\begin{observation}
  Let $\gamma \ge \om$ be a limit ordinal.

  Then $\Gamma_{\gamma}$ is a monotone, additive class of linear orders.
\end{observation}


% Definition of the Hausdorff rank associated with $\om$.
\begin{notation}
  When we omit the subscript in $\ho{< \alpha}$,
  we mean $\hh{\Om}{< \alpha}$,
  and similarly for $\ho{\le \alpha}$, $\ho{= \alpha}$, $\ho{[\alpha, \beta)}$,
  and $\hrank{M}$.
\end{notation}
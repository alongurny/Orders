\section{General Hausdorff Rank}

The concept of rank provides a powerful tool for measuring the complexity of linear orders and related structures. In this chapter, we introduce the Hausdorff rank and its generalizations, which allow us to stratify classes of orders according to their structural depth. The results here lay the groundwork for the analysis of types and decidability in subsequent chapters.

% Definition of the Hausdorff power
\begin{definition}
  Let $\qq$ be a property of linear orders.

  We define a property $\qq^{< \alpha}$
  for every ordinal $\alpha$ as follows:

  \begin{itemize}
    \item For $\alpha = 0$, $\qq^{< 0} = \set{1}$.
    \item For $\alpha = \gamma + 1$,
          \[\qq^{< \alpha} = \sum_{\qq}{\qq^{< \gamma}}\]
    \item For $\alpha$ a limit ordinal,
          \[\qq^{< \alpha} = \bigcup_{\beta < \alpha} \qq^{< \beta}\]
  \end{itemize}
\end{definition}

\begin{example}
  Let $\qq$ be a property of linear orders.

  Then $\qq^{< 1} = \qq$.
\end{example}

% Discussion
The Hausdorff power construction enables us to build increasingly complex properties by iterating the sum operation. The
following example illustrates the base case of this construction.

\begin{lemma}\label{sum-of-ranks}
  Let $\qq$ be a property of linear orders.

  Let $\alpha, \delta$ be ordinals.

  Then,
  \[
    \qq^{< \alpha + \delta}
    = \sum_{\qq^{< \delta}}{\qq^{< \alpha}}
  \]
\end{lemma}

% Discussion
The next lemma provides a key formula for decomposing the Hausdorff power, which will be used repeatedly in the analysis of
ranks.

\begin{proof}
  We shall prove this by induction on $\delta \ge 0$.

  For $\delta = 0$ we need to prove
  \[
    \qq^{< \alpha} = \sum_{\qq^0}{\qq^{< \alpha}}.
  \]

  Which is true by definition, since $\qq^0 = \set{1}$.

  For $\delta = \gamma + 1$, using the induction hypothesis,
  \begin{align*}
    \qq^{< \alpha + \delta}
     & = \qq^{< \alpha + \gamma + 1}                       \\
     & = \sum_{\qq}{\qq^{< \alpha + \gamma}}               \\
     & = \sum_{\qq}{\sum_{\qq^{< \gamma}}{\qq^{< \alpha}}} \\
     & = \sum_{\sum_{\qq}{\qq^{< \gamma}}}{\qq^{< \alpha}} \\
     & = \sum_{\qq^{< \gamma + 1}}{\qq^{< \alpha}}         \\
     & = \sum_{\qq^{< \delta}}{\qq^{< \alpha}}
  \end{align*}

  For $\delta$ a limit ordinal, using the induction hypothesis,
  \begin{align*}
    \qq^{< \alpha + \delta}
     & = \bigcup_{\gamma < \delta}{\qq^{< \alpha + \gamma}}               \\
     & = \bigcup_{\gamma < \delta}{\sum_{\qq^{< \gamma}}{\qq^{< \alpha}}} \\
     & = \sum_{\bigcup_{\gamma < \delta}{\qq^{< \gamma}}}{\qq^{< \alpha}} \\
     & = \sum_{\qq^{< \delta}}{\qq^{< \alpha}}
  \end{align*}

\end{proof}

% Definition of exact power
\begin{definition}
  Let $\qq$ be a property of linear orders.

  Let $\alpha, \beta$ be ordinals with $\alpha < \beta$.

  We define,
  \begin{enumerate}
    \item $\qq^{\le \alpha} := \qq^{< \alpha + 1}$
    \item $\qq^{= \alpha} := \qq^{\le \alpha} \setminus \qq^{< \alpha}$
    \item $\qq^{[\alpha, \beta)} := \qq^{< \beta} \setminus \qq^{< \alpha}$
  \end{enumerate}
\end{definition}

% Discussion
We now introduce the notion of exact power and the Hausdorff rank itself, which formalize the idea of measuring the
complexity of a linear order relative to a given property.

% Definition of the Hausdorff rank
\begin{definition}
  Let $\qq$ be a property of linear orders.

  Let $M$ be a linear order, such that
  $M \in \ps{\qq^{< \alpha}}^+$ for some ordinal $\alpha$.

  We define the \emph{$\qq$-Hausdorff rank} of $M$ as
  \[
    \hrank{\qq}{M} = \sup \set{\beta : M \notin \ps{\qq^{< \beta}}^{+}}
  \]

  where the supremum is taken over all ordinals $\beta$.
  (Recall that the supremum of the empty set is defined to be $0$.)

\end{definition}
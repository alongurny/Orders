
\section{General Hausdorff Rank}

% Definition of the Hausdorff power
\begin{definition}
  Let $\qq$ be a property of linear orders.

  We define a property $\qq^{\alpha}$
  for every ordinal $\alpha$ as follows:

  \begin{itemize}
    \item For $\alpha = 0$, $\qq^{0} = \set{1}$.
    \item For $\alpha = \gamma + 1$,
          \[\qq^{\alpha} = \sum_{\qq}{\qq^{\gamma}}\]
    \item For $\alpha$ a limit ordinal,
          \[\qq^{\alpha} = \bigcup_{\beta < \alpha} \qq^{\beta}\]
  \end{itemize}
\end{definition}

\begin{example}
  Let $\qq$ be a property of linear orders.

  Then $\qq^{1} = \qq$.
\end{example}

\begin{lemma}\label{sum-of-ranks}
  Let $\qq$ be a property of linear orders.

  Let $\alpha, \delta$ be ordinals.

  Then,
  \[
    \qq^{\alpha + \delta}
    = \sum_{\qq^{\delta}}{\qq^{\alpha}}
  \]
\end{lemma}

\begin{proof}
  We shall prove this by induction on $\delta \ge 0$.

  For $\delta = 0$ we need to prove
  \[
    \qq^{\alpha} = \sum_{\qq^0}{\qq^{\alpha}}.
  \]

  Which is true by definition, since $\qq^0 = \set{1}$.

  For $\delta = \gamma + 1$, using the induction hypothesis,
  \[
    \begin{aligned}
      \qq^{\alpha + \delta}
       & = \qq^{\alpha + \gamma + 1}                     \\
       & = \sum_{\qq}{\qq^{\alpha + \gamma}}             \\
       & = \sum_{\qq}{\sum_{\qq^{\gamma}}{\qq^{\alpha}}} \\
       & = \sum_{\sum_{\qq}{\qq^{\gamma}}}{\qq^{\alpha}} \\
       & = \sum_{\qq^{\gamma + 1}}{\qq^{\alpha}}         \\
       & = \sum_{\qq^{\delta}}{\qq^{\alpha}}
    \end{aligned}
  \]

  For $\delta$ a limit ordinal, using the induction hypothesis,
  \[
    \begin{aligned}
      \qq^{\alpha + \delta}
       & = \bigcup_{\gamma < \delta}{\qq^{\alpha + \gamma}}             \\
       & = \bigcup_{\gamma < \delta}{\sum_{\qq^{\gamma}}{\qq^{\alpha}}} \\
       & = \sum_{\bigcup_{\gamma < \delta}{\qq^{\gamma}}}{\qq^{\alpha}} \\
       & = \sum_{\qq^{\delta}}{\qq^{\alpha}}
    \end{aligned}
  \]

\end{proof}

% Definition of exact power
\begin{definition}
  Let $\qq$ be a property of linear orders.

  Let $\alpha$ be an ordinal.

  We define $\qq^{= \alpha} := \qq^{\alpha + 1} \setminus \qq^{\alpha}$.
\end{definition}

% Definition of the Hausdorff rank
\begin{definition}
  Let $\qq$ be a property of linear orders.

  Let $M$ be a linear order, such that
  $M \in \ps{\qq^{\alpha}}^+$ for some ordinal $\alpha$.

  We define the \emph{$\qq$-Hausdorff rank} of $M$ as
  \[
    \hrank{\qq}{M} = \sup \set{\beta : M \notin \ps{\qq^{\beta}}^{+}}
  \]

  where the supremum is taken over all ordinals $\beta$.
  (Recall that the supremum of the empty set is defined to be $0$.)

\end{definition}

\begin{example}
  Let $\qq$ be a property of linear orders.

  Let $M$ be a linear order.

  Then $\hrank{\qq}{M} = 0$ if and only $M$ is finite.
\end{example}

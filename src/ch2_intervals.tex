\section{Linear Orders}

In this chapter we focus on linear orders, also known as
total orders, intervals and chains.

% Definition of a linear order
\begin{definitions}[Linear order]
  A \emph{linear order} is a preorder which is antisymmetric and total.
\end{definitions}

% Definition of a class
\begin{definition}[class of linear orders]
  A \emph{class} $\pp$ of linear orders is a class of linear orders which
  is closed under isomorphism.
\end{definition}

% Definition of subintervals and bounded subintervals
\begin{definition}[Subintervals]
  Let $M$ be a linear order,
  and let $x, y \in M$, such that $x \le y$.

  Then we define the \emph{bounded subintervals} $[x, y]$,
  $(x, y]$, $[x, y)$ and $(x, y)$ as usual.

          We also define the \emph{semi-bounded subintervals} $(-\infty, x]$,
  $[x, \infty)$, $(-\infty, x)$ and $(x, \infty)$ as usual.

  We also define \emph{the unbounded subinterval} $(-\infty, \infty)$ as the whole linear order $M$,
  as usual.

  A \emph{subinterval} is either
  a bounded subinterval, a semi-bounded subinterval or the unbounded subinterval.

  If $x > y$ then we define the intervals as follows:
  \begin{align*}
    [x, y] & := [y, x] \\
    (x, y] & := (y, x] \\
    [x, y) & := [y, x) \\
    (x, y) & := (y, x)
  \end{align*}

\end{definition}

% Definition of left/right/bi-directionally cofinal
\begin{definition}
  Let $M$ be a linear order.

  A set $A \subseteq M$ is \emph{left cofinal} in $M$ if for every $x \in M$,
  there exists $y \in A$ such that $y < x$.

  A set $A \subseteq M$ is \emph{right cofinal} in $M$ if for every $x \in M$,
  there exists $y \in A$ such that $x < y$.

  A set $A \subseteq M$ is \emph{bi-directionally cofinal} in $M$ if it is both left and right cofinal.
\end{definition}

\begin{lemma}\label{additive-interval}
  Let $\pp$ be an additive class of linear orders.

  Let $M \in \pp$ be a linear order.

  Let $x, y \in M$ be any two points in $M$.

  Then, $[x, y] \in \pp$.
\end{lemma}

\begin{proof}
  WLOG, suppose $x \le y$.

  Note that,
  \[
    M = (-\infty, \infty) = (-\infty, x) + [x, y] + (y, \infty)
  \]

  when $(-\infty, x)$ and/or $(y, \infty)$ may be empty.

  Since $\pp$ is an additive class, we conclude that $[x, y] \in \pp$.
\end{proof}

\begin{corollary}\label{additive-1}
  Let $\pp$ be a nontrivial additive class of linear orders.

  Then $1 \in \pp$.
\end{corollary}

\begin{proof}
  Let $M \in \pp$ be any linear order and let
  $x \in M$ be any point in $M$.

  Apply~\cref{additive-interval} to the linear order $M$,
  and the points $x$ and $x$, to conclude that
  $[x, x] \equiv 1 \in \pp$.
\end{proof}

\begin{note}
  Note that~\cref{additive-1} is false if we do not restrict ourselves to linear orders.

  For example, $\ps{\mathbf{1} \uplus \mathbf{1}}^+$ is a class of preorders
  which is additive, but does not contain $\mathbf{1}$.
\end{note}

\begin{corollary}\label{additive-endpoints}
  Let $\pp$ be an additive class of linear orders.

  Let $M$ be a linear order.

  Let $x, y \in M$ be any two points in a linear order $M$.
  Then the following are equivalent:

  \begin{enumerate}
    \item $(x, y) \in \pp$
    \item $(x, y] \in \pp$
    \item $[x, y) \in \pp$
    \item $[x, y] \in \pp$
  \end{enumerate}
\end{corollary}

\begin{proof}
  This is just applying the definition of an additive class
  to the orders $[x, y]$ and $1$.
\end{proof}

\begin{corollary}\label{additive-transitivity}
  Let $\pp$ be an additive class of linear orders.

  Let $M$ be a linear order.

  Let $x, y, z \in M$ be any three points in a linear order $M$,
  such that $[x, y] \in \pp$ and $[y, z] \in \pp$.

  Then $[x, z] \in \pp$.
\end{corollary}

\begin{proof}
  If $y \in [x, z]$, then $[x, z] = [x, y] + (y, z]$,
  and $(y, z] \in \pp$ by~\cref{additive-endpoints}.

  Otherwise, either $x \in [y, z]$ or $z \in [x, y]$.
  WLOG, suppose $z \in [x, y]$.

  Then $[x, y] = [x, z] + (z, y]$,
  so $[x, z] \in \pp$ by the fact that $\pp$ is additive.
\end{proof}

% Definition of $\bounded{\pp}$
\begin{definitions}
  Let $\pp$ be a class of linear orders.

  We define the following classes of linear orders:
  \begin{itemize}
    \item $\bounded{\pp}$ is the class of linear orders $M$ such that for every $x, y \in M$,
          the bounded subinterval $[x, y]$ is in $\pp$.
    \item $\lb{\pp}$ is the class of linear orders $M$ such that for every $x \in M$,
          the left-bounded ray $[x, \infty)$ is in $\pp$.
    \item $\rb{\pp}$ is the class of linear orders $M$ such that for every $x \in M$,
          the right-bounded ray $(-\infty, x]$ is in $\pp$.
  \end{itemize}
\end{definitions}

% Definition of a star class
\begin{definition}
  a class $\pp$ of linear orders is a \emph{star class} if
  for every linear orders $M$, and every family $\mathcal{F} \subseteq \pp$
  of subintervals of $M$ such that $J_1 \cap J_2 \ne \emptyset$
  for every $J_1, J_2 \in \mathcal{F}$, we have that
  $\bigcup \mathcal{F} \in \pp$.
\end{definition}

% Partition according to a star class
\begin{lemma}
  Let $\pp$ be a star class.

  Then for every linear order $M$,
  and every point $x \in M$, there exists a largest subinterval $J \subseteq M$ such that
  $J \in \pp$.

  Equivalently, we can define a convex equivalence relation $\sim_{\pp}$ on $M$ such that $x \sim_{\pp} y$ iff $[x, y] \in \pp$.

  That is,
  $x \sim_{\pp} y$ iff $x$ and $y$ are in the same largest $\pp$-subinterval.

\end{lemma}

\begin{proof}
  Let $J \subseteq M$ be the union of all $\bounded{\pp}$-subintervals containing $x$.
  All such subintervals intersect at $x$.

  Therefore, by the star lemma, $J$ is in $\bounded{\pp}$, and by definition
  $J$ is the largest $\pp$-subinterval containing $x$.

  Thus we can define the equivalence relation $\sim_{\pp}$ as above.
\end{proof}

% Lemma: $\bounded{\pp}$ is a star class
\begin{lemma}[Star Lemma]\label{star-lemma}
  Let $\pp$ be an additive class of linear orders.

  Then the class $\bounded{\pp}$ is a star class.
\end{lemma}

% Proof of the star lemma
\begin{proof}
  Let $M$ be a linear order,
  and let $\mathcal{F} \subseteq \bounded{\pp}$ be a family of subintervals of $M$.

  Let $[x, y] \subseteq \bigcup \mathcal{F}$ be any bounded subinterval. We need to prove
  it is in $\pp$.

  Suppose $x \in J_1$ and $y \in J_2$ for $J_1, J_2 \in \mathcal{F}$.

  Since $J_1 \cap J_2 \ne \emptyset$, we can take $z \in J_1 \cap J_2$.

  Then $[x, z] \subseteq J_1$ and $[z, y] \subseteq J_2$,
  and thus by the definition of $\bounded{\pp}$, $[x, z], [z, y] \in \pp$.
  Since $\pp$ is additive, by~\cref{additive-transitivity}, we have $[x, y] \in \pp$.
\end{proof}

\begin{lemma}\label{bounded-classes}
  Let $\pp$ be an additive class of linear orders.

  Then,
  \begin{enumerate}
    \item $\lb{\pp} = \set{M : M + 1 \in \bounded{\pp}}$
    \item $\rb{\pp} = \set{M : 1 + M \in \bounded{\pp}}$
    \item $\pp = \lb{\pp} \cap \rb{\pp} = \set{M : 1 + M + 1 \in \bounded{\pp}}$
  \end{enumerate}
\end{lemma}

\begin{proof}
  Let $M$ be a linear order.

  \begin{enumerate}
    % M + 1 case
    \item
          Suppose $M + \set{\infty} \in \bounded{\pp}$.
          Then for every $x \in M$, we have $[x, \infty] \in \pp$,
          and thus $[x, \infty) \in \pp$.
          Therefore, $M \in \lb{\pp}$.

          Conversely, if $M \in \lb{\pp}$,
          let $x, y \in M$ be any two points in $M + 1$.

          If $y < \infty$, then $[x, y] \subseteq [x, \infty)$.
          Since $[x, \infty) \in \pp$, we conclude that $[x, y] \in \pp$.
          Otherwise, if $y = \infty$, then $[x, y] = [x, \infty] = [x, \infty) + \set{\infty}$, and thus $[x, y] \in \pp$.

          % 1 + M case
    \item
          The second case is dual to the first case.

          % 1 + M + 1 case
    \item
          We will show a triple inclusion.

          If $M \in \pp$, then by additivity,
          $1 + M \in \pp$ and $M + 1 \in \pp$,
          and thus $M \in \lb{\pp} \cap \rb{\pp}$.

          If $M \in \lb{\pp} \cap \rb{\pp}$, then by~\cref{star-lemma},
          $1 + M + 1 \in \bounded{\pp}$.

          If $1 + M + 1 \in \bounded{\pp}$, then $M$ is a bounded
          subinterval of $1 + M + 1$, so $M \in \bounded{\pp}$.
  \end{enumerate}
\end{proof}

\begin{lemma}
  Let $\pp$ be an additive class of linear orders.

  Then,
  \begin{align*}
    \bounded{\pp} & = \pp                                                            \\
                  & \uplus \ps{\lb{\pp} \setminus \rb{\pp}}                          \\
                  & \uplus \ps{\rb{\pp} \setminus \lb{\pp}}                          \\
                  & \uplus \ps{ \bounded{\pp} \setminus \ps{\lb{\pp} \cup \rb{\pp}}}
  \end{align*}
\end{lemma}

\begin{proof}
  By~\cref{bounded-classes}, we conclude that
  $\lb{\pp}, \rb{\pp} \subseteq \bounded{\pp}$,
  since $M + 1 \in \pp$ and $1 + M \in \pp$ both imply $1 + M + 1 \in \pp$.

  Thus,
  \begin{align*}
    \bounded{\pp} & = \ps{\lb{\pp} \cap \rb{\pp}}                                    \\
                  & \uplus \ps{\lb{\pp} \setminus \rb{\pp}}                          \\
                  & \uplus \ps{\rb{\pp} \setminus \lb{\pp}}                          \\
                  & \uplus \ps{ \bounded{\pp} \setminus \ps{\lb{\pp} \cup \rb{\pp}}}
  \end{align*}

  Since by~\cref{bounded-classes} $\pp = \lb{\pp} \cap \rb{\pp}$,
  we conclude what we wanted to prove.
\end{proof}

\begin{lemma}
  Let $\pp$ be an additive class of linear orders.

  Let $M, M_1, M_2$ be linear orders such that
  $M = M_1 + M_2$.

  Then,
  \begin{enumerate}
    \item
          $M \in \bounded{\pp} \iff M_1 \in \lb{\pp} \wedge M_2 \in \rb{\pp}$
  \end{enumerate}
\end{lemma}

\begin{proof}
  From~\cref{bounded-classes}, we know that
  \begin{enumerate}
    \item
          \begin{align*}
            M \in \bounded{\pp}
             & \iff M_1 + M_2 \in \bounded{\pp}                                \\
             & \iff M_1 + 1 \in \bounded{\pp} \wedge 1 + M_2 \in \bounded{\pp} \\
             & \iff M_1 \in \lb{\pp} \wedge M_2 \in \rb{\pp}
          \end{align*}
  \end{enumerate}

\end{proof}

\begin{corollary}\label{bounded-is-left-plus-right}
  Let $\pp$ be an additive class of linear orders.

  Then,

  \[
    \bounded{\pp} \setminus \ps{\lb{\pp} \cup \rb{\pp}}
    = \ps{\lb{\pp} \setminus \rb{\pp}} + \ps{\rb{\pp} \setminus \lb{\pp}}
  \]
\end{corollary}

\begin{definition}
  We define $\cnt$ as the class of all countable linear orders.
\end{definition}

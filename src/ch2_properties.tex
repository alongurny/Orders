\section{Properties}
% Definition of left/right/bi-directionally cofinal
\begin{definition}
  Let $M$ be a linear order.

  A set $A \subseteq M$ is \emph{left cofinal} in $M$ if for every $x \in M$,
  there exists $y \in A$ such that $y \le x$.

  A set $A \subseteq M$ is \emph{right cofinal} in $M$ if for every $x \in M$,
  there exists $y \in A$ such that $x \le y$.

  A set $A \subseteq M$ is \emph{bi-directionally cofinal} in $M$ if it is both left and right cofinal.
\end{definition}

% Definitions of special orders
\begin{definition}
  Let $M$ be a linear order.

  Then $M^\ast$ is the dual - the reverse order of $M$.
\end{definition}

% Definition of a property
\begin{definition}
  A \emph{property} $\pp$ of linear orders is a class of linear orders which
  is closed under isomorphism.
\end{definition}


% Definition of extension of finitary operations
\begin{definition}
  Let $\kappa \ge 0$ be an ordinal.

  Let $\mathbf{op}$ be a $\kappa$-ary operation on linear orders.

  Let $\set{\pp_i}_{i \in \kappa}$ be properties of linear orders.

  Then,
  \[
    \mathbf{op} \ps{ \set{\pp_i}_{i \in \kappa} }
    := \set{ \mathbf{op} \ps{ \set{\pp_i}_{i \in \kappa} } : \forall i \in \kappa, \pp_i \in \pp }
  \]

  If $\mathbf{op}$ is replaced with a nondeterministic partial operation, 
  we define the extension of the operation as a deterministic
  operation which returns the image of the domain under the nondeterministic operation.
\end{definition}

% Definition of a monotone property
\begin{definition}
  A property $\pp$ of linear orders is \emph{monotone} if for every linear order $M$,
  $M \in \pp$ implies that every suborder of $M$ is in $\pp$.
\end{definition}

% Definition of a symmetric property
\begin{definition}
  A property $\pp$ of linear orders is \emph{symmetric} if for every linear order $M$,
  $M \in \pp$ iff $M^\ast \in \pp$, i.e., if $\pp^\ast = \pp$.
\end{definition}

% Definition of an additive property
\begin{definition}
  A property $\pp$ of linear orders is an \emph{additive property} if for every linear orders $M_1$ and $M_2$,
  $M_1 + M_2 \in \pp$ iff $M_1, M_2 \in \pp$, i.e., if $\pp + \pp = \pp$.
\end{definition}

% Definition of $\bounded{\pp}$
\begin{definition}
  Let $\pp$ be a property of linear orders.

  We define $\bounded{\pp}$ to be the class of linear orders $M$ such that for every $x, y \in M$,
  the bounded subinterval $[x, y]$ is in $\pp$.
\end{definition}

% Definition of an almost anti-symmetric property
\begin{definition}
  A property $\pp$ of linear orders is \emph{almost anti-symmetric}
  if for every linear order $M$,
  $M \in \pp$ and $M^\ast \in \pp$ imply that $M$ is finite.
\end{definition}

% Definition of a good property
\begin{definition}
  A property $\pp$ of linear orders is \emph{good} if it is
  monotone, additive and $\oo \in \pp$.
\end{definition}

% Definition of a star property
\begin{definition}
  A property $\pp$ of linear orders is a \emph{star property} if
  for every linear orders $M$, and every family $\mathcal{F} \subseteq \pp$
  of subintervals of $M$ such that $J_1 \cap J_2 \ne \emptyset$
  for every $J_1, J_2 \in \mathcal{F}$, we have that
  $\bigcup \mathcal{F} \in \pp$.
\end{definition}


% Lemma: $\bounded{\pp}$ is a star property
\begin{lemma}[Star Lemma]\label{star-lemma}
  Let $\pp$ be an additive property of linear orders.

  Then the property $\bounded{\pp}$ is a star property.
\end{lemma}

% Proof of the star lemma
\begin{proof}
  Let $M$ be a linear order,
  and let $\mathcal{F} \subseteq \bounded{\pp}$ be a family of subintervals of $M$.

  Let $[x, y] \subseteq \bigcup \mathcal{F}$ be any bounded subinterval. We need to prove
  it is in $\pp$.

  Suppose $x \in J_1$ and $y \in J_2$ for $J_1, J_2 \in \mathcal{F}$.

  Since $J_1 \cap J_2 \ne \emptyset$, we can take $z \in J_1 \cap J_2$.

  Then $[x, z] \subseteq J_1$ and $[z, y] \subseteq J_2$,
  and thus by $\bounded{\pp}$, $[x, z], [z, y] \in \pp$.
  However, $\pp$ is additive. Since $[x, y]$ is either the sum
  or difference of $[x, z]$ and $[z, y]$, we have that $[x, y] \in \pp$.
\end{proof}

% Partition according to a star property
\begin{lemma}
  Let $\pp$ be a star property.

  Then for every linear order $M$,
  and every point $x \in M$, there exists a largest subinterval $J \subseteq M$ such that
  $J \in \pp$.

  Equivalently, we can define a convex equivalence relation $\sim_{\pp}$ on $M$ such that $x \sim_{\pp} y$ iff $[x, y] \in \pp$.

  That is,
  $x \sim_{\pp} y$ iff $x$ and $y$ are in the same largest $\pp$-subinterval.

\end{lemma}

\begin{proof}
  Let $J \subseteq M$ be the union of all $\bounded{\pp}$-subintervals containing $x$.
  All such subintervals intersect at $x$.

  Therefore, by the star lemma, $J$ is in $\bounded{\pp}$, and by definition
  $J$ is the largest $\pp$-subinterval containing $x$.

  Thus we can define the equivalence relation $\sim_{\pp}$ as above.
\end{proof}

% Definition of sum
\begin{definition}
  Let $I$ be a linear order labeled with colors $\vec{C} = \set{C_k}_{k=1}^m$.
  Let $\gamma : I \to \vec{C}$ be the coloring function.

  Let $\vec{M} = \set{M_k}_{k=1}^m$ be a family of labeled linear orders.

  Then,
  \[\sumv{I}{\vec{C}}{\vec{M}}
    := \sum_{i \in I}{M_{\gamma(i)}}\]
\end{definition}

\begin{lemma}[Associativity of sum]
  Let $\pp_1$, $\pp_2$ and $\pp_3$ be properties.

  Then $\sum_{\pp_1}{\sum_{\pp_2}{\pp_3}} = \sum_{\sum_{\pp_1}{\pp_2}}{\pp_3}$.
\end{lemma}

\begin{proof}
  It follows directly from the associativity of the sum operation on
  linear orders. Actually, it generalizes to any algebraic equation
  which holds on linear orders.
\end{proof}

\begin{lemma}[Union-associativity of sum]
  Let $\ppp$ be a family of properties.
  
  Let $\qq$ be a property.

  Then $\sum_{\bigcup \ppp}{\qq} = \bigcup_{\pp \in \ppp}{\sum_{\pp}{\qq}}$.
\end{lemma}

\begin{proof}
  This is obvious from the definition of the sum operation.
\end{proof}

\begin{definition}
  Let $\mathcal{C}$ be a property of linear orders.

  A property $\pp$ of linear orders is \emph{$\mathcal{C}$-computable} if
  $\type{n}{\pp}$ and $\type{n}{\mathcal{C} \setminus \pp}$ are both
  computable as a function of $n$.
\end{definition}                                                                                                                                                                                                                                                                                                                                                                                                             


\begin{definition}
  Let $\beta \ge \omega$ be a limit ordinal.

  We define $\Gamma_{\beta} := \sum_{\mathbf{Finite}}^{\set{\gamma : \gamma \subseteq \beta^\ast + \beta}}$.
\end{definition}

\begin{example}
  \[
    \qqo = \mathbf{Finite} \uplus \set{\omega, \agemo, \oo}
  \]
\end{example}

\begin{observation}
  Let $\beta \ge \omega$ be a limit ordinal.

  Then $\Gamma_{\beta}$ is a good property of linear orders.
\end{observation}
\section{\texorpdfstring{$\omega$}{omega}-Hausdorff rank}

\begin{definitions}
  $\qqo = \set{M : \exists n \in \NN . M \subseteq \ZZ \cdot n}$.

  We define $\hrank{M} = \alpha$ iff $M \in \qqo^{= \alpha}$.

  $\mathcal{B}_{\alpha} := \bounded{\qqo^{< \alpha}}$ is the class
  of linear orders of rank $< \alpha$ on bounded intervals.
\end{definitions}

\begin{lemma}
  Let $\alpha > 0$ be an ordinal.

  Let $M$ be a countable linear order.

  Then $\hrank{M} \le \alpha$ iff $M$ is a finite sum of $\mathcal{B}_{\alpha}$-subintervals.
\end{lemma}

\begin{proof}
  From the previous lemma, it is clear that if $M$ is a finite sum of $\mathcal{B}_{\alpha}$-subintervals,
  then $\hrank{M} \le \alpha$, since the rank bound is preserved under finite sums.

  Conversely, suppose $\hrank{M} \le \alpha$.

  \emph{If} $M = \sum_{i \in I} M_i$ for some $M_i$ of Hausdorff rank $< \alpha$ for $I \in \Gamma_{\omega}$,
  take $x, y \in M$. Then let $x \in M_{i_1}$ and $y \in M_{i_2}$.

  Then $[x, y] \subseteq \sum_{i \in [i_1, i_2]} M_i$.

  But the distance between $i_1$ and $i_2$ is at most $1$, so $\sum_{i \in [i_1, i_2]} M_i$ is a finite sum of
  $\mathcal{B}_{\alpha}$-subintervals.

  So we have proven that each interval which is a $\Gamma_{\omega}$-sum of linear orders of
  $\omega$-rank $< \alpha$ is in $\mathcal{B}_{\alpha}$.

  But generally, $M$ is a finite sum of such $\ZZ$-sums, so it is a finite sums
  of $\mathcal{B}_{\alpha}$-subintervals.
\end{proof}

\begin{lemma}
  Let $\alpha \ge \omega$ be a limit ordinal.

  Let $M = \sum_{i \in I} M_i$ where $I \in \qqo$ and $\hrank{M_i} < \alpha$
  for all $i \in I$.

  Then,
  \[\hrank{M} = \alpha \iff \limsup_{i \in I} \parens{\hrank{M_i}+1} = \alpha\]
\end{lemma}

\begin{proof}
  Clearly $\hrank{M} \le \alpha$.

  If $\limsup_{i \in I}{\hrank{M_i}} = \alpha$,
  since $\hrank{M} \ge \hrank{M_i}$ for all $i \in I$,
  we get $\hrank{M} \ge \alpha$.

  If $\limsup_{i \in I}{\hrank{M_i}} = \beta < \alpha$,
  then $\hrank{M} \le \beta + 1 < \alpha$, since
  $\alpha$ is a limit ordinal.
\end{proof}

\begin{definitions}
  Let $\alpha > 0$ be an ordinal.

  Let $M$ be a linear order.

  We define:
  \begin{enumerate}
    \item $\mathcal{L}_{\alpha} := \set{M \in \mathcal{B}_{\alpha} : 1 + M \in \mathcal{B}_{\alpha}}$
    \item $\mathcal{R}_{\alpha} := \set{M \in \mathcal{B}_{\alpha} : M + 1 \in \mathcal{B}_{\alpha}}$
  \end{enumerate}

  And then:
  \begin{enumerate}
    \item $\mathcal{S}^{1}_{\alpha} := L_{\alpha} \cap R_{\alpha}$
    \item $\mathcal{S}^{\omega}_{\alpha} := L_{\alpha} \setminus R_{\alpha}$
    \item $\mathcal{S}^{\omega^\ast}_{\alpha} := R_{\alpha} \setminus L_{\alpha}$
    \item $\mathcal{S}^{\ZZ}_{\alpha} := B_{\alpha} \setminus (L_{\alpha} \cup R_{\alpha})$
  \end{enumerate}

  In particular, by the definition,
  \[
    \mathcal{B}_{\alpha}
    = \mathcal{S}^{1}_{\alpha}
    \uplus \mathcal{S}^{\omega}_{\alpha}
    \uplus \mathcal{S}^{\omega^\ast}_{\alpha}
    \uplus \mathcal{S}^{\ZZ}_{\alpha}
  \]

\end{definitions}

\begin{lemma}
  Let $\alpha \ge \omega$ be an ordinal.

  Let $M$ be a linear order.

  Suppose $M = \sum_{i \in \omega} M_i$
  where $\hrank{M_i} < \alpha$ for all $i \in \omega$,
  such that $\limsup_{i < \omega} \bs{\hrank{M_i} + 1} = \alpha$.

  Then $M + 1 \notin \mathcal{B}_{\alpha}$.
\end{lemma}

\begin{proof}
  On the contrary, suppose $M + 1 \in \mathcal{B}_{\alpha}$.

  Pick $x_i \in M_i$. Let $\infty$ be the last element of $M + 1$.

  Then $[x_i, \infty]$ is a bounded interval, and thus
  $\hrank{[x_i, \infty]} < \alpha$.

  The $\omega$-sequence $\set{\hrank{[x_i, \infty]}}$
  is decreasing and therefore it stabilizes at some $\beta < \alpha$.
  In particular, $\limsup_{i < \omega}{\hrank{[x_i, \infty]}} = \beta$.
\end{proof}


\begin{lemma}[Characterization of $\mathcal{B}_{\alpha}$]
  Let $\alpha \ge \omega$ be a limit ordinal.

  Let $M \in \mathcal{B}_{\alpha}$.

  Then $M$ \emph{has a unique} $\alpha$-shape in $\set{1, \omega, \omega^\ast, \ZZ}$.
\end{lemma}

\begin{proof}
  $\implies$: Let $M \in \mathcal{B}_{\alpha}$. By the lemma,
  there exists a decomposition $M = \sum_{i \in I} M_i$ where
  $I \subseteq \ZZ$ and $\hrank{M_i} < \alpha$ for all $i \in I$.

  If $I$ is finite, then $M$ is the $1$-sum of $\sum_{i \in I} M_i$,
  and thus it has rank $< \alpha$, so we are done.

  The existence is clear, but I am having trouble writing
  it formally. The idea is to just sum up $M_i$ in one/both directions,
  as long as it does not increase the rank.

  TBC.

  Uniqueness follows from the previous lemma.
\end{proof}

\begin{definition}
  Let $M \in \mathcal{B}_{\alpha}$. We define \emph{the} $\alpha$-shape
  of $M$ to be the $I \in \set{1, \omega, \omega^\ast, \ZZ}$
  for which the previous lemma holds.

  We define $\mathcal{S}^{s}_{\alpha}$ to be the class of linear orders
  whose $\alpha$-shape is $s$, for $s \in \set{1, \omega, \omega^\ast, \ZZ}$.
\end{definition}

\begin{lemma}
  Let $\alpha > 0$ be an ordinal.

  Let $s \in \set{1, \omega, \omega^\ast, \ZZ}$, and suppose that
  $\alpha = \sup_{i \in s} \bs{\alpha_i + 1}$.
  Then,

  \[
    \mathcal{S}^{s}_{\alpha} = \sum_{i \in s}{\Gamma_{\omega}^{< \alpha_i}}
  \]
\end{lemma}

\begin{corollary}
  Let $\alpha > 0$, $\delta \ge 0$ be ordinals.

  Let $s \in \set{1, \omega, \omega^\ast, \ZZ}$

  Then,
  
  \[
    \mathcal{S}^{s}_{\alpha + \delta}
    = \sum_{\mathcal{S}^{s}_{1 + \delta}}{\Gamma_{\omega}^{< \alpha}}
  \]
\end{corollary}

\begin{proof}
  Suppose that $\delta = \sup_{i \in s} \bs{\delta_i + 1}$.

  Then $\alpha + \delta = \sup_{i \in s} \bs{\alpha_i + 1 + \delta_i}$.

  \[
    \mathcal{S}^{s}_{\alpha + \delta}
    = \sum_{i \in s}{\mathcal{S}^{s}_{\alpha + \delta_i}}
    = \sum_{i \in s}\sum_{\Gamma_{\omega}^{< 1 + \delta_i}}{\Gamma_{\omega}^{< \alpha}}
    = \sum_{\sum_{i \in s}{\Gamma_{\omega}^{< 1 + \delta_i}}}{\Gamma_{\omega}^{< \alpha}}
    = \sum_{\mathcal{S}^{s}_{1 + \delta}}{\Gamma_{\omega}^{< \alpha}}
  \]
\end{proof}
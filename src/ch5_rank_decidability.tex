% % We continue this section.
% \section{Decidability of the Rank}

\begin{definition}
  Let $\qq$ be a class of linear orders.

  Let $M$ be a linear order, and let $J \subseteq M$ be a subset of $M$.

  We define the predicate $\Int{\qq} \ps{J}$ as
  true in $M$ iff $J$ is a $\qq$-subinterval of $M$.
\end{definition}

\begin{lemma}
  Let $\alpha > 0$ be an ordinal.

  Then predicates $\Int{\ho{\le \alpha}}$, $\Int{\ho{= \alpha}}$
  are expressible in $\mso[\Int{\ho{< \alpha}}]$.
\end{lemma}

\begin{proof}
  Obviously, \[
    \Int{\ho{= \alpha}}
    \iff \Int{\ho{\le \alpha}} \wedge \neg \Int{\ho{< \alpha}}
  \]

  So it is enough to express $\Int{\ho{\le \alpha}}$.

  Now, $J$ is a $\ho{\le \alpha}$-subinterval of $M$ iff
  $J$ is a subinterval of $M$ and $J \in \sum_{\Om}{\ho{< \alpha}}$.

  But this can be expressed in $\mso$ since it is expressible
  to check whether an arbitrary subset is in $\Om$.
\end{proof}

\begin{definition}
  Let $\alpha > 0$ be an ordinal.

  Let $M$ be a linear order and $x \in M$.

  We define the convex equivalence relation:
  \[\sim_\alpha := \sim_{\bounded{\ho{< \alpha}}}\]
  and $\bs{x}_{\alpha} := \bs{x}_{\bounded{\ho{< \alpha}}}$.

  That is,
  $\bs{x}_{\alpha}$ is the largest $\bounded{\ho{< \alpha}}$-subinterval
  containing $x$ in $M$.

  We define
  $\mathbf{L}_{\alpha} \ps{x} = \mathbf{1}_{\lb{\ho{< \alpha}}} \ps{\bs{x}_{\alpha}}$ and
  $\mathbf{R}_{\alpha} \ps{x} = \mathbf{1}_{\rb{\ho{< \alpha}}} \ps{\bs{x}_{\alpha}}$

  (where $\mathbf{1}_{A}$ is the indicator function of a set $A$).

  We define the \emph{the $\alpha$-shape}, $\sigma_{\alpha} \ps{x}$ as follows:
  \[
    \sigma_{\alpha} \ps{x} := \begin{cases}
      \ho{< \alpha} & \text{if } \mathbf{L}_{\alpha} \ps{x} = \mathbf{R}_{\alpha} \ps{x} = 0    \\
      \rmj{\alpha}  & \text{if } \mathbf{L}_{\alpha} \ps{x} = 0, \mathbf{R}_{\alpha} \ps{x} = 1 \\
      \lmj{\alpha}  & \text{if } \mathbf{L}_{\alpha} \ps{x} = 1, \mathbf{R}_{\alpha} \ps{x} = 0 \\
      \bmj{\alpha}  & \text{if } \mathbf{L}_{\alpha} \ps{x} = \mathbf{R}_{\alpha} \ps{x} = 1
    \end{cases}
  \]
\end{definition}

\begin{lemma}\label{alpha-expressible}
  Let $M$ be a linear order and $\alpha > 0$ an ordinal.

  Let $J \subseteq M$ be an interval.

  Then $J \in \ho{< \alpha}$ iff
  it is contained in a single $\sim_{\alpha}$-equivalence class $K$, such that:
  \begin{itemize}
    \item Either $K \in \lb{\ho{< \alpha}}$ or
          there exists some $x \in K$ such that $x < J$.
    \item Either $K \in \rb{\ho{< \alpha}}$ or
          there exists some $x \in K$ such that $x > J$.
  \end{itemize}
\end{lemma}

\begin{proof}
  Suppose $J \in \ho{< \alpha}$.
  Then obviously $J$ is contained in a single $\sim_{\alpha}$-equivalence class $K$.

  We will show the first condition, the second is symmetric.

  Suppose that for all $x \in K$, $J \le x$.
  Then we can write $K = J + J'$.
  Since $J \in \ho{< \alpha}$, it follows that $K \in \lb{\ho{< \alpha}}$.
\end{proof}

\begin{corollary}\label{int-expressible}
  Let $\alpha > 0$ be an ordinal.

  Let $P_{\alpha}$ be any predicate representing $\sim_{\alpha}$,
  let $L_{\alpha} = \mathbf{L}_{\alpha}$ and $R_{\alpha} = \mathbf{R}_{\alpha}$.

  Then $\Int{\ho{< \alpha}}$ is $\mso$-expressible over
  $\mso[P_{\alpha}, L_{\alpha}, R_{\alpha}]$.
\end{corollary}

% Sum of computable classes over a computable index
\begin{theorem}\label{computable-sum}
  Let $\pp$ be a computable class of linear orders of some finite signature,
  including $C_1, \ldots, C_k$.

  Let $\qqq$ be a \emph{finite} set of computable classes of linear orders
  over some finite signature which is disjoint from the signature of $\pp$.

  Let $F : 2^k \to \qqq$ be any function.

  Then $\bigcup_{I \in \pp} \sum_{i \in I} F(C_1(i), \ldots, C_k(i))$ is a computable class of linear orders.
\end{theorem}

\begin{proof}
  We will use the decomposition theorem.
  Let $\varphi$ be a formula of quantifier depth $n$. WLOG, $\varphi$ is a sentence.

  Then we can compute a formula $\psi(\xi)$ (where
  $\xi$ has the type of a coloring whose range is
  the set of $n$-types) such that
  for any linear order $M = \sum_{i \in I} M_i$,

  \[
    M \models \varphi \iff I \models \psi(\Xi)
  \]

  where $\Xi$ is the coloring assigning $i \in I$ the $n$-type of $M_i$.

  Thus, there is some $M \in \bigcup_{I \in \pp} \sum_{i \in I} \qq_i$,
  such that $M \models \varphi$
  iff there exists some $I \in \pp$, and assignment $\Xi$ of $n$-types,
  such that $\Xi(i)$ is satisfiable in $\qq_i$ for all $i \in I$, and $I \models \psi(\Xi)$.

  Equivalently, $\varphi$ is satisfiable over $\bigcup_{I \in \pp} \sum_{i \in I} \qq_i$
  iff
  \begin{align*}
    \exists \xi. \psi(\xi)
     & \wedge \xi \text{ is a coloring with }n\text{-types}             \\
     & \wedge \forall i. \xi(i) \in \type{n}{F(C_1(i), \ldots, C_k(i))}
  \end{align*}
  is satisfiable over $\pp$.

  Since $\qqq$ has only computable classes, We can pre-compute $\type{n}{F(\vec{c})}$
  for any value $\vec{c} \in 2^k$ so we can actually write the formula above in
  $\mso$. Furthermore, since $\pp$ is computable, we can check whether it is satisfiable
  over $\pp$. So we are done.
\end{proof}

\begin{definition}
  Let $\alpha$ be an ordinal.

  We define the $\mso[P_\alpha, L_\alpha, R_\alpha]$ formula $\good{\alpha}$ as follows:
  $\good{\alpha}$ is true in a linear order $I$ iff
  for every pair $i, i' \in M$ such that $i'$ is the successor of $i$, the following conditions hold:
  \begin{itemize}
    \item $P_\alpha(i) \ne P_\alpha(i')$
    \item $R_\alpha(i) = 0$ or $L_\alpha(i') = 0$
  \end{itemize}

  We further define the class
  \[
    \mathbf{Good}_{\alpha} := \set{I \in \cnt [P_\alpha, L_\alpha, R_\alpha] : I \models \good{\alpha}}
  \]

  as the class of all $\good{\alpha}$ linear orders.
\end{definition}

\begin{definition}
  Let $\alpha$ be an ordinal.

  We define the class of linear orders
  $\cnt \bs{\alpha}$ as the class of all linear orders
  labeled with $P_\alpha$, $L_\alpha$ and $R_\alpha$
  such that $P_\alpha$ represents the equivalence relation $\sim_{\alpha}$,
  $L_\alpha = \mathbf{L}_{\alpha}$, and $R_\alpha = \mathbf{R}_{\alpha}$.
\end{definition}

\begin{lemma}{\label{cnt-decomposition-single-ordinal}}
  Let $\alpha$ be an ordinal.

  Then,
  \[
    \cnt \bs{\alpha} = \bigcup_{I \in \mathbf{Good}_{\alpha}} \sum_{i \in I} \sigma_\alpha(i)
  \]
\end{lemma}

\begin{proof}
  ($\subseteq$) Let $M$ be a countable linear order labeled with $P_\alpha$, $L_\alpha$ and $R_\alpha$ as above.

  Let $I = M / \sim_{\alpha}$ be the quotient of $M$ by the equivalence relation $\sim_{\alpha}$.

  Then $M = \sum_{i \in I} M_i$,
  where $\set{M_i}_{i \in I}$ are the $\sim_{\alpha}$-equivalence class of $I$.

  Then for each $i \in I$, $M_i \in \bounded{\ho{< \alpha}}$,
  and by definition $\sigma_\alpha(i) = M_i$.

  Let $i'$ be the successor of $i$ in $I$.

  Then $P_\alpha(i) \ne P_\alpha(i')$ since $P_\alpha$ represents $\sim_{\alpha}$.

  Furthermore, suppose $R_\alpha(i) = L_\alpha(i') = 1$ holds.
  Then $M_i \in \rb{\ho{< \alpha}}$ and $M_{i'} \in \lb{\ho{< \alpha}}$.
  so $M_i$ and $M_{i'}$ are the same $\sim_{\alpha}$-equivalence class of $M$,
  which is a contradiction.

  Thus either $R_\alpha(i) = 0$ or $L_\alpha(i') = 0$.

  ($\supseteq$) Let $M = \sum_{i \in I} M_i$ be a linear order
  such that $I \in \mathbf{Good}_{\alpha}$ and $M_i \in \sigma_\alpha(i)$ for each $i \in I$.

  In particular $M_i \in \bounded{\ho{< \alpha}}$ for each $i \in I$, so it is contained
  in a single $\sim_{\alpha}$-equivalence class of $M$.

  Suppose that there exist distinct $j ,k \in I$ such that $j < k$, and
  $M_j, M_k$ are in the same $\sim_{\alpha}$-equivalence class.

  Let $x \in M_j$ and $y \in M_k$.
  Then $[x, y] \in \ho{< \alpha}$,
  and thus $[j, k] \in \ho{< \alpha}$,
  and in particular it is sparse.

  Then there exist some $j', k' \in I$ such that $j < j' < k' < k$,
  and $k'$ is the successor of $j'$ in $I$.

  Then $M_{j'}$ and $M_{k'}$ are in the same $\sim_{\alpha}$-equivalence class.
  Thus it must be the case that $M_{j'} \in \rb{\ho{< \alpha}}$ and $M_{k'} \in \lb{\ho{< \alpha}}$,
  which implies $R_\alpha(j') = L_\alpha(k') = 1$, which is a contradiction.

  Thus $\set{M_i}_{i \in I}$ are pairwise distinct $\sim_{\alpha}$-equivalence classes,
  and obviously the conditions holds,
  so $M \in C$ and we are done.
\end{proof}

\begin{lemma}\label{ho-decomposition-single-ordinal}
  Let $\alpha > 0$ be an ordinal
  and let $\delta \ge \omega$ be a limit ordinal.

  Then,
  \[
    \ho{< \alpha + \delta} \bs{\alpha} = \bigcup_{I \in \mathbf{Good}_{\alpha} \wedge \ho{< \delta}} \sum_{i \in I} \sigma_{\alpha}(i)
  \]

\end{lemma}

\begin{proof}
  ($\subseteq$) Let $M \in \ho{< \alpha + \delta} \bs{\alpha}$. By definition, $M$
  is a linear order labeled with $P_\alpha, L_\alpha, R_\alpha$ such that the
  underlying order is in $\ho{< \alpha + \delta}$, $P_\alpha$ represents
  $\sim_\alpha$, and $L_\alpha, R_\alpha$ are as defined.

  Let $I = M / \sim_\alpha$ be the quotient of $M$ by the equivalence relation
  $\sim_\alpha$. Then $M = \sum_{i \in I} M_i$, where $M_i$ are the
  $\sim_\alpha$-equivalence classes. By the definition of $\sim_\alpha$, each
  $M_i \in \bounded{\ho{< \alpha}}$, and by the definition of $\sigma_\alpha(i)$,
  $M_i \in \sigma_\alpha(i)$.

  Since $M \in \ho{< \alpha + \delta}$, the quotient $I$ is in $\ho{< \delta}$.
  For each pair $i, i'$ of consecutive elements in $I$, the labeling ensures that
  $P_\alpha(i) \ne P_\alpha(i')$ and either $R_\alpha(i) = 0$ or $L_\alpha(i') = 0$,
  so $I \in \mathbf{Good}_\alpha$. Thus, $M \in \sum_{i \in I} \sigma_\alpha(i)$ for
  some $I \in \mathbf{Good}_\alpha \wedge \ho{< \delta}$.

  ($\supseteq$) Let $M = \sum_{i \in I} M_i$ where $I \in \mathbf{Good}_\alpha
    \wedge \ho{< \delta}$ and $M_i \in \sigma_\alpha(i)$ for each $i \in I$. The
  labeling $P_\alpha, L_\alpha, R_\alpha$ on $M$ is as required by the definition
  of $\mathbf{Good}_\alpha$, and each $M_i \in \bounded{\ho{< \alpha}}$. Since
  $I \in \ho{< \delta}$, $M \in \ho{< \alpha + \delta}$. Thus,
  $M \in \ho{< \alpha + \delta} \bs{\alpha}$.

  Therefore,
  \[
    \ho{< \alpha + \delta} \bs{\alpha} =
    \bigcup_{I \in \mathbf{Good}_{\alpha} \wedge \ho{< \delta}}
    \sum_{i \in I} \sigma_{\alpha}(i)
  \]
\end{proof}

\begin{corollary}\label{rmj-decomposition-single-ordinal}
  Let $\alpha > 0$ be an ordinal
  and let $\delta \ge \omega$ be a limit ordinal.

  Then,
  \[
    \rmj{\alpha + \delta} \bs{\alpha} = \bigcup_{I \in \mathbf{Good}_{\alpha} \wedge \rmj{\delta}}
    \sum_{i \in I} \sigma_{\alpha}(i)
  \]
\end{corollary}

\begin{proof}
  It follows from~\cref{ho-decomposition-single-ordinal}
  together with~\cref{rmj-decomposition}.
\end{proof}

\begin{lemma}\label{cnt-decidable-single-ordinal}
  Let $\alpha > 0$ be an ordinal.

  Then $\cnt \bs{\alpha}$ is a computable class of linear orders.
\end{lemma}

\begin{proof}
  Since $\mathbf{Good}_{\alpha}$ is clearly computable,
  it follows from combining~\cref{computable-sum}
  with~\cref{cnt-decomposition-single-ordinal}
  and the computability of $\ho{< \alpha}$,
  $\rmj{\alpha}$, $\lmj{\alpha}$ and $\bmj{\alpha}$.
\end{proof}

\begin{theorem}\label{single-ordinal-satisfiability}
  Let $\alpha > 0$ be an ordinal.

  Satisfiability of $\mso[\Int{\ho{< \alpha}}]$
  over $\cnt$ is decidable.
\end{theorem}

\begin{proof}
  First, by~\cref{int-expressible}, we can convert
  any formula $\varphi$ in $\mso[\Int{\ho{< \alpha}}]$
  to a formula $\varphi'$ in $\mso \bs{P_\alpha, L_\alpha, R_\alpha}$
  such that $\varphi$ is satisfiable over $\cnt$
  iff $\varphi'$ is satisfiable over $\cnt \bs{\alpha}$.

  This is decidable by~\cref{cnt-decidable-single-ordinal}.
\end{proof}

We can extend these results to multiple ordinals.

\begin{lemma}\label{extend-lemma-single-ordinal}
  Let $\alpha$ be an ordinal.

  Let $I$ be a linear order
  and let $\set{M_i}_{i \in I}$ be a family of linear orders,
  such that for each pair $i, i' \in I$
  such that $i'$ is the successor of $i$ in $I$,
  either $\mathbf{R}_\alpha(M_{i}) = 0$ or $\mathbf{L}_\alpha(M_{i'}) = 0$.

  Then,

  \[
    \ps { \sum_{i \in I} M_i } \bs{\alpha} = \sum_{i \in I} \ps { M_i \bs{\alpha} }
  \]
\end{lemma}

\begin{proof}
  It is obvious, but TBC.
\end{proof}

\begin{notation}
  Let $\alpha_1 < \ldots < \alpha_k$ be ordinals.

  Let $\pp$ be a class of linear orders.

  Then,
  \[
    \pp \bs{\alpha_1, \ldots, \alpha_k} := \pp \bs{\alpha_1} \cdots \bs{\alpha_k}
  \]
\end{notation}

\begin{corollary}\label{extend-lemma-multiple-ordinals}
  Let $\alpha_1 < \ldots < \alpha_k < \alpha$ be ordinals.

  Let $I$ be a linear order
  and let $\set{M_i}_{i \in I}$ be a family of linear orders,
  such that for each pair $i, i' \in I$
  such that $i'$ is the successor of $i$ in $I$,
  either $\mathbf{R}_\alpha(M_{i}) = 0$ or $\mathbf{L}_\alpha(M_{i'}) = 0$.

  Then,
  \[
    \ps { \sum_{i \in I} M_i } \bs{\alpha_1, \ldots, \alpha_k, \alpha}
    = \sum_{i \in I} \ps { M_i \bs{\alpha_1, \ldots, \alpha_k, \alpha} }
  \]
\end{corollary}

\begin{proof}
  If $\mathbf{R}_\alpha(i) = 0$, then in particular
  $\mathbf{R}_{\alpha_j}(i) = 0$ for all $j \in \bs{k}$,
  and similarly for $\mathbf{L}_\alpha(i')$.

  So the condition for $\alpha$ implies
  the similar conditions for $\alpha_1, \ldots, \alpha_k$.

  Now, we can apply~\cref{extend-lemma-single-ordinal} inductively
  to obtain the result.
\end{proof}

\begin{lemma}\label{cnt-decomposition-multiple-ordinals}
  Let $\alpha_1 < \ldots < \alpha_k < \alpha$ be ordinals. Then,
  \[
    \cnt \bs{\alpha_1, \ldots, \alpha_k, \alpha}
    = \bigcup_{I \in \mathbf{Good}_{\alpha}} \sum_{i \in I} \sigma_{\alpha}(i) \bs{\alpha_1, \ldots, \alpha_k}
  \]
\end{lemma}

\begin{proof}
  This is a consequence of~\cref{cnt-decomposition-single-ordinal}
  and~\cref{extend-lemma-multiple-ordinals}.
\end{proof}

\begin{lemma}\label{ho-decomposition-multiple-ordinals}
  Let $\alpha_1 < \ldots < \alpha_k < \alpha$ and $\delta > 1$ be ordinals. Then,
  \[
    \ho{< \alpha + \delta} \bs{\alpha_1, \ldots, \alpha_k, \alpha}
    = \bigcup_{I \in \mathbf{Good}_{\alpha} \wedge \ho{< \delta}}
    \sum_{i \in I} \sigma_{\alpha}(i) \bs{\alpha_1, \ldots, \alpha_k}
  \]
\end{lemma}

\begin{proof}
  This is a consequence of~\cref{ho-decomposition-single-ordinal}
  and~\cref{extend-lemma-multiple-ordinals}.
\end{proof}

\begin{lemma}\label{rmj-decomposition-multiple-ordinals}
  Let $\alpha_1 < \ldots < \alpha_k < \alpha$ and $\delta$ be ordinals.
  Then,
  \[
    \rmj{\alpha + \delta} \bs{\alpha_1, \ldots, \alpha_k, \alpha}
    = \bigcup_{I \in \mathbf{Good}_{\alpha} \wedge \rmj{\delta}}
    \sum_{i \in I} \sigma_{\alpha}(i) \bs{\alpha_1, \ldots, \alpha_k}
  \]
\end{lemma}

\begin{proof}
  This is a consequence of~\cref{rmj-decomposition-single-ordinal}
  and~\cref{extend-lemma-multiple-ordinals}.
\end{proof}

\begin{lemma}\label{cnt-decidable-multiple-ordinals}
  Let $\alpha_1 < \ldots < \alpha_k < \alpha$ be ordinals.

  Then $\cnt \bs{\alpha_1, \ldots, \alpha_k, \alpha}$
  is a computable class of linear orders.
\end{lemma}

\begin{proof}
  Since $\mathbf{Good}_{\alpha}$ is computable,
  it follows from combining~\cref{computable-sum}
  with~\cref{cnt-decomposition-multiple-ordinals} and the computability
  of $\ho{< \alpha} \bs{\alpha_1, \ldots, \alpha_k}$,
  $\rmj{\alpha} \bs{\alpha_1, \ldots, \alpha_k}$,
  $\lmj{\alpha} \bs{\alpha_1, \ldots, \alpha_k}$ and
  $\bmj{\alpha} \bs{\alpha_1, \ldots, \alpha_k}$.
\end{proof}

\begin{theorem}\label{multiple-ordinals-satisfiability}
  Let $\alpha_1 < \ldots < \alpha_k$ be ordinals.

  Satisfiability of $\mso[\Int{\ho{< \alpha_1}}, \ldots, \Int{\ho{< \alpha_k}}]$
  over $\cnt$ is decidable.
\end{theorem}

\begin{proof}
  First, by~\cref{int-expressible}, we can convert
  any formula $\varphi$ in \[
    \mso[\Int{\ho{< \alpha_1}}, \ldots, \Int{\ho{< \alpha_k}}]
  \]
  to a formula $\varphi'$ in \[
    \mso \bs{P_{\alpha_1}, L_{\alpha_1}, R_{\alpha_1}, \ldots, P_{\alpha_k}, L_{\alpha_k}, R_{\alpha_k}}
  \]
  such that $\varphi$ is satisfiable over $\cnt$
  iff $\varphi'$ is satisfiable over $\cnt \bs{\alpha_1, \ldots, \alpha_k}$.

  This is decidable by~\cref{cnt-decidable-multiple-ordinals} and~\cref{cnt-decomposition-multiple-ordinals}.
\end{proof}
\section{Type Theory}
% Types of a property
\begin{definition}
    Let $\pp$ be a property of preorders.

    Let $n \in \NN$.

    We define $\type{n}{\pp}$ as the set of all
    $n$-types satisfiable in $\pp$.
\end{definition}

% Definition of a computable property
\begin{definition}
    A property $\pp$ of preorders is \emph{computable} if
    $n \mapsto \type{n}{\pp}$ is a computable function.
\end{definition}

\begin{lemma}\label{f-lemma}
    Let $\qq$ be a property of preorders.

    There exists a computable function $f_{\qq} = f: \NN \to \NN$ such that
    for every $n \in \NN$ and every ordinal $\alpha \ge f(n)$,
    $\type{n}{\qq^{< \alpha}} = \type{n}{\qq^{< f(n)}}$.
\end{lemma}

\begin{proof}
    Since there are only finitely many $n$-types,
    and the ordinal sequence \[\braces{\type{n}{\qq^{< \kappa}}}_{\kappa}\]
    is monotone,
    there must be some minimal $\kappa_0 \in \omega$ where the sequence stabilizes.

    This $\kappa_0$ is computable as a function of $n$, because
    $\type{n}{\qq^{< \kappa}}$ is computable for every finite $\kappa$.
\end{proof}

\begin{lemma}\label{ab-lemma}
    There exist global computable functions $a, b : \NN \to \NN$ such that
    for all $n, c_1, c_2 \in \NN$ such that $c_1, c_2 \ge a(n)$ and $c_1 \equiv c_2 \mod b(n)$,
    \[\type{n}{\qq^{= c_1}} = \type{n}{\qq^{= c_2}}\]
\end{lemma}

\begin{proof}
    Let $n \in \NN$.

    Since there are only finitely many sets of $n$-types,
    there exist (and can be computed)
    some $a(n) \ge f(n)$, $a(n) + b(n)$ such that

    \[
        \type{n}{\qq^{= a(n)}} = \type{n}{\qq^{= a(n) + b(n)}}
    \]

    By induction if follows that for all $c \ge a(n)$,
    \[\type{n}{\qq^{= c}} = \type{n}{\qq^{= c + b(n)}}\]
    since $\qq^{= c + 1} = \sum_{\qq} \qq^{= c}$.
\end{proof}

\begin{corollary}\label{s-alpha-finite}
    Let $n \in \NN$, and let $\alpha \ge \omega$ be an ordinal.

    Let $s \in \set{1, \omega, \agemo, \oo}$ be a shape.

    Then there exists a computable function $b(n)$ such that
    for all $c_1, c_2 \in \NN$ such that $c_1, c_2 \ge a(n)$ and $c_1 \equiv c_2 \mod b(n)$,
    we have
    \[\type{n}{\mathcal{S}^{s}_{c_1}} = \type{n}{\mathcal{S}^{s}_{c_2}}\]
\end{corollary}

\begin{proof}
    For $s = 1$, it follows from~\cref{f-lemma},
    since $\mathcal{S}^{1}_{c} = \qq^{< c}$
    and $c \ge a(n) \ge f(n)$ for $c \in \set{c_1, c_2}$.

    For $s \in \set{\omega, \agemo, \oo}$, it follows easily from~\cref{shape-structure}
    and~\cref{ab-lemma}.
\end{proof}

\begin{lemma}
    For every $n \in \NN$ and for every ordinal $\alpha \ge \omega$,
    \[
        \type{n}{\qq^{= \alpha}} = \type{n}{\bigcup_{c < b(n)}{\qq^{= a(n) + c}}}
    \]
    In particular, $\type{n}{\qq^{= \alpha}}$ can be computed,
    and is independent of the choice of $\alpha \ge \omega$.
\end{lemma}

\begin{proof}
    By induction on $\alpha \ge \omega$.

    Let $\set{\alpha_i}_{i \in \omega}$ be an $\omega$-sequence of ordinals
    such that $a(n) \le \alpha_i$ for all $i \in \omega$,
    and $\limsup_{i \in \omega} \ps{\alpha_i + 1} = \alpha$.

    Then $\qq^{= \alpha} = \sum_{\qq}{\bigcup_{i \in \omega} \qq^{= \alpha_i}}$ and thus,
    \begin{align*}
        \type{n}{\qq^{= \alpha}}
        &= \type{n}{\sum_{\qq} \bigcup_{i \in \omega} \qq^{= \alpha_i}} \\
        &= \type{n}{\sum_{\qq} \bigcup_{i \in \omega} \bigcup_{c < b(n)}{\qq^{= a(n) + c}}} \\
        &= \type{n}{\sum_{\qq} \bigcup_{c < b(n)}{\qq^{= a(n) + c}}} \\
        &= \type{n}{\bigcup_{c < b(n)}{\sum_{\qq} \qq^{= a(n) + c}}} \\
        &= \type{n}{\bigcup_{c < b(n)}{\qq^{= a(n) + c + 1}}} \\
        &= \type{n}{\bigcup_{c < b(n)}{\qq^{= a(n) + c}}}
    \end{align*}

    where the last transition is because $\type{n}{\qq^{= a(n)}} = \type{n}{\qq^{= a(n) + b(n)}}$.
\end{proof}

\begin{corollary}
    Let $n \in \NN$, and let $\alpha \ge \omega$ be an ordinal.

    Let $s \in \set{\omega, \agemo, \oo}$ be a shape.

    \[
        \type{n}{\mathcal{S}^{s}_{\alpha}}
        = \type{n}{\sum_{s} {\bigcup_{c < b(n)}{\qqo^{= a(n) + c}}}}
    \]

    In particular, $\type{n}{\mathcal{S}^{s}_{\alpha}}$ can be
    computed, and is independent of the choice of $\alpha \ge \omega$.
\end{corollary}

\begin{proof}
    There exists a sequence $\set{\alpha_i}_{i \in s}$ such that
    $a(n) \le \alpha_i$ for all $i \in s$,
    and $\limsup_{i \in s} \ps{\alpha_i + 1} = \alpha$.

    Then $\mathcal{S}^{s}_{\alpha} = \sum_{i \in s} \qqo^{= \alpha_i}$,
    and thus,
    \begin{align*}
        \type{n}{\mathcal{S}^{s}_{\alpha}}
        &= \type{n}{\sum_{i \in s} \qqo^{= \alpha_i}} \\
        &= \type{n}{\sum_{s} \bigcup_{c < b(n)}{\qqo^{= a(n) + c}}} \\
        &= \type{n}{\bigcup_{c < b(n)}{\sum_{s} \qqo^{= a(n) + c}}} \\
        &= \type{n}{\bigcup_{c < b(n)}{\mathcal{S}^{s}_{a(n) + c + 1}}} \\
        &= \type{n}{\bigcup_{c < b(n)}{\mathcal{S}^{s}_{a(n) + c}}} \\
    \end{align*}

    where the last transition is by~\cref{s-alpha-finite}.

\end{proof}



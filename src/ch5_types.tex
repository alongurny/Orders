\section{Type Theory}
% Types of a property
\begin{definition}
    Let $\pp$ be a property of preorders.

    Let $n \in \NN$.

    We define $\type{n}{\pp}$ as the set of all
    $n$-types satisfiable in $\pp$.
\end{definition}

% Definition of a computable property
\begin{definition}
    A property $\pp$ of preorders is \emph{computable} if
    $n \mapsto \type{n}{\pp}$ is a computable function.
\end{definition}

\begin{lemma}\label{f-lemma}
    Let $\qq$ be a property of preorders,
    labeled with finitely many colors.

    There exists a computable function $f_{\qq} = f: \NN \to \NN$ such that
    for all $n \in \NN$, and all $a \in \NN$ such that $a \ge f(n)$,
    $\type{n}{\qq^{a}} = \type{n}{\qq^{f(n)}}$.
\end{lemma}

\begin{proof}
    Since there are only finitely many $n$-types,
    and the $\omega$-sequence \[\braces{\type{n}{\qq^{k}}}_{k \in \omega}\]
    is monotone,
    there must be some $k$ where the sequence stabilizes.

    This point $k$ is computable as a function of $n$, because
    $\type{n}{\qq^{k}}$ is computable for every finite $k$.
\end{proof}

\begin{lemma}\label{ab-lemma}
    There exist global computable functions $a, b : \NN \to \NN$ such that
    for all $n, c_1, c_2 \in \NN$ such that $c_1, c_2 \ge a(n)$ and $c_1 \equiv c_2 \mod b(n)$,
    \[\type{n}{\qq^{= c_1}} = \type{n}{\qq^{= c_2}}\]
\end{lemma}

\begin{proof}
    Let $n \in \NN$.

    Since there are only finitely many sets of $n$-types,
    there exist (and can be computed)
    some $a(n) \ge f(n)$, $a(n) + b(n)$ such that

    \[
        \type{n}{\qq^{= a(n)}} = \type{n}{\qq^{= a(n) + b(n)}}
    \]

    We shall prove by induction that for all $c \ge a(n)$,

    \[\type{n}{\qq^{= c}} = \type{n}{\qq^{= c + b(n)}}\]

    This will complete the proof.

    The base case $c = a(n)$ is by the definition of $a(n)$ and $b(n)$.

    Suppose the induction hypothesis holds for $c$.

    Let $M \in \qq^{= c + 1}$.

    Write $M = \sum_{i \in I} M_i$ where $M_i \in \qq^{< c + 1}$,
    and $M_i \in \qq^{= c}$ infinitely many times.

    By the induction hypothesis,
    if $M_i \in \qq^{= c}$, we can find $N_i \equiv_n M_i$ with $N_i \in \qq^{= c + b(n)}$.
    Setting $N_i := M_i$ for all other $i$, we conclude that $N := \sum_{i \in I} N_i$
    is $n$-equivalent to $M$.

    However, clearly $N \in \qq^{= c + b(n) + 1}$. So overall,
    \[\type{n}{\qq^{= c + 1}} \subseteq \type{n}{\qq^{= c + b(n) + 1}}\]

    Conversely, suppose $M \in \qq^{= c + b(n) + 1}$.
    Write $M$ = $\sum_{i \in I} M_i$ where $M_i \in \qq^{< c + b(n) + 1}$,
    and $M_i \in \qq^{= c + b(n)}$ infinitely many times.

    By the induction hypothesis,
    we can find for all $i$ such that $M_i \in \qq^{= c + b(n)}$ some
    $N_i \equiv_n M_i$ with $N_i \in \qq^{= c}$.
    Furthermore, since $c \ge a(n) \ge f(n)$, we can
    find $N_i \equiv_n M_i$ with $N_i \in \qq^{f(n)} \subseteq \qq^{c}$ for all other $i$.

    We conclude that $N := \sum_{i \in I} N_i$ is $n$-equivalent to $M$.
    However, clearly $N \in \qq^{= c + 1}$. So overall,
    \[\type{n}{\qq^{= c + b(n) + 1}} \subseteq \type{n}{\qq^{= c + 1}}\]

    So we have proven the induction step, and the lemma follows.
\end{proof}

\begin{corollary}
    Let $n \in \NN$, and let $\alpha \ge \omega$ be an ordinal.

    Let $s \in \set{1, \omega, \agemo, \oo}$ be a shape.

    Then there exists a computable function $b(n)$ such that
    for all $c_1, c_2 \in \NN$ such that $c_1, c_2 \ge a(n)$ and $c_1 \equiv c_2 \mod b(n)$,
    we have
    \[\type{n}{\mathcal{S}^{s}_{c_1}} = \type{n}{\mathcal{S}^{s}_{c_2}}\]
\end{corollary}

\begin{proof}
    For $s = 1$, it follows from~\cref{f-lemma},
    since $\mathcal{S}^{1}_{c} = \qq^{c}$
    and $c \ge a(n) \ge f(n)$ for $c \in \set{c_1, c_2}$.

    For $s \in \set{\omega, \agemo, \oo}$, it follows easily from~\cref{b_alpha-structure-lemma}
    and~\cref{ab-lemma}.
\end{proof}

\begin{lemma}
    Let $n \in \NN$, and let $\alpha \ge \omega$ be an ordinal.
    \[
        \type{n}{\mathcal{S}^{1}_{\alpha}}
        = \type{n}{\bigcup_{c < a(n) + b(n)} {\Omega^{= c}}}
    \]

    In particular, $\type{n}{\mathcal{S}^{1}_{\alpha}}$ can be
    computed, and is independent of the choice of $\alpha \ge \omega$.
\end{lemma}

\begin{proof}
    TBC.
\end{proof}

\begin{lemma}
    Let $n \in \NN$, and let $\alpha \ge \omega$ be an ordinal.

    Let $s \in \set{\omega, \agemo, \oo}$ be a shape.

    \[
        \type{n}{\mathcal{S}^{s}_{\alpha}}
        = \type{n}{\sum_{s} {\bigcup_{c < b(n)}{\qqo^{= a(n) + c}}}}
    \]

    In particular, $\type{n}{\mathcal{S}^{s}_{\alpha}}$ can be
    computed, and is independent of the choice of $\alpha \ge \omega$.
\end{lemma}

\begin{proof}
    TBC.
\end{proof}


\begin{lemma}
    Let $n \in \NN$, and let $\alpha \ge \omega$ be an ordinal.

    Let $\delta > 0$ be an ordinal.

    \[
        \type{n}{\mathcal{S}^{1}_{\alpha + \delta}}
        = \type{n}{\bigcup_{c < a(n) + b(n)} {\Omega^{= c}}}
    \]

    In particular, $\type{n}{\mathcal{S}^{1}_{\alpha}}$ can be
    computed, and is independent of the choice of $\alpha \ge \omega$.
\end{lemma}

\begin{proof}
    TBC.
\end{proof}

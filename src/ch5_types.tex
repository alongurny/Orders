\section{Type Theory}

Type theory provides a framework for analyzing the expressive power of logical languages over classes of structures. In this chapter, we introduce the notion of types for classes of preorders and study their computability. The results here connect the structural classes of orders with logical definability, setting the stage for the study of decidability in the next chapter.

% Types of a class
\begin{definition}
    Let $\pp$ be a class of preorders.

    Let $n \in \NN$.

    We define $\type{n}{\pp}$ as the set of all
    $n$-types satisfiable in $\pp$.
\end{definition}

% Definition of a computable class
\begin{definition}
    a class $\pp$ of preorders is \emph{computable} if
    $n \mapsto \type{n}{\pp}$ is a computable function.

    Equivalently, satisfiability of $\mso$ over $\pp$ is decidable.
\end{definition}

\begin{lemma}\label{f-lemma}
    Let $\qq$ be a class of preorders.

    There exists a computable function $f_{\qq} = f: \NN \to \NN$ such that
    for every $n \in \NN$ and every ordinal $\alpha \ge f(n)$,
    $\type{n}{\hh{\qq}{< \alpha}} = \type{n}{\hh{\qq}{< f(n)}}$.
\end{lemma}

\begin{proof}
    Since there are only finitely many $n$-types,
    and the ordinal sequence \[\braces{\type{n}{\hh{\qq}{< \kappa}}}_{\kappa}\]
    is monotone,
    there must be some minimal $\kappa_0 \in \om$ where the sequence stabilizes.

    This $\kappa_0$ is computable as a function of $n$ by successive iteration.
\end{proof}

\begin{lemma}\label{ab-lemma}
    There exist global computable functions $a, b : \NN \to \NN$ such that
    for all $n, c_1, c_2 \in \NN$ such that $c_1, c_2 \ge a(n)$ and $c_1 \equiv c_2 \mod b(n)$,
    \[\type{n}{\hh{\qq}{= c_1}} = \type{n}{\hh{\qq}{= c_2}}\]
\end{lemma}

\begin{proof}
    Let $n \in \NN$.

    Since there are only finitely many sets of $n$-types,
    there exist (and can be computed)
    some $a(n) \ge f(n)$, $a(n) + b(n)$ such that

    \[
        \type{n}{\hh{\qq}{= a(n)}} = \type{n}{\hh{\qq}{= a(n) + b(n)}}
    \]

    By induction if follows that for all $c \ge a(n)$,
    \[
        \type{n}{\hh{\qq}{= c}} = \type{n}{\hh{\qq}{= c + b(n)}}
    \]
    since $\hh{\qq}{= c + 1} = \sum_{\qq} \hh{\qq}{= c}$.
\end{proof}

\begin{corollary}\label{right-major-computable}
    Let $n \in \NN$, and let $\alpha \ge \om$ be an ordinal.

    Then for all $c_1, c_2 \in \NN$ such that $c_1, c_2 \ge a(n)$ and $c_1 \equiv c_2 \mod b(n)$,
    we have
    \[\type{n}{\rmj{c_1}} = \type{n}{\rmj{c_2}}\]
\end{corollary}

\begin{proof}
    By~\cref{right-major-successor-decomposition}, $\rmj{c} = \sum_{\om}{\ho{= c}}$.

    Then it follows immediately from~\cref{ab-lemma}.
\end{proof}

\begin{lemma}
    For every $n \in \NN$ and for every pair of ordinals $\alpha \ge \om$,
    $\beta > \alpha$,
    \[
        \type{n}{\hh{\qq}{[\alpha, \beta)}} = \type{n}{\bigcup_{c < b(n)}{\hh{\qq}{= a(n) + c}}}
    \]
    In particular, $\type{n}{\hh{\qq}{= \alpha}}$ can be computed,
    and is independent of the choice of $\alpha \ge \om$.
\end{lemma}

\begin{proof}
    It is enough to prove that
    \[
        \type{n}{\hh{\qq}{= \alpha}} = \type{n}{\bigcup_{c < b(n)}{\hh{\qq}{= a(n) + c}}}
    \]

    We thus proceed by induction on $\alpha \ge \om$.

    Let $\set{\alpha_i}_{i \in \om}$ be an increasing $\om$-sequence of ordinals
    such that $a(n) \le \alpha_i$ for all $i \in \om$,
    and $\sup_{i \in \om} \ps{\alpha_i + 1} = \alpha$.

    Then $\hh{\qq}{= \alpha} = \sum_{\qq}{\bigcup_{i \in \om} \hh{\qq}{[\alpha_i, \alpha)}}$ and thus,
    \begin{align*}
        \type{n}{\hh{\qq}{= \alpha}}
         & = \type{n}{\sum_{\qq} \bigcup_{i \in \om} \hh{\qq}{[\alpha_i, \alpha)}}             \\
         & = \type{n}{\sum_{\qq} \bigcup_{i \in \om} \bigcup_{c < b(n)}{\hh{\qq}{= a(n) + c}}} \\
         & = \type{n}{\sum_{\qq} \bigcup_{c < b(n)}{\hh{\qq}{= a(n) + c}}}                     \\
         & = \type{n}{\bigcup_{c < b(n)}{\sum_{\qq} \hh{\qq}{= a(n) + c}}}                     \\
         & = \type{n}{\bigcup_{c < b(n)}{\hh{\qq}{= a(n) + c + 1}}}                            \\
         & = \type{n}{\bigcup_{c < b(n)}{\hh{\qq}{= a(n) + c}}}
    \end{align*}

    where the last transition is because $\type{n}{\hh{\qq}{= a(n)}} = \type{n}{\hh{\qq}{= a(n) + b(n)}}$.
\end{proof}

\begin{corollary}
    Let $n \in \NN$, and let $\alpha \ge \om$ be an ordinal.

    \[
        \type{n}{\rmj{\alpha}}
        = \type{n}{\sum_{\om} {\bigcup_{c < b(n)}{\ho{= a(n) + c}}}}
    \]

    In particular, $\type{n}{\rmj{\alpha}}$ can be
    computed, and is independent of the choice of $\alpha \ge \om$.
\end{corollary}

\begin{proof}
    There exists an increasing $\om$-sequence $\set{\alpha_i}_{i \in \om}$ such that
    $a(n) \le \alpha_i$ for all $i \in \om$,
    and $\sup_{i \in \om} \ps{\alpha_i + 1} = \alpha$.

    Then $\rmj{\alpha} = \sum_{i \in \om} \ho{= \alpha_i}$,
    and thus,
    \begin{align*}
        \type{n}{\rmj{\alpha}}
         & = \type{n}{\sum_{i \in \om} \ho{= \alpha_i}}               \\
         & = \type{n}{\sum_{\om} \bigcup_{c < b(n)}{\ho{= a(n) + c}}} \\
         & = \type{n}{\bigcup_{c < b(n)}{\sum_{\om} \ho{= a(n) + c}}} \\
         & = \type{n}{\bigcup_{c < b(n)}{\rmj{a(n) + c + 1}}}         \\
         & = \type{n}{\bigcup_{c < b(n)}{\rmj{a(n) + c}}}             \\
    \end{align*}

    where the last transition is by~\cref{right-major-computable}.
\end{proof}

\section{Preorders}

\begin{definition}[Preorder]
  A relation $\le$ is a \emph{preorder} if it is reflexive and transitive.
\end{definition}

% Definition of a property
\begin{definition}[Property of preorders]
  A \emph{property} $\pp$ of preorders is a class of preorders which
  is closed under isomorphism.
\end{definition}

% Definition of a monotone property
\begin{definition}
  A property $\pp$ of preorders is \emph{monotone} if for every preorder $M$,
  $M \in \pp$ implies that every suborder of $M$ is in $\pp$.
\end{definition}

% Definition of an additive property
\begin{definition}
  A property $\pp$ of preorders is an \emph{additive property} if for every preorders $M_1$ and $M_2$,
  $M_1 + M_2 \in \pp$ iff $M_1, M_2 \in \pp$, i.e., if $\pp + \pp = \pp$.
\end{definition}

% Definition of a good property
\begin{definition}
  A property $\pp$ of preorders is a \emph{good property} if it is monotone and additive,
  and $\omega^\ast + \omega \in \pp$.
\end{definition}

% Definitions of special preorders.
\begin{definition}
  Let $M$ be a preorder.

  Then $M^\ast$ is the dual - the reverse preorder of $M$.
\end{definition}

\begin{definition}[Sum of preorders]
  Let $I$ be a preorder, and let $\set{M_i}_{i \in I}$ be a family of preorders.

  The sum $M = \sum_{i \in I} M_i$ is defined as follows:

  The domain is $M = \biguplus_{i \in I} M_i$ (a disjoint union).

  Let $\le_i$ be the preorder on $M_i$.

  The order is defined as follows:
  \[
    x \le y \iff \begin{cases}
      \exists i \in I. x, y \in M_i \land x \le_i y \\
      \exists i, j \in I. x \in M_i \land y \in M_j \land i < j
    \end{cases}
  \]

  If $I = 2$, we define $M_1 + M_2 := \sum_{i \in 2} M_i$.
\end{definition}

\begin{lemma}
  Let $I$ be a preorder, and let $\set{M_i}_{i \in I}$ be a family of preorders.

  Then $M = \sum_{i \in I} M_i$ is a preorder.
\end{lemma}

\begin{proof}
  Reflexivity is clear.

  For transitivity, suppose $x \le y$ and $y \le z$.

  Suppose $x \in M_i$, $y \in M_j$, $z \in M_k$.

  Then $i \le j$ and $j \le k$, so $i \le k$.
  If $i = k$, then necessarily $i = j = k$, and so $x \le_i y$ and $y \le_i z$,
  so $x \le_i z$, so $x \le z$, as required.

  Otherwise, $i < k$, and thus $x \le z$, as required.
\end{proof}

\begin{definition}
  Let $\pp_1$ and $\pp_2$ be properties of preorders.

  Then we define
  \[
    \pp_1 + \pp_2 := \set{M_1 + M_2 : M_1 \in \pp_1 \wedge M_2 \in \pp_2}
  \]
\end{definition}

\begin{definition}
  Let $\pp$ and $\qq$ be a property of preorders.

  Then we define
  \[
    \sum_{\pp}{\qq} := \set{\sum_{i \in I}{M_i} : I \in \pp \wedge \forall i \in I. M_i \in \qq}
  \]

  Furthermore, if $\pp = \set{I}$ is a singleton, we define $\sum_{I}{\qq} := \sum_{\pp}{\qq}$.
\end{definition}

\begin{definition}[Kleene plus]
  Let $\pp$ be a property of preorders.

  We define its Kleene plus as the smallest property of preorders $\pp^+$ which contains $\pp$ and is closed under
  finite sums.

  That is, $1^+ = \set{1, 2, \ldots}$, and $\pp^+ = \sum_{1^+}{\pp}$.
\end{definition}

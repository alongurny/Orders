\section{Preorders}

Preorders form the foundational structure upon which more complex order-theoretic concepts are built. In this chapter, we
introduce the basic definitions and properties of preorders, setting the stage for the study of linear orders, ranks, and types
in subsequent chapters. We motivate the study of preorders by highlighting their role as a generalization of partial and linear
orders, and by showing how properties and operations on preorders can be systematically developed.

We begin by studying the properties of preorders. Basically, we define a \emph{property} as a class which is close under
isomorphism. We then define the sum operation on preorders. This will be used to create new properties from old ones.

% Definition of a preorder
\begin{definitions} [Preorder]
  A \emph{(labeled) preorder} is a a set $M$
  together with a binary relation $\le$ on $M$ such that
  $\le$ is reflexive and transitive,
  possibly endowed with monadic predicates (labels)
  over some first-order monadic signature.
\end{definitions}

% Discussion
The notion of a property of preorders allows us to classify preorders according to structural features that are preserved under
isomorphism. This abstraction is crucial for developing general results that apply to broad classes of structures.

% Definition of a property
\begin{definition}[Property of preorders]
  A \emph{property} $\pp$ of preorders is a class of preorders which
  is closed under isomorphism.
\end{definition}

% Discussion
Monotonicity is a natural strengthening of the notion of a property, ensuring that substructures inherit the property. This is
particularly useful when analyzing how properties behave under restrictions to suborders.

% Definition of a monotone property
\begin{definition}
  A property $\pp$ of preorders is \emph{monotone} if for every preorder $M$,
  $M \in \pp$ implies that every suborder of $M$ is in $\pp$.
\end{definition}

% Definitions of special preorders.
\begin{definition}
  Let $M$ be a preorder.

  Then $M^\ast$ is the dual/reverse preorder of $M$.
\end{definition}

% Discussion
The sum operation on preorders is a key construction that enables us to build larger preorders from smaller ones. The following
definitions and lemmas formalize this operation and explore its basic properties.

\begin{definition}[Sum of preorders]
  Let $I$ be a preorder, and let $\set{M_i}_{i \in I}$ be a family of preorders over
  a disjoint signature (i.e., for every $i \in I$, $I$ and $M_i$ have disjoint sets of labels).

  The sum $M = \sum_{i \in I} M_i$ is defined as follows:

  The domain is $M = \biguplus_{i \in I} M_i$ (a disjoint union).

  Let $\le_i$ be the preorder on $M_i$.

  Let $x \in M_i$ and $y \in M_j$.

  Then we define $x \le y$ iff either
  $i = j$ and $x \le_i y$ or $i < j$.

  The labels are inherited from either $I$ or the $M_i$'s.

  If $I = 2$, we define $M_1 + M_2 := \sum_{i \in 2} M_i$.
\end{definition}

\begin{lemma}
  Let $I$ be a preorder, and let $\set{M_i}_{i \in I}$ be a family of preorders,
  over a disjoint signature.

  Then $M = \sum_{i \in I} M_i$ is a preorder.
\end{lemma}

\begin{proof}
  Reflexivity is clear.

  For transitivity, suppose $x \le y$ and $y \le z$.

  Suppose $x \in M_i$, $y \in M_j$, $z \in M_k$.

  Then $i \le j$ and $j \le k$, so $i \le k$.
  If $i = k$, then necessarily $i = j = k$, and so $x \le_i y$ and $y \le_i z$,
  so $x \le_i z$, so $x \le z$, as required.

  Otherwise, $i < k$, and thus $x \le z$, as required.
\end{proof}

% Discussion
The next lemma confirms that the sum of preorders, as defined, indeed yields a preorder. This is essential for ensuring that our
constructions remain within the intended class of structures.

% Definition of sum of two properties
\begin{definition}
  Let $\pp_1$ and $\pp_2$ be properties of preorders.

  Then we define
  \[
    \pp_1 + \pp_2 := \set{M_1 + M_2 : M_1 \in \pp_1 \wedge M_2 \in \pp_2}
  \]

  The labels are inherited from either $\pp_1$ or $\pp_2$.
\end{definition}

% Discussion
We now extend the sum operation to properties of preorders, allowing us to combine classes of structures in a systematic way.
This leads to the notion of additive properties, which are stable under sums.

% Definition of an additive property
\begin{definition}
  A property $\pp$ of preorders is an \emph{additive property} if for every preorders $M_1$ and $M_2$,
  $M_1 + M_2 \in \pp$ iff $M_1, M_2 \in \pp$.
\end{definition}

% Discussion
The Kleene plus operation captures the idea of closure under finite sums, which is a recurring theme in the study of algebraic
structures. It provides a convenient way to generate new properties from existing ones.

\begin{definition}[Kleene plus]
  Let $\pp$ be a property of preorders.

  We define its Kleene plus as the smallest property of preorders $\pp^+$ which contains $\pp$ and is closed under
  finite sums.

  That is, $1^+ = \set{1, 2, \ldots}$, and $\pp^+ = \sum_{1^+}{\pp}$.
\end{definition}

\begin{definition}[Sum of a property over a preorder]
  Let $I$ be a preorder.

  Let $\qq$ be a property of preorders.

  Then we define
  \[
    \sum_{I}{\qq} := \set{\sum_{i \in I}{M_i} : \forall i \in I. M_i \in \qq}
  \]
\end{definition}

\begin{definition}[Sum of a family of properties over a preorder]
  Let $I$ be a preorder.

  Let $\set{\qq_i}_{i \in I}$ be a family of properties of preorders
  over a disjoint signature.

  Then we define
  \[
    \sum_{i \in I}{\qq_i} := \set{\sum_{i \in I}{M_i} : \forall i \in I. M_i \in \qq_i}
  \]

  The labels are inherited from either $I$ or the $\qq_i$'s.
\end{definition}

\begin{note}
  Let $I$ be a preorder, and let $\qq$ be a property of preorders.

  By the previous two definitions,
  \[
    \sum_{I}{\qq} = \sum_{i \in I}{\qq}
  \]
\end{note}

\begin{definition}[Sum of a property over a property]
  Let $\pp$ be a property of preorders.

  Let $\qq$ be a property of preorders over a disjoint signature.

  Then we define,
  \[
    \sum_{\pp}{\qq} := \set{\sum_{I}{\qq} : I \in \pp}
  \]
\end{definition}

% Discussion
Finally, we generalize the sum operation to families of properties and to sums indexed by properties themselves. These
generalizations will be useful in later chapters when we analyze more complex constructions.

\section{Preorders}

\begin{definition}
  A relation $\le$ is a \emph{preorder} if it is reflexive and transitive.
\end{definition}

\begin{definition}
  Let $I$ be a preorder, and let $\set{M_i}_{i \in I}$ be a family of preorders.

  The sum $M = \sum_{i \in I} M_i$ is defined as follows:

  The domain is $M = \biguplus_{i \in I} M_i$ (a disjoint union).

  Let $\le_i$ be the preorder on $M_i$.

  The order is defined as follows:
  \[
    x \le y \iff \begin{cases}
      \exists i \in I. x, y \in M_i \land x \le_i y \\
      \exists i, j \in I. x \in M_i \land y \in M_j \land i < j
    \end{cases}
  \]

  If $I = 2$, we define $M_1 + M_2 := \sum_{i \in 2} M_i$.
\end{definition}

\begin{lemma}
  Let $I$ be a preorder, and let $\set{M_i}_{i \in I}$ be a family of preorders.

  Then $M = \sum_{i \in I} M_i$ is a preorder.
\end{lemma}

\begin{proof}
  Reflexivity is clear.

  For transitivity, suppose $x \le y$ and $y \le z$.

  Suppose $x \in M_i$, $y \in M_j$, $z \in M_k$.

  Then $i \le j$ and $j \le k$, so $i \le k$.
  If $i = k$, then necessarily $i = j = k$, and so $x \le_i y$ and $y \le_i z$,
  so $x \le_i z$, so $x \le z$, as required.

  Otherwise, $i < k$, and thus $x \le z$, as required.
\end{proof}

\begin{definition}
  Let $\pp$ and $\qq$ be a property of preorders.

  Then we define
  \[
    \sum_{\pp}^{\qq} := \set{\sum_{i \in I}{M_i} : I \in \pp \wedge \forall i \in I. M_i \in \qq}
  \]

  Furthermore, if $\pp = \set{I}$ is a singleton, we define $\sum_{I}^{\qq} := \sum_{\pp}^{\qq}$.
\end{definition}

\begin{theorem}[Decomposition theorem]
  There exists a computable translation $\mathcal{T}$
  from $MSO$ formulae $\varphi \ps{\vec{X}}$,

  such that for any $M = \sum_{i \in I} M_i$, and formula $\varphi$,
  if $n$ is the quantifier-depth of $\varphi$,
  then
  \[
    M, \vec{X} \models \varphi \iff I, \Pi \models \mathcal{T} {\varphi}
  \]

  where $\Pi \ps{i} = \type{n}{M_i}$.
\end{theorem}

\begin{corollary}
  Let $\pp$ be a property of linear orders
  labeled with colors $\vec{C} = \set{C_k}_{k=1}^m$.

  Let $\set {\qq_k}_{k=1}^m$ be a family of (possibly labeled) properties.

  Let $n \in \NN$.

  We can compute the type $\type{n}{\sumv{\pp}{\vec{C}}{\vec{\qq}}}$
  from the types $\type{n}{\pp}$ and $\type{n}{\qq_k}$ for all $k$.
\end{corollary}

\begin{proof}
  Let $\tau$ be an $n$-type of the appropriate signature.
  
  Assume we have $\type{n}{\pp}$ and $\type{n}{\qq_k}$ for all $k$.
\end{proof}
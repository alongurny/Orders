\documentclass{article}

\usepackage{amsmath}
\usepackage{amssymb}
\usepackage{amsthm}

\usepackage{todonotes}

% Theorems
\newtheorem{theorem}{Theorem}
\newtheorem{corollary}{Corollary}
\newtheorem{observation}{Observation}
\newtheorem{lemma}{Lemma}
\newtheorem{proposition}{Proposition}
\newtheorem{conjecture}{Conjecture}
\newtheorem{claim}{Claim}

% Definitions
\newtheorem{definition}{Definition}
\newtheorem{definitions}{Definitions}
\newtheorem{notation}{Notation}
\newtheorem{notations}{Notations}

% Delimiters
\newcommand{\parens}[1]{\left( {#1} \right)}
\newcommand{\brackets}[1]{\left[ {#1} \right]}
\newcommand{\braces}[1]{\left\{ {#1} \right\}}

% Macros
\newcommand{\setcomp}[1]{\braces{#1}}
\newcommand{\hrank}[1]{\mathbf{hrank}\left( #1 \right)}
\newcommand{\bigforall}{\mbox{\Large $\mathsurround0pt\forall$}} 

% Letters
\newcommand{\agemo}{-\omega}
\newcommand{\otp}{\mathbf{otp}}
\newcommand{\fso}{\mathbf{FSO}}
\newcommand{\mso}{\mathbf{MSO}}
\newcommand{\bounded}{\mathbf{bounded}}
\newcommand{\pp}{\mathbf{P}}
\newcommand{\qq}{\mathbf{Q}}

% Sets
\newcommand{\NN}{\mathbb{N}}
\newcommand{\ZZ}{\mathbb{Z}}

\author{Alon Gurny}
\title{Orders}
\date{\today}

\begin{document}

\maketitle

% Definition of the Hausdorff rank
\begin{definition}
  Let $L$ be a linear order.

  We define $\hrank{L} \le 0$ iff $L$ is finite.

  Let $\alpha > 0$ be an ordinal.

  We define $\hrank{L} \le \alpha$
  iff $L = \sum_{i \in I} L_i$ for some linear order $I$,
  where $\hrank{L_i} < \alpha$ and $I$ is a finite sum of $1$, $\omega$ and
  $\agemo$.

  We write $\hrank{L} = \alpha$ iff $\alpha$ is the
  least ordinal such that $\hrank{L} \le \alpha$.

  We write $\hrank{L} = \bot$ iff there is no ordinal $\alpha$ such that
  $\hrank{L} \le \alpha$.
\end{definition}

We will be working with scattered linear orders.

\begin{claim}
  Let $L$ be a countable linear order.

  Then $\hrank{L}$ is defined iff $L$ is scattered.
\end{claim}

\begin{proof}
  To prove $\implies$ is easy, as a scattered sum of scattered linear orders is scattered.

  For the other direction... TODO.
\end{proof}

\begin{notations}
  Let $\mathcal{H}_{< \alpha}$ be the class of linear orders of Hausdorff rank
  $< \alpha$ and $\mathcal{H}_{=\alpha}$ be the class of linear orders of
  Hausdorff rank $= \alpha$.

  Let $\mathcal{B}_{< \alpha}$ be the class of linear orders
  of Hausdorff rank $< \alpha$ on bounded subintervals.

  Let $\mathcal{Q}_{< \alpha} = \setcomp{L : 1 + L \in \mathcal{B}_{< \alpha}}$.

  Let $\mathcal{R}_{< \alpha} = \setcomp{L : L + 1 \in \mathcal{B}_{< \alpha}}$.

  Clearly,
  $\mathcal{H}_{< \alpha}, \mathcal{Q}_{< \alpha}, \mathcal{R}_{< \alpha} \subseteq \mathcal{B}_{< \alpha}$.
\end{notations}

\begin{claim}
  $ \mathcal{Q}_{< \alpha} \cap \mathcal{R}_{< \alpha} = \setcomp{L : 1 + L + 1 \in \mathcal{B}_{< \alpha}}$.
\end{claim}

\begin{proof}
  The $\supseteq$ direction is obvious. The $\subseteq$ direction follows
  from the star property of $\mathcal{B}_{< \alpha}$.
\end{proof}

\begin{lemma}
  Let $L$ be a linear order. Then there exists a largest subinterval $M \subseteq L$ such that
  $x \in M$ and $M \in \mathcal{B}_{< \alpha}$.
\end{lemma}

\begin{definition}
  Let $L$ be a linear order. Let $x \in L$. We define $M_{\alpha}[x]$ to be the largest subinterval
  $M \subseteq L$ such that $x \in M$ and $M \in \mathcal{B}_{< \alpha}$.

  We define $\sim_{\alpha}$ to be the equivalence relation on $L$ such that $x \sim_{\alpha} y$ iff
  $M_{\alpha}[x] = M_{\alpha}[y]$.
\end{definition}

\begin{lemma}
  Let $L$ be a linear order. Let $P, Q, R \subseteq L$ be relations, such that:

  \begin{itemize}
    \item $P$ represents $\sim_{\alpha}$ on $L$.
    \item $Q$ is such that $x \in Q$ iff $M_{\alpha}[x] \in \mathcal{Q}_{< \alpha}$.
    \item $R$ is such that $x \in R$ iff $M_{\alpha}[x] \in \mathcal{R}_{< \alpha}$.
  \end{itemize}

  Then for some linear order $I$ there exists a decomposition
  $L = \sum_{i \in I} L_i$ such that $L_i \in \mathcal{B}_{< \alpha}$ for all $i \in I$,
  $L_i$ is monochromatic with respect to $P$, $Q$ and $R$.

  Furthermore, let $\tau_i$ be the $n$-type of $L_i, p_i, q_i, r_i$ in $\mso[p, q, r]$,
  where $p_i = 1_{L_i \subseteq P}$, $q_i = 1_{L_i \subseteq Q}$ and $r_i = 1_{L_i \subseteq R}$.
  Then the following hold
  \begin{itemize}
    \item if $i$ has a successor, $p(\tau_i) \ne p(\tau_{i+1})$
    \item if $i$ has a successor, either $r(\tau_i) = 0$ or $q(\tau_{i+1}) = 0$
  \end{itemize}
\end{lemma}
\begin{proof}
  Take $I = L / \sim_{\alpha}$.

  Then $L = \sum_{i \in I} L_i$ where $L_i$ is the $\sim_{\alpha}$-equivalence class of $i$.

  Then $L_i$ is monochromatic with respect to $P$, $Q$ and $R$.

  The only thing left to prove is the last two conditions. The first follows from
  the fact that $P$ represents $\sim_{\alpha}$.

  The second follows because if it were not the case, then $L_i$ and $L_{i+1}$ would
  be the same $\sim_{\alpha}$-equivalence class.
\end{proof}

\begin{lemma}
  Let $I$ be a linear order. Let $n \in \NN$. Let $p, q, r$ be boolean variables.

  Let $\tau_i$ be an assignment of satisfiable $n$-types in $\mso[p, q, r]$ for all $i \in I$. Assume that
  \begin{itemize}
    \item if $i$ has a successor, $p(\tau_i) \ne p(\tau_{i+1})$
    \item if $i$ has a successor, either $r(\tau_i) = 0$ or $q(\tau_{i+1}) = 0$
  \end{itemize}

  Then there exists a linear order $L$ and $P, Q, R \subseteq L$ such that:
  \begin{itemize}
    \item $P$ represents $\sim_{\alpha}$ on $L$.
    \item $Q$ is such that $x \in Q$ iff $M_{\alpha}[x] \in \mathcal{Q}_{< \alpha}$.
    \item $R$ is such that $x \in R$ iff $M_{\alpha}[x] \in \mathcal{R}_{< \alpha}$.
  \end{itemize}

  such that for all $i \in I$, $L_i$ is a $\sim_{\alpha}$-equivalence class of $L$,
  and is thus monochromatic with respect to $P$, $Q$ and $R$.

  Furthermore, the $n$-type of $L_i, p_i, q_i, r_i$ in $\mso[p, q, r]$ is $\tau_i$, where
  $p_i = 1_{L_i \subseteq P}$, $q_i = 1_{L_i \subseteq Q}$ and $r_i = 1_{L_i \subseteq R}$,
\end{lemma}

\begin{proof}
  Since $\tau_i$ is satisfiable, we can take $L_i$ to be a linear order of $n$-type
  $\tau_i$ such that:

  \begin{itemize}
    \item If $q(\tau_i) = r(\tau_i) = 1$, then $L_i \in \mathcal{Q}_{< \alpha} \cap \mathcal{R}_{< \alpha}$.
    \item If $q(\tau_i) = 1$ and $r(\tau_i) = 0$, then $L_i \in \mathcal{Q}_{< \alpha} - \mathcal{R}_{< \alpha}$.
    \item If $q(\tau_i) = 0$ and $r(\tau_i) = 1$, then $L_i \in \mathcal{R}_{< \alpha} - \mathcal{Q}_{< \alpha}$.
    \item If $q(\tau_i) = r(\tau_i) = 0$, then $L_i \in \mathcal{B}_{< \alpha} - (\mathcal{Q}_{< \alpha} \cup \mathcal{R}_{< \alpha})$.
  \end{itemize}

  Let $L = \sum_{i \in I} L_i$.
  
  By definition each $L_i$ is in $\mathcal{B}_{< \alpha}$. We need to prove
  that each $L_i$ is a largest $\mathcal{B}_{< \alpha}$-subinterval in $L$.

  On the contrary, suppose that there exist $i' \ne i$ such that $[L_i, L_{i'}] \in \mathcal{B}_{< \alpha}$.
  WLOG, $L_i < L_{i'}$.
  
  Since $I$ is scattered, take some $i \le a < b \le i'$ such that 
  there is no element between $a$ and $b$ in $I$.

  Then $L_a \in \mathcal{R}_{< \alpha}$ and $L_b \in \mathcal{Q}_{< \alpha}$, in contradiction.
\end{proof}

\begin{lemma}
  Let $L$ be a scattered countable linear order.

  Let $J \subseteq L$ be some subinterval in $\mathcal{B}_{< \alpha}$.

  Then $\hrank{J} \le \alpha$.

  Furthermore, $\hrank{J} < \alpha$ iff $J \in \mathcal{Q_{< \alpha}} \cap \mathcal{R_{< \alpha}}$.
\end{lemma}

\begin{proof}
  Let $\setcomp{x_i}_{i \in I} \subseteq J$ be a bidirectional, cofinal, weakly monotone $I$-sequence in $J$, i.e,
  $x_i \le x_j$ if $i \le j$ for $I \subseteq \ZZ$.

  Write $J = \sum_{i \in I} [x_i, x_{i+1}]$. Then every $[x_i, x_{i+1}]$ is of Hausdorff rank $< \alpha$.

  Thus, $\hrank{J} \le \alpha$.

  Suppose $\hrank{J} < \alpha$, then obviously $J \in \mathcal{Q_{< \alpha}} \cap \mathcal{R_{< \alpha}}$.

  Conversely, suppose $J \in \mathcal{Q_{< \alpha}} \cap \mathcal{R_{< \alpha}}$.

  Then $1 + J + 1 \in \mathcal{B}_{< \alpha}$. But it is a bounded interval,
  so $\hrank{1 + J + 1} < \alpha$ and thus $\hrank{J} < \alpha$.
\end{proof}


\begin{lemma}
  Let $J \subseteq L$ be a subinterval.

  Then $\hrank{J} \le \alpha$ iff $J$ is a finite sum of $\mathcal{B}_{< \alpha}$-subintervals.
\end{lemma}

\begin{proof}
  From the previous lemma, it is clear that if $J$ is a finite sum of $\mathcal{B}_{< \alpha}$-subintervals,
  then $\hrank{J} \le \alpha$, since the rank bound is preserved under finite sums.

  Conversely, suppose $\hrank{J} \le \alpha$.

  If $J = \sum_{i \in \ZZ} J_i$ for some $J_i$ of Hausdorff rank $< \alpha$,
  take $x, y \in J$. Then let $x \in J_{i_1}$ and $y \in J_{i_2}$.
  
  Then $[x, y] \subseteq \sum_{i \in [i_1, i_2]} J_i$. But the last sum is of rank $< \alpha$
  and thus $[x, y]$ is of rank $< \alpha$. That is, $J \in \mathcal{B}_{< \alpha}$.

  Since every subinterval of rank $\le \alpha$ is a finite sum of $\ZZ$-sums of intervals of rank $< \alpha$,
  we are done.
\end{proof}

\begin{corollary}
  Let $J \subseteq L$ be a subinterval.

  Then $\hrank{J} \le \alpha$ iff $J$ is a finite sum of largest $\mathcal{B}_{< \alpha}$-subintervals in $L$
\end{corollary}


\begin{lemma}
  Let $\mathcal{C} \in \setcomp{\mathcal{Q}_{< \alpha}, \mathcal{R}_{< \alpha}}$.

  There exists a computable function $f : \NN \to \NN$ such that for all $n \in \NN$,
  for every ordinal $\alpha \ge f(n)$,
  and for every linear order $L$ with $\hrank{L} \ge f(n)$,
  there exists some linear order $L' \in \mathcal{C}$ such that
  $L \equiv_n L'$.
\end{lemma}

\begin{corollary}
  Over scattered with interpretations of $P$, $Q$ and $R$ as above, the properties
  $\hrank{\cdot} \le \alpha$, $\hrank{\cdot} < \alpha$ and $\hrank{\cdot} = \alpha$
  over subintervals are all expressible in $\mso[P, Q, R]$.
\end{corollary}

\begin{proof}
  For $\hrank{\cdot} \le \alpha$ and $\hrank{\cdot} < \alpha$, we can use the previous lemmas.

  For $\hrank{\cdot} = \alpha$, we can use the previous two.
\end{proof}

\begin{theorem}
  There is a an algorithm solving satisfiability for $\mso[P, Q, R]$ over scattered linear orders,
  given an oracle which solves the satisfiability problem for $\mso$ over scattered linear orders.
\end{theorem}

\begin{proof}
  By the decomposition theorem, there exists a translation,
  that given an $\mso[P, Q, R]$ formula $\varphi$ of quantifier-depth $n$.
  outputs an $\mso[\setcomp{X_\tau}_\tau]$ formula $\psi$.

  Let $P_L, Q_L, R_L$ be the interpretations of $P, Q, R$ on $L$.

  Then

  $$
    L, P := P_L, Q := Q_L, R := R_L \models \varphi \iff I, \setcomp{X_\tau := I_\tau}_\tau \models \psi
  $$

  Where $I_\tau = \setcomp{i \in I : L_i \models \tau}$ for every $n$-type $\tau$.

  Let $T$ be the set of $n$-types in $\mso[p, q, r]$ which satisfy
  $q(\tau) = 1 \iff \tau \in \mathcal{Q}_{< \alpha}$ and $r(\tau) = 1 \iff \tau \in \mathcal{R}_{< \alpha}$.

  Let $S = \setcomp{(\tau_1, \tau_2) : p(\tau_1) \ne p(\tau_2) \land (r(\tau_1) = 0 \lor q(\tau_2) = 0)}$.

  Then $T$ and $S$ can be calculated using the oracle.

  Then $\psi$ is an $\mso[T, S]$ formula.

  Then we define an $\mso[p, q, r]$ formula $\psi'$ as follows:

  $\psi'$ claims that there exists a partition (with possible empty sets) $\setcomp{Y_\tau}_{\tau}$ of $I$ such that
  \begin{itemize}
    \item Every $i \in I$ is in some $Y_\tau$ for $\tau \in T$.
    \item If $i' = i+1$ in $I$, then for some $(\tau_1, \tau_2) \in S$, $i \in Y_{\tau_1}$ and $i' \in Y_{\tau_2}$.
  \end{itemize}

  Now we claim that $\varphi$ is satisfiable in some linear order, iff $\psi'$ is satisfiable in some
  linear order.
  
  Suppose $\varphi$ is satisfiable in some linear order $L$.

  Take a decomposition $L = \sum_{i \in I} L_i$ as in lemma 2.

  Then $\psi$ holds over the assignment $X_\tau := I_\tau$. But by lemma 2, this assignment
  satisfies the condition required for $\psi'$ to hold. Then $\psi'$ holds over $I$.

  Conversely, suppose $psi'$ holds in $I$.

  Let $X_\tau := Z_\tau$ be the assignment that is guaranteed by $psi'$.

  Let $tau_i$ be the unique $\tau$ such that $i \in Z_\tau$.

  Then the conditions for lemma 3 are guaranteed.

  Thus, take $L$ as in lemma 3. Then $\psi$ holds over $I$ when we set $X_i := Z_\tau$.
  But $Z_\tau = I_\tau$ for all $\tau$, so $\varphi$ holds over $L$.
\end{proof}

\end{document}

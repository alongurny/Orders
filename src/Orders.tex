\documentclass{article}

\usepackage{amsmath}
\usepackage{amssymb}
\usepackage{amsthm}

\usepackage{todonotes}

% Theorems
\newtheorem{theorem}{Theorem}
\newtheorem{corollary}{Corollary}
\newtheorem{observation}{Observation}
\newtheorem{lemma}{Lemma}
\newtheorem{proposition}{Proposition}
\newtheorem{conjecture}{Conjecture}
\newtheorem{claim}{Claim}

% Definitions
\newtheorem{definition}{Definition}
\newtheorem{definitions}{Definitions}
\newtheorem{notation}{Notation}
\newtheorem{notations}{Notations}

% Delimiters
\newcommand{\parens}[1]{\left( {#1} \right)}
\newcommand{\brackets}[1]{\left[ {#1} \right]}
\newcommand{\braces}[1]{\left\{ {#1} \right\}}

% Macros
\newcommand{\setcomp}[1]{\braces{#1}}
\newcommand{\hrank}[1]{\mathbf{hrank}_{\qq}\left( #1 \right)}
\newcommand{\bigforall}{\mbox{\Large $\mathsurround0pt\forall$}} 

% Letters
\newcommand{\agemo}{-\omega}
\newcommand{\otp}{\mathbf{otp}}
\newcommand{\fso}{\mathbf{FSO}}
\newcommand{\mso}{\mathbf{MSO}}
\newcommand{\bounded}{\mathbf{bounded}}
\newcommand{\pp}{\mathbf{P}}
\newcommand{\qq}{\mathbf{Q}}

% Sets
\newcommand{\NN}{\mathbb{N}}
\newcommand{\ZZ}{\mathbb{Z}}

% Types
\newcommand{\type}[2]{\mathbf{type}_{#1} \brackets{#2}}

\author{Alon Gurny}
\title{Orders}
\date{\today}

\begin{document}

\maketitle

\section{Properties}
% Definition of a property
\begin{definition}
  A \emph{property} $\pp$ of linear orders is a class of linear orders which
  is closed under isomorphism.
\end{definition}

% Definition of a monotone property
\begin{definition}
  A property $\pp$ of linear orders is \emph{monotone} if for every linear order $L$,
  $L \in \pp$ implies that every suborder of $L$ is in $\pp$.
\end{definition}

% Definition of a symmetric property
\begin{definition}
  A property $\pp$ of linear orders is \emph{symmetric} if for every linear order $L$,
  $L \in \pp$ iff $L^R \in \pp$.
\end{definition}

% Definition of an additive property
\begin{definition}
  A property $\pp$ of linear orders is an \emph{additive property} if for every linear orders $L_1$ and $L_2$,
  $L_1 + L_2 \in \pp$ iff $L_1, L_2 \in \pp$.
\end{definition}

% Definition of a star property
\begin{definition}
  A property $\pp$ of linear orders is a \emph{star property} if for every family
  $\mathcal{F}$ of linear orders in $\pp$ such that $\bigcap \mathcal{F} \ne \emptyset$,
  $\bigcup \mathcal{F} \in \pp$.
\end{definition}

% Definition of $\bounded-\pp$
\begin{definition}
  Let $\pp$ be a property of linear orders.

  We define $\bounded-\pp$ to be the class of linear orders $L$ such that for every $x, y \in L$,
  the bounded subinterval $[x, y]$ is in $\pp$.
\end{definition}

% Definition of an almost anti-symmetric property
\begin{definition}
  A property $\pp$ of linear orders is \emph{almost anti-symmetric}
  if for every linear order $L$,
  $L \in \pp$ and $L^R \in \pp$ imply that $L$ is finite.
\end{definition}

% Lemma: $\bounded-\pp$ is a star property
\begin{lemma}
  Let $\pp$ be an additive property of linear orders.

  Then the property $\bounded-\pp$ is has the star property.
\end{lemma}

\section{Hausdorff Rank}

% Definition of the Hausdorff rank
\begin{definition}
  Let $\qq$ be a property of linear orders.

  We define a property $\qq^\alpha$ for every ordinal $\alpha$ as follows:

  Let $L$ be a linear order.

  We define $\hrank{L} \le 0$ iff $L$ is finite.

  Let $\alpha > 0$ be an ordinal.

  We define $\hrank{L} \le \alpha$
  iff $L = \sum_{i \in I} L_i$ for some linear order $I$,
  where $\hrank{L_i} < \alpha$ and $I$ is a finite sum of $1$, $\omega$ and
  $\agemo$.

  We write $\hrank{L} = \alpha$ iff $\alpha$ is the
  least ordinal such that $\hrank{L} \le \alpha$.
\end{definition}

We will be working with countable linear orders.

\begin{claim}
  Let $L$ be a countable linear order.

  Then $\hrank{L}$ is defined iff $L$ is scattered.
\end{claim}

\begin{proof}
  To prove $\implies$ is easy, as a scattered sum of scattered linear orders is scattered.

  For the other direction... TODO.
\end{proof}

\begin{notations}
  Let $\mathcal{H}_{\alpha}$ be the class of linear orders of Hausdorff rank
  $< \alpha$ and $\mathcal{H}_{=\alpha}$ be the class of linear orders of
  Hausdorff rank $= \alpha$.

  Let $\mathcal{B}_{\alpha}$ be the class of linear orders
  of Hausdorff rank $< \alpha$ on bounded subintervals.

  Let $\mathcal{Q}_{\alpha} = \setcomp{L : 1 + L \in \mathcal{B}_{\alpha}}$.

  Let $\mathcal{R}_{\alpha} = \setcomp{L : L + 1 \in \mathcal{B}_{\alpha}}$.

  Clearly,
  $\mathcal{H}_{\alpha}, \mathcal{Q}_{\alpha}, \mathcal{R}_{\alpha} \subseteq \mathcal{B}_{\alpha}$.

  Clearly,
  $\mathcal{H}_{\alpha + 1} = \setcomp{L : \hrank{L} \le \alpha}$.
\end{notations}

\begin{claim}
  The following are equal:

  \begin{enumerate}
    \item $\mathcal{H}_{\alpha}$
    \item $\setcomp{L : 1 + L + 1 \in \mathcal{B}_{\alpha}}$.
    \item $\mathcal{Q}_{\alpha} \cap \mathcal{R}_{\alpha}$
  \end{enumerate}

\end{claim}

\begin{proof}
  The equivalence of 1 and 2 is clear, and obviously
  2 implies 3.

  The other direction (3 implies 2) follows from the star property
  of $\mathcal{B}_{\alpha}$.
\end{proof}

\begin{lemma}
  Let $L$ be a linear order. Then there exists a largest subinterval $M \subseteq L$ such that
  $x \in M$ and $M \in \mathcal{B}_{\alpha}$.
\end{lemma}

\begin{definition}
  Let $L$ be a linear order. Let $x \in L$. We define $M_{\alpha}[x]$ to be the largest subinterval
  $M \subseteq L$ such that $x \in M$ and $M \in \mathcal{B}_{\alpha}$.

  We define $\sim_{\alpha}$ to be the equivalence relation on $L$ such that $x \sim_{\alpha} y$ iff
  $M_{\alpha}[x] = M_{\alpha}[y]$.
\end{definition}

\begin{lemma}
  Let $L$ be a linear order. Let $P, Q, R \subseteq L$ be relations, such that:

  \begin{itemize}
    \item $P$ represents $\sim_{\alpha}$ on $L$.
    \item $Q$ is such that $x \in Q$ iff $M_{\alpha}[x] \in \mathcal{Q}_{\alpha}$.
    \item $R$ is such that $x \in R$ iff $M_{\alpha}[x] \in \mathcal{R}_{\alpha}$.
  \end{itemize}

  Then for some linear order $I$ there exists a decomposition
  $L = \sum_{i \in I} L_i$ such that $L_i \in \mathcal{B}_{\alpha}$ for all $i \in I$,
  $L_i$ is monochromatic with respect to $P$, $Q$ and $R$.

  Furthermore, let $\tau_i$ be the $n$-type of $L_i, p_i, q_i, r_i$ in $\mso[p, q, r]$,
  where $p_i = 1_{L_i \subseteq P}$, $q_i = 1_{L_i \subseteq Q}$ and $r_i = 1_{L_i \subseteq R}$.
  Then the following hold
  \begin{itemize}
    \item if $i$ has a successor, $p(\tau_i) \ne p(\tau_{i+1})$
    \item if $i$ has a successor, either $r(\tau_i) = 0$ or $q(\tau_{i+1}) = 0$
  \end{itemize}
\end{lemma}
\begin{proof}
  Take $I = L / \sim_{\alpha}$.

  Then $L = \sum_{i \in I} L_i$ where $L_i$ is the $\sim_{\alpha}$-equivalence class of $i$.

  Then $L_i$ is monochromatic with respect to $P$, $Q$ and $R$.

  The only thing left to prove is the last two conditions. The first follows from
  the fact that $P$ represents $\sim_{\alpha}$.

  The second follows because if it were not the case, then $L_i$ and $L_{i+1}$ would
  be the same $\sim_{\alpha}$-equivalence class.
\end{proof}

\begin{lemma}
  Let $I$ be a linear order. Let $n \in \NN$. Let $p, q, r$ be boolean variables.

  Let $\tau_i$ be an assignment of satisfiable $n$-types in $\mso[p, q, r]$ for all $i \in I$. Assume that
  \begin{itemize}
    \item if $i$ has a successor, $p(\tau_i) \ne p(\tau_{i+1})$
    \item if $i$ has a successor, either $r(\tau_i) = 0$ or $q(\tau_{i+1}) = 0$
  \end{itemize}

  Then there exists a linear order $L$ and $P, Q, R \subseteq L$ such that:
  \begin{itemize}
    \item $P$ represents $\sim_{\alpha}$ on $L$.
    \item $Q$ is such that $x \in Q$ iff $M_{\alpha}[x] \in \mathcal{Q}_{\alpha}$.
    \item $R$ is such that $x \in R$ iff $M_{\alpha}[x] \in \mathcal{R}_{\alpha}$.
  \end{itemize}

  such that for all $i \in I$, $L_i$ is a $\sim_{\alpha}$-equivalence class of $L$,
  and is thus monochromatic with respect to $P$, $Q$ and $R$.

  Furthermore, the $n$-type of $L_i, p_i, q_i, r_i$ in $\mso[p, q, r]$ is $\tau_i$, where
  $p_i = 1_{L_i \subseteq P}$, $q_i = 1_{L_i \subseteq Q}$ and $r_i = 1_{L_i \subseteq R}$,
\end{lemma}

\begin{proof}
  Since $\tau_i$ is satisfiable, we can take $L_i$ to be a linear order of $n$-type
  $\tau_i$ such that:

  \begin{itemize}
    \item If $q(\tau_i) = r(\tau_i) = 1$, then $L_i \in \mathcal{Q}_{\alpha} \cap \mathcal{R}_{\alpha}$.
    \item If $q(\tau_i) = 1$ and $r(\tau_i) = 0$, then $L_i \in \mathcal{Q}_{\alpha} - \mathcal{R}_{\alpha}$.
    \item If $q(\tau_i) = 0$ and $r(\tau_i) = 1$, then $L_i \in \mathcal{R}_{\alpha} - \mathcal{Q}_{\alpha}$.
    \item If $q(\tau_i) = r(\tau_i) = 0$, then $L_i \in \mathcal{B}_{\alpha} - (\mathcal{Q}_{\alpha} \cup \mathcal{R}_{\alpha})$.
  \end{itemize}

  Let $L = \sum_{i \in I} L_i$.

  By definition each $L_i$ is in $\mathcal{B}_{\alpha}$. We need to prove
  that each $L_i$ is a largest $\mathcal{B}_{\alpha}$-subinterval in $L$.

  On the contrary, suppose that there exist $i' \ne i$ such that $[L_i, L_{i'}] \in \mathcal{B}_{\alpha}$.
  WLOG, $L_i < L_{i'}$.

  Since $I$ is scattered, take some $i \le a < b \le i'$ such that
  there is no element between $a$ and $b$ in $I$.

  Then $L_a \in \mathcal{R}_{\alpha}$ and $L_b \in \mathcal{Q}_{\alpha}$, in contradiction.
\end{proof}

\begin{lemma}
  Let $L$ be a countable linear order.

  Let $J \subseteq L$ be some subinterval in $\mathcal{B}_{\alpha}$.

  Then $\hrank{J} \le \alpha$.

  Furthermore, $\hrank{J} < \alpha$ iff $J \in \mathcal{Q_{< \alpha}} \cap \mathcal{R_{< \alpha}}$.
\end{lemma}

\begin{proof}
  Let $\setcomp{x_i}_{i \in I} \subseteq J$ be a bidirectional, cofinal, weakly monotone $I$-sequence in $J$, i.e,
  $x_i \le x_j$ if $i \le j$ for $I \subseteq \ZZ$.

  Write $J = \sum_{i \in I} [x_i, x_{i+1}]$. Then every $[x_i, x_{i+1}]$ is of Hausdorff rank $< \alpha$.

  Thus, $\hrank{J} \le \alpha$.

  Suppose $\hrank{J} < \alpha$, then obviously $J \in \mathcal{Q_{< \alpha}} \cap \mathcal{R_{< \alpha}}$.

  Conversely, suppose $J \in \mathcal{Q_{< \alpha}} \cap \mathcal{R_{< \alpha}}$.

  Then $1 + J + 1 \in \mathcal{B}_{\alpha}$. But it is a bounded interval,
  so $\hrank{1 + J + 1} < \alpha$ and thus $\hrank{J} < \alpha$.
\end{proof}

\begin{lemma}
  Let $J \subseteq L$ be a subinterval.

  Then $\hrank{J} \le \alpha$ iff $J$ is a finite sum of $\mathcal{B}_{\alpha}$-subintervals.

  Note: this lemma does not work if we take a general $\qq$ property.
\end{lemma}

\begin{proof}
  From the previous lemma, it is clear that if $J$ is a finite sum of $\mathcal{B}_{\alpha}$-subintervals,
  then $\hrank{J} \le \alpha$, since the rank bound is preserved under finite sums.

  Conversely, suppose $\hrank{J} \le \alpha$.

  If $J = \sum_{i \in \ZZ} J_i$ for some $J_i$ of Hausdorff rank $< \alpha$,
  take $x, y \in J$. Then let $x \in J_{i_1}$ and $y \in J_{i_2}$.

  Then $[x, y] \subseteq \sum_{i \in [i_1, i_2]} J_i$. But the last sum is of rank $< \alpha$
  and thus $[x, y]$ is of rank $< \alpha$. That is, $J \in \mathcal{B}_{\alpha}$.

  Since every subinterval of rank $\le \alpha$ is a finite sum of $\ZZ$-sums of intervals of rank $< \alpha$,
  we are done.
\end{proof}

\begin{corollary}
  Let $J \subseteq L$ be a subinterval.

  Then $\hrank{J} \le \alpha$ iff $J$ is a finite sum of largest $\mathcal{B}_{\alpha}$-subintervals in $L$
\end{corollary}

\begin{lemma}
  There exists a global computable function $f : \NN \to \NN$ such that
  for all $n \in \NN$,
  $\type{n}{\mathcal{H}_{f(n) + 1}} = \type{n}{\mathcal{H}_{f(n)}}$.

  Equivalently, every linear order of finite rank is $n$-equivalent to some linear order of rank $\le f(n)$.
\end{lemma}

\begin{proof}
  Since there exist only a finite number of $n$-types,
  and the $\omega$-sequence $\braces{\type{n}{\mathcal{H}_{k}}}_{k \in \omega}$ is monotone,
  it must stabilize at some point.

  This point is computable as a function of $n$, because
  $\type{n}{\mathcal{H}_{k}}$ is computable for every finite $k$.
\end{proof}


\begin{lemma}
  There exist global computable functions $a, b : \NN \to \NN$ such that
  for all $n, c_1, c_2 \in \NN$ such that $c_1, c_2 \ge a(n)$ and $c_1 \equiv c_2 \mod b(n)$,
  $$\type{n}{\mathcal{H}_{c_1}} = \type{n}{\mathcal{H}_{c_2}}$$

  Equivalently, the sequence $\braces{\type{n}{\mathcal{H}_{k}}}_{k \in \omega}$ 
  is ultimately periodic for all $n \in \NN$. Furthermore, the starting point and the period
  itself can be computed as a function of $n$.
\end{lemma}

\begin{proof}
  Let $n \in \NN$.

  Since there exist only a finite number of possible
  sets of $n$-types, there exist (and can be computed)
  some $a(n) > f(n)$, $a(n) + b(n)$ such that
  
  $$\type{n}{\mathcal{H}_{a(n)}} = \type{n}{\mathcal{H}_{a(n) + b(n)}}$$
  
  We shall prove by induction that for all $c \ge a(n) + b(n)$,

  $$\type{n}{\mathcal{H}_{c}} = \type{n}{\mathcal{H}_{c + b(n)}}$$

  This will complete the proof.

  The base case $c = a(n)$ has been proven in the beginning.

  Suppose the induction hypothesis holds for $c$.

  Let $L$ be of rank $c + 1$.

  Write $L = \sum_{i \in I} L_i$ where $\hrank{L_i} < c + 1$,
  and $\hrank{L_i} = c$ infinitely many times.

  By the induction hypothesis,
  if $\hrank{L_i} = c$, we can find $N_i \equiv_n L_i$ with $\hrank{N_i} = c + b(n)$.
  Setting $N_i := L_i$ for all other $i$, we conclude that $N := \sum_{i \in I} N_i$
  is $n$-equivalent to $L$.

  However, clearly $\hrank{N} = c + b(n) + 1$. So overall,
  $$\type{n}{\mathcal{H}_{c + 1}} \subseteq \type{n}{\mathcal{H}_{c + b(n) + 1}}$$

  Conversely, suppose $L$ is of rank $c + b(n) + 1$.
  Write $L$ = $\sum_{i \in I} L_i$ where $\hrank{L_i} < c + b(n) + 1$,
  and $\hrank{L_i} = c + b(n)$ infinitely many times.

  By the induction hypothesis,
  we can find for all $i$ such that $\hrank{L_i} = c + b(n)$ some 
  $N_i \equiv_n L_i$ with $\hrank{N_i} = c$.
  Furthermore, since $c \ge a(n) > f(n)$, we can
  find $N_i \equiv_n L_i$ with $\hrank{N_i} \le f(n) < c$ for all other $i$.

  We conclude that $N := \sum_{i \in I} N_i$ is $n$-equivalent to $L$.
  However, clearly $\hrank{N} = c + 1$. So overall,
  $$\type{n}{\mathcal{H}_{c + b(n) + 1}} \subseteq \type{n}{\mathcal{H}_{c + 1}}$$

  So we have proven the induction step, and the lemma follows.
\end{proof}

\begin{lemma}
  Let $n \in \NN$, and let $\alpha \ge \omega$ be an ordinal.

  Then,
  $$\type{n}{\mathcal{H}_{\alpha}} = \bigcup_{c < b(n)}{\type{n}{\mathcal{H}_{c + b(n)}}}$$

  In particular, $\type{n}{\mathcal{H}_{\alpha}}$ can be
  computed, and is independent of the choice $\alpha \ge \omega$.
\end{lemma}

\begin{proof}
  TBC.
\end{proof}

\begin{proof}
  By induction on $\alpha \ge f(n)$
  suppose that for all $f(n) \le \beta < \alpha$,
  $$\type{n}{\mathcal{H}_{\beta + 1}} = \type{n}{\mathcal{H}_{f(n)}}$$

  Let $L$ be a scattered linear order of rank $\alpha$.

  Then $L = \sum_{i \in I} L_i$ where $\hrank{L_i} < \alpha$.
  By the induction hypothesis, we can find $N_i \equiv_n L_i$ with $\hrank{N_i} < f(n)$.

  Let $N = \sum_{i \in I} N_i$. Then $L \equiv_n N$.
  
  Additionally, $\hrank{N} < f(n) + 1$ by the rank definition. However, that means that 
  we can find $N' \equiv_n N$ with $\hrank{N'} \le f(n)$ by the definition of $f(n)$.
\end{proof}

\begin{corollary}
  The following sequences stabilize at $f(n)$:

  \begin{itemize}
    \item $\type{n}{\mathcal{H}_{\alpha}}$
    \item $\type{n}{\mathcal{B}_{\alpha}}$
    \item $\type{n}{\mathcal{Q}_{\alpha}}$
    \item $\type{n}{\mathcal{R}_{\alpha}}$
    \item $\type{n}{\mathcal{Q}_{\alpha} - \mathcal{R}_{\alpha}}$
    \item $\type{n}{\mathcal{R}_{\alpha} - \mathcal{Q}_{\alpha}}$
    \item $\type{n}{\mathcal{B}_{\alpha} - (\mathcal{Q}_{\alpha} \cup \mathcal{R}_{\alpha})}$
  \end{itemize}
\end{corollary}

\begin{proof}
  Let $A_k$ be the set of all satisfiable $n$-types of rank $< k$.
  Then $A_{k+1}$ is the closure of $A_k$ under finite sums
  of $\subseteq \ZZ$-sums.

  The sequence $A_0 \subseteq A_1 \subseteq \ldots$ stabilizes at some point.
  Suppose $A_{f(n)} = A_{f(n) + 1}$.

  Suppose $L$ has rank $\beta \ge f(n)$.

  Write $L = \sum_{i \in I} L_i$ where $\hrank{L_i} < \beta$, and
  $I$ is a finite sum of $\subseteq \ZZ$.

  If $\beta$ is a limit ordinal, then there must be
  a bi-cofinal sequence $i_{k}$ such that $\hrank{L_{i_k}} \to \beta$.

  If $\beta$ is a successor ordinal, then $\hrank{L_i} = \beta - 1$ must hold
  infinitely many times.

  Now we proceed by induction on $\alpha \ge f(n)$.

  \begin{enumerate}
    \item If $\mathcal{C} = \mathcal{H}_{\alpha}$, we take $L' \in A_{f(n)}$,
          which necessarily has rank $< f(n) \le \alpha$.
    \item If $\mathcal{C} = \mathcal{Q}_{\alpha} - \mathcal{R}_{\alpha}$,
          we take an $\omega$-sequence $\alpha_k$ such that $\alpha_k \to \alpha$
          (if $\alpha$ is a limit ordinal) or $\alpha_k = \alpha-1$
          (if $\alpha$ is a successor ordinal).

          Then we take $L' = \sum_{i \in \omega} L'_i$ where $\hrank{L'_{i_k}} = \alpha_k$
          (and $\hrank{L'_i} = \hrank{L_i}$ for every other $i$).
          Then $L' \in \mathcal{Q}_{\alpha} - \mathcal{R}_{\alpha}$,
          but also $L' \equiv_n L$.
    \item This is just the same with $\agemo$ instead of $\omega$.
    \item This is just the same with $\ZZ$ instead of $\agemo$.
  \end{enumerate}
\end{proof}

\begin{corollary}
  Over countable linear orders with interpretations of $P$, $Q$ and $R$ as above, the properties
  $\hrank{\cdot} \le \alpha$, $\hrank{\cdot} < \alpha$ and $\hrank{\cdot} = \alpha$
  over subintervals are all expressible in $\mso[P, Q, R]$.
\end{corollary}

\begin{proof}
  For $\hrank{\cdot} \le \alpha$ and $\hrank{\cdot} < \alpha$, we can use the previous lemmas.

  For $\hrank{\cdot} = \alpha$, we can use the previous two.
\end{proof}

\begin{theorem}
  There is a an algorithm solving satisfiability for $\mso[P, Q, R]$ over countable linear orders,
  given an oracle which solves the satisfiability problem for $\mso$ over countable linear orders.
\end{theorem}

\begin{proof}
  By the decomposition theorem, there exists a translation,
  that given an $\mso[P, Q, R]$ formula $\varphi$ of quantifier-depth $n$.
  outputs an $\mso[\setcomp{X_\tau}_\tau]$ formula $\psi$.

  Let $P_L, Q_L, R_L$ be the interpretations of $P, Q, R$ on $L$.

  Then

  $$
    L, P := P_L, Q := Q_L, R := R_L \models \varphi \iff I, \setcomp{X_\tau := I_\tau}_\tau \models \psi
  $$

  Where $I_\tau = \setcomp{i \in I : L_i \models \tau}$ for every $n$-type $\tau$.

  Let $T$ be the set of $n$-types in $\mso[p, q, r]$ which satisfy
  $q(\tau) = 1 \iff \tau \in \mathcal{Q}_{\alpha}$ and $r(\tau) = 1 \iff \tau \in \mathcal{R}_{\alpha}$.

  Let $S = \setcomp{(\tau_1, \tau_2) : p(\tau_1) \ne p(\tau_2) \land (r(\tau_1) = 0 \lor q(\tau_2) = 0)}$.

  Then $T$ and $S$ can be calculated using the oracle.

  Then $\psi$ is an $\mso[T, S]$ formula.

  Then we define an $\mso[p, q, r]$ formula $\psi'$ as follows:

  $\psi'$ claims that there exists a partition (with possible empty sets) $\setcomp{Y_\tau}_{\tau}$ of $I$ such that
  \begin{itemize}
    \item Every $i \in I$ is in some $Y_\tau$ for $\tau \in T$.
    \item If $i' = i+1$ in $I$, then for some $(\tau_1, \tau_2) \in S$, $i \in Y_{\tau_1}$ and $i' \in Y_{\tau_2}$.
  \end{itemize}

  Now we claim that $\varphi$ is satisfiable in some linear order, iff $\psi'$ is satisfiable in some
  linear order.

  Suppose $\varphi$ is satisfiable in some linear order $L$.

  Take a decomposition $L = \sum_{i \in I} L_i$ as in lemma 2.

  Then $\psi$ holds over the assignment $X_\tau := I_\tau$. But by lemma 2, this assignment
  satisfies the condition required for $\psi'$ to hold. Then $\psi'$ holds over $I$.

  Conversely, suppose $\psi'$ holds in $I$.

  Let $X_\tau := Z_\tau$ be the assignment that is guaranteed by $\psi'$.

  Let $tau_i$ be the unique $\tau$ such that $i \in Z_\tau$.

  Then the conditions for lemma 3 are guaranteed.

  Thus, take $L$ as in lemma 3. Then $\psi$ holds over $I$ when we set $X_i := Z_\tau$.
  But $Z_\tau = I_\tau$ for all $\tau$, so $\varphi$ holds over $L$.
\end{proof}

\end{document}

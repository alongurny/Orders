\section{Multiple Ordinals}
We can extend the results from the previous section to multiple ordinals.

\begin{lemma}\label{ho-decomposition-single-ordinal}
  Let $\alpha > 0$ be an ordinal
  and let $\delta \ge \om$ be a limit ordinal.

  Then,
  \[
    \ho{< \alpha + \delta} \bs{\alpha} = \bigcup_{I \in \mathbf{Good}_{\alpha} \wedge \ho{< \delta}} \sum_{i \in I} \sigma_{\alpha}(i)
  \]

\end{lemma}

\begin{proof}
  ($\subseteq$) Let $M \in \ho{< \alpha + \delta} \bs{\alpha}$. By definition, $M$
  is a linear order labeled with $P_\alpha, L_\alpha, R_\alpha$ such that the
  underlying order is in $\ho{< \alpha + \delta}$, $P_\alpha$ represents
  $\sim_\alpha$, and $L_\alpha, R_\alpha$ are as defined.

  Let $I = M / \sim_\alpha$ be the quotient of $M$ by the equivalence relation
  $\sim_\alpha$. Then $M = \sum_{i \in I} M_i$, where $M_i$ are the
  $\sim_\alpha$-equivalence classes. By the definition of $\sim_\alpha$, each
  $M_i \in \bounded{\ho{< \alpha}}$, and by the definition of $\sigma_\alpha(i)$,
  $M_i \in \sigma_\alpha(i)$.

  Since $M \in \ho{< \alpha + \delta}$, the quotient $I$ is in $\ho{< \delta}$.
  For each pair $i, i'$ of consecutive elements in $I$, the labeling ensures that
  $P_\alpha(i) \ne P_\alpha(i')$ and either $R_\alpha(i) = 0$ or $L_\alpha(i') = 0$,
  so $I \in \mathbf{Good}_\alpha$. Thus, $M \in \sum_{i \in I} \sigma_\alpha(i)$ for
  some $I \in \mathbf{Good}_\alpha \wedge \ho{< \delta}$.

  ($\supseteq$) Let $M = \sum_{i \in I} M_i$ where $I \in \mathbf{Good}_\alpha
    \wedge \ho{< \delta}$ and $M_i \in \sigma_\alpha(i)$ for each $i \in I$. The
  labeling $P_\alpha, L_\alpha, R_\alpha$ on $M$ is as required by the definition
  of $\mathbf{Good}_\alpha$, and each $M_i \in \bounded{\ho{< \alpha}}$. Since
  $I \in \ho{< \delta}$, $M \in \ho{< \alpha + \delta}$. Thus,
  $M \in \ho{< \alpha + \delta} \bs{\alpha}$.

  Therefore,
  \[
    \ho{< \alpha + \delta} \bs{\alpha} =
    \bigcup_{I \in \mathbf{Good}_{\alpha} \wedge \ho{< \delta}}
    \sum_{i \in I} \sigma_{\alpha}(i)
  \]
\end{proof}

\begin{corollary}\label{rmj-decomposition-single-ordinal}
  Let $\alpha > 0$ be an ordinal
  and let $\delta \ge \om$ be a limit ordinal.

  Then,
  \[
    \rmj{\alpha + \delta} \bs{\alpha} = \bigcup_{I \in \mathbf{Good}_{\alpha} \wedge \rmj{\delta}}
    \sum_{i \in I} \sigma_{\alpha}(i)
  \]
\end{corollary}

\begin{proof}
  It follows from~\cref{ho-decomposition-single-ordinal}
  together with~\cref{rmj-decomposition}.
\end{proof}

The first lemma we shall use is the "extension lemma".

\begin{lemma}\label{extend-lemma-single-ordinal}
  Let $\alpha$ be an ordinal.

  Let $I$ be a linear order
  and let $\set{M_i}_{i \in I}$ be a family of linear orders,
  such that for each pair $i, i' \in I$
  such that $i'$ is the successor of $i$ in $I$,
  either $\mathbf{R}_\alpha(M_{i}) = 0$ or $\mathbf{L}_\alpha(M_{i'}) = 0$.

  Then,

  \[
    \ps { \sum_{i \in I} M_i } \bs{\alpha} = \sum_{i \in I} \ps { M_i \bs{\alpha} }
  \]
\end{lemma}

\begin{proof}
  It is obvious, but TBC.
\end{proof}

\begin{notation}
  Let $\alpha_1 < \ldots < \alpha_k$ be ordinals.

  Let $\pp$ be a class of linear orders.

  Then,
  \[
    \pp \bs{\alpha_1, \ldots, \alpha_k} := \pp \bs{\alpha_1} \cdots \bs{\alpha_k}
  \]
\end{notation}

\begin{corollary}\label{extend-lemma-multiple-ordinals}
  Let $\alpha_1 < \ldots < \alpha_k < \alpha$ be ordinals.

  Let $I$ be a linear order
  and let $\set{M_i}_{i \in I}$ be a family of linear orders,
  such that for each pair $i, i' \in I$
  such that $i'$ is the successor of $i$ in $I$,
  either $\mathbf{R}_\alpha(M_{i}) = 0$ or $\mathbf{L}_\alpha(M_{i'}) = 0$.

  Then,
  \[
    \ps { \sum_{i \in I} M_i } \bs{\alpha_1, \ldots, \alpha_k, \alpha}
    = \sum_{i \in I} \ps { M_i \bs{\alpha_1, \ldots, \alpha_k, \alpha} }
  \]
\end{corollary}

\begin{proof}
  If $\mathbf{R}_\alpha(i) = 0$, then in particular
  $\mathbf{R}_{\alpha_j}(i) = 0$ for all $j \in \bs{k}$,
  and similarly for $\mathbf{L}_\alpha(i')$.

  So the condition for $\alpha$ implies
  the similar conditions for $\alpha_1, \ldots, \alpha_k$.

  Now, we can apply~\cref{extend-lemma-single-ordinal} inductively
  to obtain the result.
\end{proof}

\begin{lemma}\label{cnt-decomposition-multiple-ordinals}
  Let $\alpha_1 < \ldots < \alpha_k < \alpha$ be ordinals. Then,
  \[
    \cnt \bs{\alpha_1, \ldots, \alpha_k, \alpha}
    = \bigcup_{I \in \mathbf{Good}_{\alpha}} \sum_{i \in I} \sigma_{\alpha}(i) \bs{\alpha_1, \ldots, \alpha_k}
  \]
\end{lemma}

\begin{proof}
  This is a consequence of~\cref{cnt-decomposition-single-ordinal}
  and~\cref{extend-lemma-multiple-ordinals}.
\end{proof}

\begin{lemma}\label{ho-decomposition-multiple-ordinals}
  Let $\alpha_1 < \ldots < \alpha_k < \alpha$ and $\delta > 1$ be ordinals. Then,
  \[
    \ho{< \alpha + \delta} \bs{\alpha_1, \ldots, \alpha_k, \alpha}
    = \bigcup_{I \in \mathbf{Good}_{\alpha} \wedge \ho{< \delta}}
    \sum_{i \in I} \sigma_{\alpha}(i) \bs{\alpha_1, \ldots, \alpha_k}
  \]
\end{lemma}

\begin{proof}
  This is a consequence of~\cref{ho-decomposition-single-ordinal}
  and~\cref{extend-lemma-multiple-ordinals}.
\end{proof}

\begin{lemma}\label{rmj-decomposition-multiple-ordinals}
  Let $\alpha_1 < \ldots < \alpha_k < \alpha$ and $\delta$ be ordinals.
  Then,
  \[
    \rmj{\alpha + \delta} \bs{\alpha_1, \ldots, \alpha_k, \alpha}
    = \bigcup_{I \in \mathbf{Good}_{\alpha} \wedge \rmj{\delta}}
    \sum_{i \in I} \sigma_{\alpha}(i) \bs{\alpha_1, \ldots, \alpha_k}
  \]
\end{lemma}

\begin{proof}
  This is a consequence of~\cref{rmj-decomposition-single-ordinal}
  and~\cref{extend-lemma-multiple-ordinals}.
\end{proof}

\begin{lemma}\label{cnt-decidable-multiple-ordinals}
  Let $\alpha_1 < \ldots < \alpha_k < \alpha$ be ordinals.

  Then the $\mso$-theory of $\cnt \bs{\alpha_1, \ldots, \alpha_k, \alpha}$
  is decidable.
\end{lemma}

\begin{proof}
  Since $\mathbf{Good}_{\alpha}$ is computable,
  it follows from combining~\cref{computable-sum}
  with~\cref{cnt-decomposition-multiple-ordinals} and the computability
  of $\ho{< \alpha} \bs{\alpha_1, \ldots, \alpha_k}$,
  $\rmj{\alpha} \bs{\alpha_1, \ldots, \alpha_k}$,
  $\lmj{\alpha} \bs{\alpha_1, \ldots, \alpha_k}$ and
  $\bmj{\alpha} \bs{\alpha_1, \ldots, \alpha_k}$.
\end{proof}

\begin{theorem}\label{multiple-ordinals-satisfiability}
  Let $\alpha_1 < \ldots < \alpha_k$ be ordinals.

  Satisfiability of $\mso[\Int{\ho{< \alpha_1}}, \ldots, \Int{\ho{< \alpha_k}}]$
  over $\cnt$ is decidable.
\end{theorem}

\begin{proof}
  First, by~\cref{int-expressible}, we can convert
  any formula $\varphi$ in \[
    \mso[\Int{\ho{< \alpha_1}}, \ldots, \Int{\ho{< \alpha_k}}]
  \]
  to a formula $\varphi'$ in \[
    \mso \bs{P_{\alpha_1}, L_{\alpha_1}, R_{\alpha_1}, \ldots, P_{\alpha_k}, L_{\alpha_k}, R_{\alpha_k}}
  \]
  such that $\varphi$ is satisfiable over $\cnt$
  iff $\varphi'$ is satisfiable over $\cnt \bs{\alpha_1, \ldots, \alpha_k}$.

  This is decidable by~\cref{cnt-decidable-multiple-ordinals} and~\cref{cnt-decomposition-multiple-ordinals}.
\end{proof}

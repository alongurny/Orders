
\section{General Hausdorff Rank}

% Definition of the Hausdorff rank
\begin{definition}
  Let $\qq$ be a good property of linear orders.

  We define a property $\qq^{< \alpha}$
  for every ordinal $\alpha > 0$ as follows:

  \begin{itemize}
    \item For $\alpha = 1$, $\qq^{< 1} = \mathbf{Finite}$.
    \item For $\alpha = \gamma + 1$,
          \[\qq^{< \alpha} = \sum_{\qq}{\qq^{< \gamma}}\].
    \item For $\alpha$ a limit ordinal,
          \[\qq^{< \alpha} = \bigcup_{\beta < \alpha} \qq^{< \beta}\].
  \end{itemize}

  We define further $\qq^{\le \alpha} = \qq^{< \alpha + 1}$
  and $\qq^{= \alpha} = \qq^{\le \alpha} - \qq^{< \alpha}$.
  
  We call such an $\alpha$, if it exists,
  \emph{the $\qq$-rank} of a linear order $M$.
\end{definition}

\begin{observations}
  Let $\alpha, \beta > 0$ be ordinals.

  Let $\qq$ be a good property.

  We claim the following without proof:

  \begin{itemize}
    \item $\qq^{\le 1} = \qq$.
    \item $\qq^{< \alpha}$ is a good property.
    \item $\qq^{< \alpha} \subsetneq \qq^{< \beta}$ iff $\alpha < \beta$.
  \end{itemize}
\end{observations}

\begin{lemma}
  Over countable linear orders,
  \[\bounded{\qq^{< \alpha}} \subseteq \qq^{\le \alpha}\]
\end{lemma}

\begin{proof}
  Let $M \in \bounded{\qq^{< \alpha}}$ be a countable linear order.

  Since $M$ is countable, there exists a $\ZZ$-sequence $\set{x_k}_{k \in \ZZ} \subseteq M$,
  which is weakly monotone bi-directionally cofinal in $M$.

  That is, $M = \sum_{k \in \ZZ} \left[ x_k, x_{k + 1} \right)$.

  Taking only those $k$ for which $x_k < x_{k + 1}$, we get a decomposition
  $M = \sum_{i \in I} M_i$ where $I \subseteq \ZZ$.

  Since $\ZZ \in \qq$, and 
  since the rank of every interval $\left[ x_k, x_{k + 1} \right)$ is $< \alpha$,
  $M \in \sum_{\qq}{\qq^{< \alpha}} = \qq^{\le \alpha}$.

\end{proof}


\begin{lemma}
  Let $\alpha > 0$, $\delta \ge 0$ be ordinals. Let $\qq$ be a good property.

  Then $\qq^{\le \alpha + \delta} = \sum_{\qq^{< \delta}}{\qq^{\le \alpha}}$.
\end{lemma}

\begin{proof}
  We prove by induction on $\delta \ge 0$.

  For $\delta = 0$, we need to show that $\qq^{\le \alpha} = \sum_{\mathbf{Finite}}\qq^{\le \alpha}$,
  which follows since $\qq^{\le \alpha}$ is an additive property.

  For $\delta = \varepsilon + 1$, we have
  $
    \qq^{\le \alpha + \delta}
    = \qq^{\le \alpha + \varepsilon + 1}
    = \sum_{\qq}{\qq^{\le \alpha + \varepsilon}}
  $.
  
  By the induction hypothesis,
  $
    \qq^{\le \alpha + \varepsilon}
    = \sum_{\qq^{< \varepsilon}}{\qq^{\le \alpha}}
  $, and thus we get
  \[
    \qq^{\le \alpha + \delta} = \sum_{\qq}{\sum_{\qq^{< \varepsilon}}{\qq^{\le \alpha}}}
  \].

  By associativity, 
  \[\qq^{\le \alpha + \delta}
  = \sum_{\sum_{\qq}{\qq^{< \varepsilon}}}{\qq^{\le \alpha}}
  =\sum_{\qq^{< \delta}}{\qq^{\le \alpha}}\]

  For $\delta$ a limit ordinal, note that $\alpha + \delta = \sup_{\varepsilon < \delta}{\alpha + \varepsilon}$.

  Then,
  \[
    \qq^{\le \alpha + \delta}
    = \bigcup_{\varepsilon < \delta}{\qq^{\le \alpha + \varepsilon}}
    = \bigcup_{\varepsilon < \delta}{\sum_{\qq^{< \varepsilon}}{\qq^{\le \alpha}}}
    = \sum_{\bigcup_{\varepsilon < \delta}{\qq^{< \varepsilon}}}{\qq^{\le \alpha}}
    = \sum_{\qq^{< \delta}}{\qq^{\le \alpha}}
  \]

  where the second equality follows by the induction hypothesis.

\end{proof}